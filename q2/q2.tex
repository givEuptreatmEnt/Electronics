\documentclass[12pt, a4paper]{article}

%%%%%%%%%%%%%%%紙張大小設定%%%%%%%%%%%%%%%
% \paperwidth=65cm
% \paperheight=160cm

%%%%%%%%%%%%%%%引入Package%%%%%%%%%%%%%%%
\usepackage[margin=3cm]{geometry} % 上下左右距離邊緣2cm
\usepackage{mathtools,amsthm,amssymb} % 引入 AMS 數學環境
\usepackage{yhmath}      % math symbol
\usepackage{graphicx}    % 圖形插入用
\usepackage{fontspec}    % 加這個就可以設定字體
\usepackage{type1cm}	 % 設定fontsize用
\usepackage{titlesec}   % 設定section等的字體
\usepackage{titling}    % 加強 title 功能
\usepackage{fancyhdr}   % 頁首頁尾
\usepackage{tabularx}   % 加強版 table
\usepackage[square, comma, numbers, super, sort&compress]{natbib}
% cite加強版
\usepackage[unicode=true, pdfborder={0 0 0}, bookmarksdepth=-1]{hyperref}
% ref加強版
\usepackage[usenames, dvipsnames]{color}  % 可以使用顏色
\usepackage[shortlabels, inline]{enumitem}  % 加強版enumerate
\usepackage{xpatch}

% \usepackage{tabto}      % tab
% \usepackage{soul}       % highlight
% \usepackage{ulem}       % 字加裝飾
%\usepackage{wrapfig}     % 文繞圖
%\usepackage{floatflt}    % 浮動 figure
\usepackage{float}       % 浮動環境
\usepackage{caption}    % caption 增強
\usepackage{subcaption}    % subfigures
% \usepackage{setspace}    % 控制空行
% \usepackage{mdframed}   % 可以加文字方框
% \usepackage{multicol}   % 多欄
\usepackage{siunitx}[config, abbreviations]      % SI unit
% \usepackage{dsfont}     % more mathbb

%%%%%%%%%%%%%%%%%%%TikZ%%%%%%%%%%%%%%%%%%%%%%
% \usepackage{tikz}
\usepackage[siunitx, americanvoltages, americancurrents]{circuitikz}

%%%%%%%%%%%%%%中文 Environment%%%%%%%%%%%%%%%
\usepackage[CheckSingle, CJKmath]{xeCJK}  % xelatex 中文
\usepackage{CJKulem}	% 中文字裝飾
\setCJKmainfont[BoldFont=cwTeX Q Hei]{cwTeX Q Ming}
\setCJKsansfont[BoldFont=cwTeX Q Hei]{cwTeX Q Ming}
\setCJKmonofont[BoldFont=cwTeX Q Hei]{cwTeX Q Ming}
% 設定中文為系統上的字型,而英文不去更動,使用原TeX字型

% \XeTeXlinebreaklocale "zh"             %這兩行一定要加,中文才能自動換行
% \XeTeXlinebreakskip = 0pt plus 1pt     %這兩行一定要加,中文才能自動換行

%%%%%%%%%%%%%%%字體大小設定%%%%%%%%%%%%%%%
% \def\normalsize{\fontsize{10}{15}\selectfont}
% \def\large{\fontsize{40}{60}\selectfont}
% \def\Large{\fontsize{50}{75}\selectfont}
% \def\LARGE{\fontsize{90}{20}\selectfont}
% \def\huge{\fontsize{34}{51}\selectfont}
% \def\Huge{\fontsize{38}{57}\selectfont}

%%%%%%%%%%%%%%%Theme Input%%%%%%%%%%%%%%%%
% \input{themes/chapter/neat}
% \input{themes/env/problist}

%%%%%%%%%%%titlesec settings%%%%%%%%%%%%%%
% \titleformat{\chapter}{\bf\Huge}
            % {\arabic{section}}{0em}{}
% \titleformat{\section}{\centering\Large}
            % {\arabic{section}}{0em}{}
% \titleformat{\subsection}{\large}
            % {\arabic{subsection}}{0em}{}
% \titleformat{\subsubsection}{\bf\normalsize}
            % {\arabic{subsubsection}}{0em}{}
% \titleformat{command}[shape]{format}{label}
            % {編號與標題距離}{before}[after]

%%%%%%%%%%%%variable settings%%%%%%%%%%%%%%
% \numberwithin{equation}{section}
% \setcounter{secnumdepth}{4}  %章節標號深度
% \setcounter{tocdepth}{1}  %目錄深度
% \graphicspath{{images/}}  % 搜尋圖片目錄

%%%%%%%%%%%%%%%頁面設定%%%%%%%%%%%%%%%
\newcolumntype{C}[1]{>{\centering\arraybackslash}p{#1}}
\setlength{\headheight}{15pt}  %with titling
\setlength{\droptitle}{-1.5cm} %title 與上緣的間距
% \posttitle{\par\end{center}} % title 與內文的間距
\parindent=24pt %設定縮排的距離
% \parskip=1ex  %設定行距
% \pagestyle{empty}  % empty: 無頁碼
% \pagestyle{fancy}  % fancy: fancyhdr

% use with fancygdr
% \lhead{\leftmark}
% \chead{}
% \rhead{}
% \lfoot{}
% \cfoot{}
% \rfoot{\thepage}
% \renewcommand{\headrulewidth}{0.4pt}
% \renewcommand{\footrulewidth}{0.4pt}

% \fancypagestyle{firststyle}
% {
  % \fancyhf{}
  % \fancyfoot[C]{\footnotesize Page \thepage\ of \pageref{LastPage}}
  % \renewcommand{\headrule}{\rule{\textwidth}{\headrulewidth}}
% }

%%%%%%%%%%%%%%%重定義一些command%%%%%%%%%%%%%%%
\renewcommand{\contentsname}{目錄}  %設定目錄的標題名稱
\renewcommand{\refname}{參考資料}  %設定參考資料的標題名稱
\renewcommand{\abstractname}{\LARGE Abstract} %設定摘要的標題名稱

%%%%%%%%%%%%%%%特殊功能函數符號設定%%%%%%%%%%%%%%%
% \newcommand{\citet}[1]{\textsuperscript{\cite{#1}}}
\DeclarePairedDelimiter{\abs}{\lvert}{\rvert}
\DeclarePairedDelimiter{\norm}{\lVert}{\rVert}
\DeclarePairedDelimiter{\inpd}{\langle}{\rangle} % inner product
\DeclarePairedDelimiter{\ceil}{\lceil}{\rceil}
\DeclarePairedDelimiter{\floor}{\lfloor}{\rfloor}
\DeclareMathOperator{\adj}{adj}
\DeclareMathOperator{\sech}{sech}
\DeclareMathOperator{\csch}{csch}
\DeclareMathOperator{\arcsec}{arcsec}
\DeclareMathOperator{\arccot}{arccot}
\DeclareMathOperator{\arccsc}{arccsc}
\DeclareMathOperator{\arccosh}{arccosh}
\DeclareMathOperator{\arcsinh}{arcsinh}
\DeclareMathOperator{\arctanh}{arctanh}
\DeclareMathOperator{\arcsech}{arcsech}
\DeclareMathOperator{\arccsch}{arccsch}
\DeclareMathOperator{\arccoth}{arccoth}
\newcommand{\np}[1]{\\[{#1}] \indent}
\newcommand{\transpose}[1]{{#1}^\mathrm{T}}
%%%% Geometry Symbol %%%%
\newcommand{\degree}{^\circ}
\newcommand{\Arc}[1]{\wideparen{{#1}}}
\newcommand{\Line}[1]{\overleftrightarrow{{#1}}}
\newcommand{\Ray}[1]{\overrightarrow{{#1}}}
\newcommand{\Segment}[1]{\overline{{#1}}}

%%%% SI unit short cut %%%%
\newcommand{\siua}{\micro\ampere}
\newcommand{\sima}{\milli\ampere}
\newcommand{\simv}{\milli\volt}
\newcommand{\siko}{\kilo\ohm}
\newcommand{\sio}{\ohm}
\newcommand{\sia}{\ampere}
\newcommand{\siv}{\volt}

%%%%%%%%%%%%%%%證明、結論、定義等等的環境%%%%%%%%%%%%%%%
\renewcommand{\proofname}{\bf 證明:} %修改Proof 標頭
\newtheoremstyle{mystyle}% 自定義Style
  {6pt}{15pt}%       上下間距
  {}%               內文字體
  {}%               縮排
  {\bf}%            標頭字體
  {.}%              標頭後標點
  {1em}%            內文與標頭距離
  {}%               Theorem head spec (can be left empty, meaning 'normal')

% 改用粗體,預設 remark style 是斜體
\theoremstyle{mystyle}	% 定理環境Style
\newtheorem{theorem}{定理}
\newtheorem{definition}{定義}
\newtheorem{formula}{公式}
\newtheorem{condition}{條件}
\newtheorem{supposition}{假設}
\newtheorem{conclusion}{結論}
\newtheorem{lemma}{引理}
\newtheorem{property}{性質}

%% Label set %%
\captionsetup[figure]{labelsep=period}

%% Ans %%
\newcommand{\Ans}{\noindent{\bf Ans:}}

%% No Indent %%
\setlength\parindent{0pt}

%% Note that for label to be correct, compile more than 2 times is needed ... %%


%%%%%%%%%%%%%%%Title的資訊%%%%%%%%%%%%%%%
\title{} %標題
\author{} %作者
\date{} %日期

\begin{document}
% \maketitle %製作tilte page
% \thispagestyle{empty}  %去除頁碼
% \thispagestyle{fancy}  %使用fancyhdr
% \tableofcontents %目錄
%%%%%%%%%%%%%%%%%%%include file here%%%%%%%%%%%%%%%%%%%%%%%%%
% 7: 9
\section{3.7}
Assuming that the diodes in the circuits of Fig.~\ref{fig:3.7} are ideal,
find the values of the labeled voltages and currents. \\
\begin{figure}[H]
  \centering
  \begin{subfigure}[H]{0.45\textwidth}
    \centering
    \begin{circuitikz}
      \draw[color=black, thick] (0, 0)node[above]{$+\SI{5}{\siv}$} to [R, l=$\SI{5}{\siko}$, *-*] ++(0, -3) node[](n1){} to [Do, l_=$D_1$, i={\color{red} $I$}] ++(0, -3) node[ground] {}
      (n1) node[left]{\color{blue} $v_1$} to[short] ++(1.5, 0) to [Do, -*, l^=$D_2$] ++(0, -3) node[](n2){} to[R, l=$\SI{10}{\siko}$, -*] ++(0, -3) node[below]{$\SI{-5}{\siv}$} 
      (n2) to[short, -o] ++(1, 0) node[right] {\color{red} $V$};
    \end{circuitikz}
    \caption{}
  \end{subfigure}
  \begin{subfigure}[H]{0.45\textwidth}
    \centering
    \begin{circuitikz}
      \draw[color=black, thick] (0, 0)node[above]{$+\SI{5}{\siv}$} to [R, l=$\SI{10}{\siko}$, *-*] ++(0, -3) node[](n1){} to [Do, l_=$D_1$, i={\color{red} $I$}] ++(0, -3) node[ground] {}
      (n1)node[left]{\color{blue} $v_1$} to[short] ++(1.5, 0) to [Do, -*, l^=$D_2$] ++(0, -3) node[](n2){} to[R, l=$\SI{5}{\siko}$, -*] ++(0, -3) node[below]{$\SI{-5}{\siv}$} 
      (n2) to[short, -o] ++(1, 0) node[right] {\color{red} $V$};
    \end{circuitikz}
    \caption{}
  \end{subfigure}
  \caption{}
  \label{fig:3.7}
\end{figure}

\Ans \\
Notice that no matter how, the voltage of $v_1$ will never be less than $V$, so $D_2$ will be on and eventually $v_1 = V$. \\
In (a), if $D_1$ is off, then 
$$v_1 = \frac{10 \cdot 5 + 5 \cdot (-5)}{10 + 5} = \frac{5}{3} > 0$$
which leads an contradiction. Hence $D_1$ is on, and $V = v_1 = 0, I = \SI{5}{\siv} / \SI{5}{\siko} - \SI{5}{\siv} / \SI{10}{\siko} = \SI{0.5}{\sima}$.  \\
In (b), if $D_1$ is on, then $I = \SI{5}{\siv} / \SI{10}{\siko} - \SI{5}{\siv} / \SI{5}{\siko} = \SI{-0.5}{\sima}$, which violates the rule of ideal diode, so $D_1$ is off. 
$$ 
V = \frac{10 \cdot 5 + 5 \cdot (-5)}{10 + 5} = \frac{-5}{3} \si{\volt}, \quad I = 0
$$
% 18: 23
\section{3.18}
The circuit in Fig.~\ref{fig:3.18} utilizes three identical diodes having
$I_S = \SI{e-14}{\sia}$. Find the value of the current $I$ required
to obtain an output voltage $V_O = \SI{2}{\siv}$. If a current of
\SI{1}{\sima} is drawn away from the output terminal by a load,
what is the change in output voltage?

\begin{figure}[H]
  \centering
  \begin{circuitikz}
    \draw[color=black, thick] (0, 0)node[above]{$+\SI{5}{\siv}$} to [I, l=$I$, *-*] ++(0, -2.5) node[](n1){} to [Do] ++(0, -1.5) to [Do] ++(0, -1.5) to [Do] ++(0, -1.5) node[ground] {}
    (n1) to[short, -o] ++(1, 0) node[right] {\color{red} $V_O$};
  \end{circuitikz}
  \caption{}
  \label{fig:3.18}
\end{figure}

\Ans \\
If $V_O = \SI{2}\V$, the voltage across a single diode would be $v = \SI{2}\V / 3$, and that $I \approx I_s e^{v / V_T}$. Plug in $I_s = \SI{1e-14}\A, V_T = \SI{25}{\mV}$ \footnote{Since it's winter and thus colder now} and $v$ we get $I \approx \SI{3.8}\mV$. \\

If now $\SI{1}\mA$ is drawn away, then $I' = \SI{2.8}\mA$ and we have
\[
  \log \frac{I'}{I} = \frac{V' - V}{V_T} \; \Rightarrow \; \Delta V = V' - V = V_T \log \frac{I'}{I} 
\]
and plug in the value we get $ \Delta V = \SI{-7.6}\mV $.


\clearpage
% 21: 26
\section{3.21}
For the circuit shown in Fig.~\ref{fig:3.21}, both diodes are identical, conducting
\SI{10}{\sima} at \SI{0.7}{\siv} and \SI{100}{\sima}
at \SI{0.8}{\siv}. Find the value of $R$ for which
$V = \SI{80}{\simv}$.

\begin{figure}[H]
  \centering
  \begin{circuitikz}
    \draw[color=black, thick] (0, 0) to [I, l=10<\mA>, *-*] ++(0, -2.5) node[](n1){} to [Do, l_=$D_2$] ++(0, -1.5) node[ground] {}
    (n1) to[short, i={\color{blue} $i_1$}] ++(1.5, 0) to [Do, -*, l^=$D_1$] ++(0, -1.5) node[](n2){} to [R, l_=$R$] ++(0, -2.5) node[ground]{} 
    (n2) to [short, -o] ++(1, 0) to [open, v^={\color{red} $V$}] ++(0, -2.5)
    ;
  \end{circuitikz}
  \caption{}
  \label{fig:3.21}
\end{figure}

\Ans \\
Model the diode as$ i = I_s e^{V / V_T} $ and solve $I_S, V_T$ with $(i, V) = (\SI{10}\mA, \SI{0.7}{\V}), (\SI{100}\mA, \SI{0.8}{\V})$. We have
\[
\frac{i_2}{i_1} = 10 = e^{\frac{V_2 - V_1}{V_T}} \; \Rightarrow \; V_T = \frac{V_2 - V_1}{ \log 10 } = \SI{43.4}\mV \]
and thus $I_S = \SI{1e-6}{\mA}$. So
\begin{align*}
  & V_{D_2} - V_{D_1} = V \\
  \Leftrightarrow \; &V_T \log \frac{i_2}{i_1} = V \\
  \Leftrightarrow \; &V_T \log \frac{\SI{10}\mA - i_1}{i_1} = \SI{80}\mV \\
  \Leftrightarrow \; &i_1 \approx \SI{1.367}{\mA}
\end{align*}
Finally $R = V / i_1 \approx \SI{80}\mV / \SI{1.367}\mA = \SI{58.5}\ohm$

% 38: 46
\section{3.38}
In the circuit shown in Fig. , $I$ is a dc current and $v_s$ is a
sinusoidal signal. Capacitors $C_1$ and $C_2$ are very large; their
function is to couple the signal to and from the diode but block the
dc current from flowing into the signal source or the load (not shown).
Use the diode small-signal mdel to show that the signal component of the
output voltage is
\[ v_o = v_s \frac{V_T}{V_T + IR_S} \]
If $v_s = \SI{10}{\simv}$, find $v_o$ for
$I = \SI{1}{\sima}, \SI{0.1}{\sima}$, and
\SI{1}{\siua}. Let $R_s = \SI{1}{\siko}$. Assume $n = 2$. At
what value of $I$ doe $v_o$ become one-half of $v_s$?
Note that this circuit functions as a signal attenuator with the
attenuation factor controlled by the value of the dc current $I$.

% 40: 48
\section{3.40}
In the capacitor-coupled attenuator circuit shown in Fig. ,
$I$ is a dc current that varies from \SI{0}{\sima} to
\SI{1}{\sima}, and $C_1$ and $C_2$ are large coupling capacitors.
For very small input signals, so that the diodes can be represented by
their small-signal resistances $r_{d1}$ and $r_{d2}$, show that
$\displaystyle \frac{v_o}{v_i} = \frac{r_{d2}}{r_{d1}+r_{d2}}$ and hence
that $\displaystyle \frac{v_o}{v_i}  = I$, where $I$ is in
\si{\sima}. Find $v_o/v_i$ for $I = \SI{0}{\siua}, \SI{1}{\siua},
\SI{10}{\siua}, \SI{100}{\siua}, \SI{500}{\siua}, \SI{600}{\siua},
\SI{900}{\siua}, \SI{990}{\siua}, \SI{1}{\sima}$.

% 45: 53
\section{3.45}
Consider the voltage-regulator circuit shown in Fig. under the
condition that a load current $I_L$ is drawn from the output terminal.
\begin{enumerate}[(a)]
  \item If the value of $I_L$ is sufficiently small that the
    corresponding change in regulator output voltage $\Delta V_O$ is
    small enough to justify using the diode small-signal model, show
    that
    \[ \frac{\Delta V_O}{I_L} = -(r_d \parallel R) \]
    This quantity is known as the load regulation and is usually
    expressed in \si{\simv/\sima}.
  \item If the value of $R$ is selected such that at no load the voltage
    across the diode is \SI{0.7}{\siv} and the diode current is $I_D$,
    show that the expression derived in (a) becomes
    \[ \frac{\Delta V_O}{I_L} =
      -\frac{nV_T}{I_D}\frac{V^+ -0.7}{V^+ - 0.7 + nV_T} \]
    Select the lowest possible value for $I_D$ that results in a load
    regulation $\le \SI{5}{\simv/\sima}$. Assume $n = 2$.
    If $V^+$ is nominally \SI{10}{\siv}, what value of $R$ is required?
    Also, specify the diode required in terms of its $I_S$.
  \item Generalize the expression derived in (b) for the case of $m$
    diodes connected in series and $R$ adjusted to obtain $V_O =
    \SI{0.7}{\simv}$ at no load.
\end{enumerate}

% 48: 56
\section{3.48}
A particular design of a voltage regulator is shown in Fig. .
Diodes $D_1$ and $D_2$ are 10-\si{\sima} units; that is,
each has a voltage drop of \SI{0.7}{\siv} at a current of
\SI{10}{\sima}. Use the diode exponential model and iterative
analysis to answer the follwing questions:
\begin{enumerate}[(a)]
  \item What is the regulator output voltage $V_O$ with the
    150-\si{\ohm} load connected?
  \item Find $V_O$ with no load.
  \item With the load connected, to what value can the 5-\si{\siv}
    supply be lowered while maintaining the loaded output voltage
    within \SI{0.1}{\siv} of its nominal value?
  \item What does the loaded output voltage become when the 5-\si{\siv}
    supply is raised by the same amount as the drop found in (c)?
  \item For the range of changes explored in (c) and (d), by what
    percentage does the output voltage change for each percentage
    change of supply voltage in the worst case?
\end{enumerate}

% 52: 61
\section{3.52}
Design a 7.5-\si{\siv} zener regulator circuit using a 7.5-\si{\siv}
zener specified at \SI{12}{\sima}. The zener has an incremental
resistance $r_z = \SI{30}{\ohm}$ and a knee current of \SI{0.5}{\sima}.
The regulator operates from a 10-\si{\siv} supply and has a
1.2-\si{\siko} load. What is the value of $R$ you have chosen?
What is the regulator output voltage when the supply is \SI{10}{\percent}
high? Is \SI{10}{\percent} low? What is the output voltage when both the
supply is \SI{10}{\percent} hight and the load is removed? What is
the smallest possible load resistor that can be ued while the zener
operates at a current no lower than the knee current while the supply
is \SI{10}{\percent} low? What is the load voltage in this case?

% 53: 63
\section{3.53}
A zener shunt regulator employ a 9.1-\si{\siv} zener diode for which
$V_Z = \SI{9.1}{\siv}$ at $I_Z = \SI{9}{\sima}$, with $r_Z = \SI{30}{\ohm}$
and $I_{ZK}=\SI{0.3}{\sima}$. The available supply voltage of \SI{15}{\siv}
can vary as much as $\pm\SI{10}{\percent}$. For this diode, what is the
value of $V_{Z0}$? For a nominal load resistance $R_L$ of \SI{1}{\siko}
and a nominal zener current of \SI{10}{\sima}, what current must flow
in the supply resistor $R$? For the nominal value of supply voltage,
select a value for resistor $R$, specified to one significant digit,
to provide at least that current. What nominal output voltage results?
For a $\pm\SI{10}{\percent}$ change in the supply voltage, what variation
in output voltage results? If the load current is reduced by
\SI{50}{\percent}, what increase in $V_O$ results? What is the smallest
value of load resistance that can be tolerated while maintaining
regulation when the supply voltage is low? What is the lowest possible
output voltage that results? Calculate values for the line regulation
and for the load regulation for this circuit using the numerical results
obtained in this problem.

% 60: 74
\section{3.60}
The circuit in Fig. implements a complementary-output rectifier.
Sketch and clearly label the waveforms of $v_O^+$ and $v_O^-$.
Assume a 0.7-\si{\siv} drop across each conducting diode. If the
magnitude of the average of each output is to be \SI{15}{\siv}, find
the required amplitude of the sine wave across the entire secondary
winding. What is the PIV of each diode?

%%%%%%%%%%%%%%%%%%%%%%%%%%%%%%%%%%%%%%%%%%%%%%%%%%%%%%%%%%%%%
% \bibliographystyle{plain}
% \begin{thebibliography}{99}
% \bibitem[1]{ex}\url{http://www.example.com/}
% \end{thebibliography}
\end{document}

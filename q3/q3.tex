\documentclass[12pt, a4paper]{article}
%%%%%%%%%%%%%%%紙張大小設定%%%%%%%%%%%%%%%
% \paperwidth=65cm
% \paperheight=160cm

%%%%%%%%%%%%%%%引入Package%%%%%%%%%%%%%%%
\usepackage[margin=3cm]{geometry} % 上下左右距離邊緣2cm
\usepackage{mathtools,amsthm,amssymb} % 引入 AMS 數學環境
\usepackage{yhmath}      % math symbol
\usepackage{graphicx}    % 圖形插入用
\usepackage{fontspec}    % 加這個就可以設定字體
\usepackage{type1cm}	 % 設定fontsize用
\usepackage{titlesec}   % 設定section等的字體
\usepackage{titling}    % 加強 title 功能
\usepackage{fancyhdr}   % 頁首頁尾
\usepackage{tabularx}   % 加強版 table
\usepackage{multirow}   % colspan
\usepackage[square, comma, numbers, super, sort&compress]{natbib}
% cite加強版
\usepackage[unicode=true, pdfborder={0 0 0}, bookmarksdepth=-1]{hyperref}
% ref加強版
\usepackage[usenames, dvipsnames]{color}  % 可以使用顏色
\usepackage[shortlabels, inline]{enumitem}  % 加強版enumerate
\usepackage{xpatch}

% \usepackage{tabto}      % tab
% \usepackage{soul}       % highlight
% \usepackage{ulem}       % 字加裝飾
\usepackage{wrapfig}     % 文繞圖
%\usepackage{floatflt}    % 浮動 figure
\usepackage{float}       % 浮動環境
\usepackage{caption}    % caption 增強
\usepackage{subcaption}    % subfigures
% \usepackage{setspace}    % 控制空行
% \usepackage{mdframed}   % 可以加文字方框
% \usepackage{multicol}   % 多欄
\usepackage[abbreviations, per-mode=symbol]{siunitx}      % SI unit
% \usepackage{dsfont}     % more mathbb

%%%%%%%%%%%%%%%%%%%TikZ%%%%%%%%%%%%%%%%%%%%%%
% \usepackage{tikz}
\usepackage[siunitx, americanvoltages, americancurrents, arrowmos]{circuitikz}
\usetikzlibrary{calc}

%%%%%%%%%%%%%%中文 Environment%%%%%%%%%%%%%%%
\usepackage[CheckSingle, CJKmath]{xeCJK}  % xelatex 中文
\usepackage{CJKulem}	% 中文字裝飾
\setCJKmainfont[BoldFont=cwTeX Q Hei]{cwTeX Q Ming}
\setCJKsansfont[BoldFont=cwTeX Q Hei]{cwTeX Q Ming}
\setCJKmonofont[BoldFont=cwTeX Q Hei]{cwTeX Q Ming}
% 設定中文為系統上的字型,而英文不去更動,使用原TeX字型

% \XeTeXlinebreaklocale "zh"             %這兩行一定要加,中文才能自動換行
% \XeTeXlinebreakskip = 0pt plus 1pt     %這兩行一定要加,中文才能自動換行

%%%%%%%%%%%%%%%字體大小設定%%%%%%%%%%%%%%%
% \def\normalsize{\fontsize{10}{15}\selectfont}
% \def\large{\fontsize{40}{60}\selectfont}
% \def\Large{\fontsize{50}{75}\selectfont}
% \def\LARGE{\fontsize{90}{20}\selectfont}
% \def\huge{\fontsize{34}{51}\selectfont}
% \def\Huge{\fontsize{38}{57}\selectfont}

%%%%%%%%%%%%%%%Theme Input%%%%%%%%%%%%%%%%
% \input{themes/chapter/neat}
% \input{themes/env/problist}

%%%%%%%%%%%titlesec settings%%%%%%%%%%%%%%
% \titleformat{\chapter}{\bf\Huge}
            % {\arabic{section}}{0em}{}
% \titleformat{\section}{\centering\Large}
            % {\arabic{section}}{0em}{}
% \titleformat{\subsection}{\large}
            % {\arabic{subsection}}{0em}{}
% \titleformat{\subsubsection}{\bf\normalsize}
            % {\arabic{subsubsection}}{0em}{}
% \titleformat{command}[shape]{format}{label}
            % {編號與標題距離}{before}[after]

%%%%%%%%%%%%variable settings%%%%%%%%%%%%%%
% \numberwithin{equation}{section}
% \setcounter{secnumdepth}{4}  %章節標號深度
% \setcounter{tocdepth}{1}  %目錄深度
% \graphicspath{{images/}}  % 搜尋圖片目錄

%%%%%%%%%%%%%%%頁面設定%%%%%%%%%%%%%%%
\newcolumntype{C}[1]{>{\centering\arraybackslash}p{#1}}
\setlength{\headheight}{15pt}  %with titling
\setlength{\droptitle}{-1.5cm} %title 與上緣的間距
% \posttitle{\par\end{center}} % title 與內文的間距
\parindent=24pt %設定縮排的距離
% \parskip=1ex  %設定行距
% \pagestyle{empty}  % empty: 無頁碼
% \pagestyle{fancy}  % fancy: fancyhdr

% use with fancygdr
% \lhead{\leftmark}
% \chead{}
% \rhead{}
% \lfoot{}
% \cfoot{}
% \rfoot{\thepage}
% \renewcommand{\headrulewidth}{0.4pt}
% \renewcommand{\footrulewidth}{0.4pt}

% \fancypagestyle{firststyle}
% {
  % \fancyhf{}
  % \fancyfoot[C]{\footnotesize Page \thepage\ of \pageref{LastPage}}
  % \renewcommand{\headrule}{\rule{\textwidth}{\headrulewidth}}
% }

%%%%%%%%%%%%%%%重定義一些command%%%%%%%%%%%%%%%
\renewcommand{\contentsname}{目錄}  %設定目錄的標題名稱
\renewcommand{\refname}{參考資料}  %設定參考資料的標題名稱
\renewcommand{\abstractname}{\LARGE Abstract} %設定摘要的標題名稱

%%%%%%%%%%%%%%%特殊功能函數符號設定%%%%%%%%%%%%%%%
% \newcommand{\citet}[1]{\textsuperscript{\cite{#1}}}
\DeclarePairedDelimiter{\abs}{\lvert}{\rvert}
\DeclarePairedDelimiter{\norm}{\lVert}{\rVert}
\DeclarePairedDelimiter{\inpd}{\langle}{\rangle} % inner product
\DeclarePairedDelimiter{\ceil}{\lceil}{\rceil}
\DeclarePairedDelimiter{\floor}{\lfloor}{\rfloor}
\DeclareMathOperator{\adj}{adj}
\DeclareMathOperator{\sech}{sech}
\DeclareMathOperator{\csch}{csch}
\DeclareMathOperator{\arcsec}{arcsec}
\DeclareMathOperator{\arccot}{arccot}
\DeclareMathOperator{\arccsc}{arccsc}
\DeclareMathOperator{\arccosh}{arccosh}
\DeclareMathOperator{\arcsinh}{arcsinh}
\DeclareMathOperator{\arctanh}{arctanh}
\DeclareMathOperator{\arcsech}{arcsech}
\DeclareMathOperator{\arccsch}{arccsch}
\DeclareMathOperator{\arccoth}{arccoth}
\newcommand{\np}[1]{\\[{#1}] \indent}
\newcommand{\transpose}[1]{{#1}^\mathrm{T}}
%%%% Geometry Symbol %%%%
\newcommand{\degree}{^\circ}
\newcommand{\Arc}[1]{\wideparen{{#1}}}
\newcommand{\Line}[1]{\overleftrightarrow{{#1}}}
\newcommand{\Ray}[1]{\overrightarrow{{#1}}}
\newcommand{\Segment}[1]{\overline{{#1}}}

%%%% SI unit short cut %%%%
\newcommand{\siua}{\micro\ampere}
\newcommand{\sima}{\milli\ampere}
\newcommand{\simv}{\milli\volt}
\newcommand{\siko}{\kilo\ohm}
\newcommand{\sio}{\ohm}
\newcommand{\sia}{\ampere}
\newcommand{\siv}{\volt}

\newcommand{\img}{\mathrm{i}}
\newcommand{\ex}{\mathsf{e}}
\newcommand{\dD}{\mathrm{d}}
\newcommand{\iD}{\:\mathrm{d}}

%%%%%%%%%%%%%%%證明、結論、定義等等的環境%%%%%%%%%%%%%%%
\renewcommand{\proofname}{\bf 證明:} %修改Proof 標頭
\newtheoremstyle{mystyle}% 自定義Style
  {6pt}{15pt}%       上下間距
  {}%               內文字體
  {}%               縮排
  {\bf}%            標頭字體
  {.}%              標頭後標點
  {1em}%            內文與標頭距離
  {}%               Theorem head spec (can be left empty, meaning 'normal')

% 改用粗體,預設 remark style 是斜體
\theoremstyle{mystyle}	% 定理環境Style
\newtheorem{theorem}{定理}
\newtheorem{definition}{定義}
\newtheorem{formula}{公式}
\newtheorem{condition}{條件}
\newtheorem{supposition}{假設}
\newtheorem{conclusion}{結論}
\newtheorem{lemma}{引理}
\newtheorem{property}{性質}

%% Label set %%
\captionsetup[figure]{labelsep=period}

%% Ans %%
\newcommand{\Ans}{\noindent{\bf Ans:}}

%% No Indent %%
\setlength\parindent{0pt}

\newcommand{\red}{\color{red}}
\newcommand{\blue}{\color{blue}}

\newcommand{\paral}{\mathbin{\|}}
\newcommand\numeq{\addtocounter{equation}{1}\tag{\theequation}}


%% Note that for label to be correct, compile more than 2 times is needed ... %%


%%%%%%%%%%%%%%%Title的資訊%%%%%%%%%%%%%%%
\title{} %標題
\author{} %作者
\date{} %日期


\begin{document}
\tikzstyle{default}=[thick, color=black]
% \maketitle %製作tilte page
% \thispagestyle{empty}  %去除頁碼
% \thispagestyle{fancy}  %使用fancyhdr
% \tableofcontents %目錄
%%%%%%%%%%%%%%%%%%%include file here%%%%%%%%%%%%%%%%%%%%%%%%%
% 7: 9
\section{5.10}
Consider an n-channel MOSFET with $t_{ox} = \SI{20}\nm$, $\mu_{n} = \SI{650}{\cm\squared\per\V\per\s}$, $V_t = \SI{0.8}\V$ and $W/L = 10$. Find the drain current in the following cases:

\begin{enumerate}[label=(\alph*)]
  \item $v_{GS} = \SI{5}{\V}$ and $v_{DS} = \SI{1}{\V} $\\[5pt]
    \Ans \\
    First we calculate 
    \[ k_n = \mu_n C_{ox} (W / L) = \mu_n \varepsilon_{ox} (W / L) /  t_{ox} \approx \SI{1.12}{\mA\per\V\squared} \]
    Since $v_{GS} > V_t$ and $v_{GS} - v_{DS} = \SI{4}{\V} > V_t$, the MOS is working on the triode region, so 
    \[
      i_D =  k_n \left(v_{GS} - V_t - \frac{1}{2}v_{DS} \right)v_{DS} \approx \SI{4.14}{\mA} 
    \]
  \item $v_{GS} = \SI{2}{\V}$ and $v_{DS} = \SI{1.2}{\V} $\\[5pt]
    \Ans \\
    Now $V_{GS} > V_t$ and $V_{GS} - V_{DS} = \SI{0.8}{\V} = V_t$, the MOS is working on the boundary of the saturation and the triode region, so 
    \[
      i_D = \frac{1}{2} k_n (V_{GS} - V_t)^2 \approx \SI{1.61}{\mA} 
    \]
  \item $v_{GS} = \SI{5}{\V}$ and $v_{DS} = \SI{0.2}{\V} $\\[5pt]
    \Ans \\
    Again $V_{GS} > V_t$ and $V_{GS} - V_{DS} = \SI{4.8}{\V} > V_t$, the MOS is working on the triode region, so 
    \[
      i_D =  k_n \left(v_{GS} - V_t - \frac{1}{2}v_{DS} \right)v_{DS} \approx \SI{0.92}{\mA} 
    \]
  \item $v_{GS} = \SI{5}{\V}$ and $v_{DS} = \SI{5}{\V} $\\[5pt]
    \Ans \\
    Finally since $V_{GS} > V_t$ and $V_{GS} - V_{DS} = \SI{0}{\V} < V_t$, the MOS is working on the saturation region, so 
    \[
      i_D =  \frac{1}{2} k_n (v_{GS} - V_t)^2 \approx \SI{9.88}{\mA} 
    \]
\end{enumerate}
% 18: 23
\section{5.20}
The table below lists 10 different cases labeled (a) to (j) for operating an NMOS transistor with $V_t = \SI{1}{\V} $. In each case the voltages at the source, gate, and drain (relative to the circuit ground) are specified. You are required to complete the tabel entries. Note that if you encounter a case for which $v_{DS}$ is negative, you should exchange the drain and source before solving the problem. You can do this because the MOSFET is a symmetric device.


\Ans \\
\begin{center}
  \begin{tabular}{|l|c|c|c|c|c|c|c|}
    \hline
    & \multicolumn{6}{c|}{Voltage \si\V} & \\ \cline{2-7}
    Case & $V_S$ & $V_G$ & $V_D$ & $V_{GS}$ & $V_{OV}$ & $V_{DS}$ & Region of operation \\
    \hline
    a & $+1.0$ & $+1.0$ & $+2.0$ & $0$ & $0$ & $1.0$ & Cut-off \\
    b & $+1.0$ & $+2.5$ & $+2.0$ & $1.5$ & $0.5$ & $1.0$ & Saturation \\
    c & $+1.0$ & $+2.5$ & $+1.5$ & $1.5$ & $0.5$ & $0.5$ & Boundary of Sat./Tri. \\
    d & $+1.0$ & $+1.5$ & $+2.0$ & $0.5$ & $0$ & $1.0$ & Cut-off \\
    e & $0$ & $+2.5$ & $+1.0$ & $2.5$ & $1.5$ & $1$ & Triode \\
    f & $+1.0$ & $+1.0$ & $+1.0$ & $0$ & $0$ & $0$ & Cut-off \\
    g & $-1.0$ & $0$ & $0$ & $1.0$ & $0$ & $1.0$ & Boundary of Cut./Sat. \\
    h & $-1.5$ & $0$ & $0$ & $1.5$ & $0.5$ & $1.5$ & Saturation \\
    i & $-1.0$ & $0$ & $+1.0$ & $1.0$ & $0$ & $2.0$ & Saturation \\
    j & $+0.5$ & $+2.0$ & $+0.5$ & $1.5$ & $0.5$ & $0$ & Triode \\
    \hline
  \end{tabular}
\end{center}

\clearpage
% 21: 26
\section{5.29}

% 38: 46
\section{5.34}
The NMOS transistors in the circuit of  Fig.~\ref{fig:3.38} have $V_t = \SI{1}{\V} , \mu_n C_{ox} = \SI{120}{\uA\per\V\squared}, \lambda = 0$ and $L_1 = L_2 = L_3 = \SI{1}{\um} $. Find the required values of gate width for each of $Q_1$, $Q_2$, and $Q_3$ to obtain the voltage and current values indicated.

\begin{figure}[H]
  \centering
  \begin{circuitikz}[>=triangle 45, scale=1, transform shape]
    \draw[default] (0, 1) node[nmos, label={[shift={(0.4, -0.3)}]$Q_1$}](nm1){};
    \draw[default] (0, 2.5) node[nmos, label={[shift={(0.4, -0.3)}]$Q_2$}](nm2){};
    \draw[default] (0, 4) node[nmos, label={[shift={(0.4, -0.3)}]$Q_3$}](nm3){};
    \draw[color=black, thick] let \p1=(nm1) in
    (0, \x1) node[ground]{} to [short] (nm1.source);
    \draw[default] (nm1.gate) |- (nm1.drain) to[short, *-o] ++(0.5, 0) node[right]{$+\SI{1.5}{\V}$};
    \draw[default] (nm2.gate) |- (nm2.drain) to[short, *-o] ++(0.5, 0) node[right]{$+\SI{3.5}{\V}$};
    \draw[default] (nm3.gate) |- (nm3.drain) to[short, *-, i<_=120<\uA>] ++(0, 0.5);
    \draw[color=black, thick, ->]
      ($(nm3.drain) + (0, 0.3)$) to ++(0, 0.5) node[above]{$+\SI{5}{\V}$}
      ;
  \end{circuitikz}
  \caption{}
  \label{fig:3.38}
\end{figure}

\Ans \\
Notice that every MOS is working on saturation region, since $v_D = v_G$,  hence $v_{DS} > v_{GS} - V_t$. Now for $Q_1, Q_3$, $V_{OV} = v_{GS} - V_t = \SI{0.5}{\V} $ and for $Q_2$, $V_{OV} = \SI{1}{\V} $.
Hence by solving 
\[
  \frac{1}{2} \mu_n C_{ox} \frac{W}{L} V_{OV}^2 = I_D = \SI{120}{\uA} 
\]
We get $W_1 = W_3 = \SI{4}{\um} , W_2 = \SI{2}{\um} $

% 40: 48
\section{5.41}
For the devices in the circuits of Figure~\ref{fig:5.41}, $ \abs{V_t} = \SI{1}{V} , \lambda = 0, \mu_n C_{ox} = \SI{50}{\uA\per\V\squared}, L = \SI{1}{\um}$, and $W = \SI{10}{\um}$. Find $V_2$ and $I_2$. How do these values change if $Q_3$ and $Q_4$ are made to have $W = \SI{100}{\um}$?
\begin{figure}[H]
  \centering
  \begin{circuitikz}[>=triangle 45]
    \draw[default] (0, 0) node[nmos, xscale=-1] (Q3) {}
    (Q3.S) node[ground] {}
    (Q3.D) to[short, i<_=$I_2$] ++(0, 0.5) node[](p1){} to [short, *-o] ++(-0.5, 0) node[left]{$V_2$}
    (p1) to[short] ++(0, 0.5) node[nmos, anchor=S](Q4){}
    (Q3.G) to[short] ++(0.5, 0) node[circ](p2){} to[short] ++(0.5, 0) node[nmos, anchor=G](Q1) {}
    (Q1.S) node[ground]{}
    (p2) |- ($(Q1.D) + (0, 0.5)$) node[circ](p3){} -- (Q1.D)
    (p3) to[short, i<^={\color{blue}$I_1$}] ++(0, 0.5) node[nmos, xscale=-1.0, anchor=S](Q2){}
    (Q1) node[right]{$Q_1$}
    (Q2) node[left]{$Q_2$}
    (Q3) node[left]{$Q_3$}
    (Q4) node[right]{$Q_4$}
    ($(p2) + (0, 4.5)$) node[]{$+\SI{5}{\V}$}
    (p3) node[right]{\color{blue}$V_1$}
    ; 
    \draw[default, ->] (Q4.D) -- ++(0, 1);
    \draw[default, ->] (Q4.G) -| ++(-0.2, 1.75);
    \draw[default, ->] (Q2.D) -- ++(0, 1);
    \draw[default, ->] (Q2.G) -| ++(0.2, 1.75);
    \draw[dashed, color=blue] (p1) -- (p3);
  \end{circuitikz}
  \caption{}
  \label{fig:5.41}
\end{figure}


\Ans \\
Notice that if we connect the blue dashed line in the figure, the circuit is now symmetric. Moreover, all the MOS are working in saturation region.
Now $Q_1, Q_2, Q_3, Q_4$ has the same $k_n$, and the current go through $Q_4, Q_2$ is equal to the current that go through $Q_3, Q_1$. So by symmetry, $V_2 = V_1 =  \SI{5}{V} / 2 = \SI{2.5}{\V}$. Notice that there is no currenct go through the blue dashed line, so if we remove the line, the voltage remains unchanged. Hence we know that in the origin circuit, $V_2 = \SI{2.5}{\V}$.
\begin{align*}
  k_n &= \mu_n C_{ox} (W / L) = \SI{0.5}{\mA} \\
  V_2 &= v_{GS} = \SI{2.5}{\V} \\
  I_2 &= \frac{1}{2} k_n (v_{GS} - V_t) ^2 = \SI{562.5}{\uA}
\end{align*}
Simmilarly, if now $W_{3,4} = \SI{100}{\um}$, $k_n$ is $10$ times greater than before, so $I_2 = \SI{5.625}{\mA}$.

% 45: 53
\section{5.43}

\section{5.46}
Figure~\ref{fig:5.46} shows an amplifier in which the load resistor $R_D$ has been replaced with another NMOS transistor $Q_1$ connected as a two-terminal device. Note that because $v_{DG}$ of $Q_2$ is zero, it will be operating in saturation at all times, even when $v_I = 0$ and $i_{D2} = i_{D1} = 0$. Note also that the two transistors conduct equal drain currents. Using $i_{D1} = i_{D2}$, show that for the range of $v_I$ over which $Q_1$ is operating in saturation, that is , for
\[ V_{t1} \leq v_I \leq v_O +V_{t1} \]
the output voltage will be given by
\[ v_O = V_{DD} - V_t + \sqrt{ \frac{W_1/L_1}{W_2/L_2} }V_t - \sqrt{ \frac{W_1/L_1}{W_2/L_2} }v_I \]

\begin{figure}[H]
  \centering
  \begin{circuitikz}[>=triangle 45]
    \draw[default] 
    (0, 0) node[ground]{} node[nmos, anchor=S](Q1){}
    (Q1.G) to[short, -o] ++(-0.5, 0) node[left]{$v_I$}
    (Q1.D) to[short, i^<={\color{red}$i_{D1}$}] ++(0, 0.5) node[](p1){} to[short, *-o] ++(0.5, 0) node[right]{$v_O$}
    (p1) to[short] ++(0, 0.5) node[nmos, anchor=S](Q2){}
    ($(Q2.D) + (0, 0.5)$) node[circ](p2){} -| (Q2.G)
    (p2) to[short, i^>={\color{red}$i_{D2}$}] (Q2.D)
    (Q1) node[right] {$Q_1$}
    (Q2) node[right] {$Q_2$}
      ;
      
    \draw[default, ->] (p2) -- ++(0, 0.5) node[above]{$V_{DD}$};
  \end{circuitikz}
  \caption{}
  \label{fig:3.45}
\end{figure}

\Ans
Since both the MOS is operating in saturation, we have
\begin{align*}
  i_{D1} &= \frac{1}{2} k'_n \frac{W_1}{L_1} (v_I - V_t)^2 \\
  i_{D2} &= \frac{1}{2} k'_n \frac{W_2}{L_2} (V_{DD} - v_O - V_t)^2 
\end{align*}

By $i_{D1} = i_{D2}$ we obtain
\begin{gather*}
  \frac{W_1}{L_1} (v_I - V_t)^2 = \frac{W_2}{L_2} (V_{DD} - v_O - V_t)^2 \\
  \Rightarrow v_O = V_{DD} - V_t + \sqrt{ \frac{W_1/L_1}{W_2/L_2} } (V_t - v_I) 
\end{gather*}
Which is the desired result.

% 48: 56
\section{5.50}
In this problem we investigate an optimum design of the CS amplifier circuit of Figure~\ref{fig:}. First. use the voltage gain expression $A_v = -g_m R_D$ together with Eq(5.57) for $g_m$ to show that
\[ A_v = -\frac{2I_D R_D}{V_{OV}} = - \frac{2(V_{DD} - V_D)}{V_{OV}} \]
Next, let the maximum positive input signal be $\hat{v_i}$. To keep the second-harmonic distortion to an acceptable level, we bias the MOSFET to operate at an overdrive voltage $V_{OV} \gg \hat{v_i}$. Let $V_{OV} = m\hat{v_i}$. Now, to maximize the voltage gain $\abs{A_v}$, we design for the lowest possible $V_D$. Show that the minimum $V_D$ that is consistent with allowing a negative signal voltage swing at the drain of $\abs{A_v}\hat{v_i}$ while maintaining saturation-mode operation is given by
\[
  V_D = \frac{V_{OV} + \hat{v_i} + 2 V_{DD} (\hat{v_i}/V_{OV})}{1 + 2(\hat{v_i}/V_{OV})} 
\]
Now, find $V_{OV}, V_D, A_v$, and $\hat{v_o}$ for the case $V_{DD} = \SI{3}{\V}, \hat{v_i} = \SI{20}{\mV}, \text{and } m = 10$. If it is desired to operate this transistor at $I_D = \SI{100}{\uA}$, find the values of $R_D$ and $W/L$, assuming that for this process technology $k'_n = \SI{100}{\uA\per\V\squared}$.

\Ans\\
\begin{align*}
  g_m &= \frac{2I_D}{V_{OV}}\\
  A_v &= -g_m R_D = -\frac{2 I_D R_D}{V_{OV}}  = -\frac{2 (V_{DD} - V_D)}{V_{OV}} 
\end{align*}
Let $v_g = V_G + v_i$ be the total gate voltage, $v_d = V_D + A_v v_i$ be the total drain voltage, the saturation condition required that $v_d \geq v_g - V_t$. so
\begin{align*}
  & v_d \geq v_g - V_t \\
  \Rightarrow & V_D + A_v v_i \geq V_G + v_i - V_t  \\
  \Rightarrow & V_D - \frac{2 (V_{DD} - V_D)}{V_{OV}} v_i \geq V_G + v_i - V_t \\
  \Rightarrow  & V_D - \frac{2 (V_{DD} - V_D)}{V_{OV}} \hat{v_i} \geq V_G + \hat{v_i} - V_t \quad \footnotemark \\
  \Rightarrow & \left(1 + \frac{2\hat{v_i}}{V_{OV}}\right) V_D \geq V_G + \hat{v_i} + \frac{2V_{DD}}{V_{OV}} \hat{v_i}  - V_t  \\
  \Rightarrow & V_D \geq \frac{V_{OV} + \hat{v_i} + 2 V_{DD} (\hat{v_i} / V_{OV}) }{1 + 2(\hat{v_i}/V_{OV})} 
\end{align*}
\footnotetext{Since $A_v, v_i$ and $v_i$ are in opposite sign, so in the worse case, we shall consider $-A_v \hat{v_i}$ and $\hat{v_i}$}
Which is the desired result.
Plug in the value and let $V_D$ be the maximum possible value we found out that
\begin{alignat*}{3}
  & V_{OV} &&= m \hat{v_i} && = \SI{200}{\mV} \\
  & V_{D} &&= \frac{V_{OV} + \hat{v_i} + 2 V_{DD} (\hat{v_i} / V_{OV}) }{1 + 2(\hat{v_i}/V_{OV})} && \approx \SI{683}{\mV} \\
  & A_v &&= -\frac{2(V_{DD} - V_D)}{V_{OV}}  && \approx -23.2 \\
  & \hat{v_o} &&= \abs{A_v} \hat{v_i} && \approx \SI{464}\mV \\
\end{alignat*}
If moreover, $I_D = \SI{100}{\uA}$,
\begin{alignat*}{3}
  & R_D &&= \frac{V_{DD} - V_D}{I_D}  && \approx \SI{23.17}{\kohm} \\
  & \frac{W}{L} &&= I_D / \left(\frac{1}{2} k'_n V_{OV}^2 \right) && = 50
\end{alignat*}


% 52: 61
\section{5.80}
The NMOS transistor in the CS amplifier shown in Figure~\ref{fig:} has $V_t = \SI{0.7}{\V}$ and $V_A = \SI{50}{\V}$.

\begin{enumerate}
  \item Neglecting the Early effect, verify that the MOSFET is operating in saturation with $I_D = \SI{0.5}{\mA}$  and $V_{OV} = \SI{0.3}{\V}$. What must the MOSFET's $k_n$ be? What is the dc voltage at the drain?
  \item Find $R_{in}$ and $G_v$.
  \item If $v_{sig}$ is a sinusoid with a peak amplitude $\hat{v}_{sig}$, find the maximum allowable value of $\hat{v}_{sig}$ for which the transistor remains in saturation. What is the corresponding amplitude of the output voltage?
  \item What is the value of resistance $R_s$ that needs to be inserted in series with capacitor $C_S$ in order to allow us to double the input signal $\hat{v}_{sig}$? What output voltage now results?
\end{enumerate}

\begin{figure}[H]
\begin{center}
  \begin{circuitikz}[>=triangle 45, scale=1, transform shape]
    \draw[default]
    (0, 0) node[ground]{} to[V, l=$v_{sig}$] ++(0, 3) to[R, l=120<\kohm>, -o] ++(2, 0) to[C, l=$C_{C1}$, -*] ++(3, 0) 
    coordinate(v1) to[R, l=200<\kohm>] ++(0, -3) node[ground]{}
    (v1) to[R, l=300<\kohm>] ++(0, 3) coordinate(v5)
    (v1) to[short] ++(1, 0) node[nmos, anchor=G](q1){}
    (q1.S) to[short] ++(0, -.5) coordinate(v2) to[R, l=2<\kohm>] ++(0, -1.75) node[ground]{}
    (v2) to[C, l=$C_S$] ++(2, 0) node[ground]{}
    (q1.D) to[short, -*] ++(.5, 0) coordinate(v3) to[R, l=5<\kohm>] ++(0, 2) coordinate(v4)
    (v3) to[C, l=$C_{C2}$, -*] ++(3, 0) coordinate(v6) to[R, l=5<\kohm>] ++(0, -2) node[ground]{} 
    (v6) to[short, -o] ++(.5, 0) node[right]{$v_o$}
    ;
    \draw[default, ->] (v5) -- ++(0, .5) node[right, above]{$+\SI{5}{\V}$};
    \draw[default, ->] (v4) -- ++(0, .75) ;
  \end{circuitikz}
\end{center}
\caption{}
\label{fig:}
\end{figure}


% 53: 63
\section{5.81}
The PMOS transistor in the CS amplifier of Figure~\ref{fig:} has $V_{tp} = \SI{-0.7}{\V}$ and a very large $\abs{V_A}$.
\begin{enumerate}[(a)]
  \item Select a value for $R_S$ to bias the transistor at $I_D = \SI{0.3}{\mA}$ and $\abs{V_{OV}} = \SI{0.3}{V}$. Assume $v_{sig}$ to have a zero dc component.
  \item Select a value for $R_D$ that results in $G_v = \SI{10}{\V\per\V}$.
  \item Find the largest sinusiod $\hat{v}_{sig}$ that the amplifier can handle while remaining in the saturation region. What is the corresponding signal at the output?
  \item If to obtain reasonably linear operation $\hat{v}_{sig}$ is limited to $\SI{50}{\mV}$, what value can $R_D$ be increased to while maintaining saturation-region operation? What is the new value of $G_v$.
\end{enumerate}

\begin{figure}[H]
  \centering
  \begin{circuitikz}[>=triangle 45, scale=1, transform shape]
    \draw[default]
    (0, 0) node[ground]{} to[V, l=$v_{sig}$] ++(0, 2) to[R, l=$R_{sig}$, -o] ++(3, 0) to[short] ++(0, 1) node[pmos, anchor=G](q1){}
      (q1.S) to[short, -*] ++(0, 0.5) coordinate(v1) to[R, l=$R_S$] ++(0, 2) coordinate(v2)
      (q1.D) to[short, -*] ++(0, -0.5) coordinate(v3) to[R, l=$R_D$] ++(0, -2) coordinate(v4)
      (v1) to[C, l=$C_S$] ++(2, 0) node[ground]{}
      (v3) to[C, l=$C_C$, -o] ++(2, 0) node[right]{$v_o$}
    ;
    \draw[default, ->] (v2) -- ++(0, 0.5) node[above]{$+\SI{2.5}{\V}$};
    \draw[default, ->] (v4) -- ++(0, -0.5) node[below]{$\SI{-2.5}{\V}$};
  \end{circuitikz}
\end{figure}

% 60: 74
\section{5.83}
\begin{enumerate}[(a)]
  \item The NMOS transistor in the source-follower circuit of Figure~\ref{fig:5.83a} has $g_m = \SI{5}{\mA\per\V}$
    and a large $r_o$. Find the open-circuit voltage gain and the output resistance.
  \item The NMOS transistor in the common-gate amplifier of Figure~\ref{fig:??} has $g_m = \SI{5}{\mA\per\V}$ and
    a large $r_o$. Find the input resistance and the voltage gain.
  \item If the output of the source follower in (a) is connected to the input of the common-gate amplifier in (b), use the results of (a) and (b) to obtain the overall voltage gain $v_o/v_i$.
\end{enumerate}

\begin{figure}[H]
  \centering
  \begin{subfigure}[b]{0.5\textwidth}
    \begin{circuitikz}[>=triangle 45, scale=1, transform shape]
      \draw[default]
      (0, 0) to[R, l={$\SI{10}\kohm$}] ++(0, 2) node[](v1){} to[C, l=$\infty$, *-o] ++(2, 0) node[right]{\color{red}$v_{o1}$}
      (v1) to[short] ++(0, 1) node[nmos, anchor=S](n1){}
      (n1.G) to[short, -o] ++(-1, 0) node[left]{\color{red}$v_i$}
      ;
      \draw[default, ->] (n1.D) -- ++(0, 1) ;
      \draw[default, ->] (0, 0) -- ++(0, -1) ;
        ;
    \end{circuitikz}
    \caption{}
    \label{fig:5.83a}
  \end{subfigure}%
  \begin{subfigure}[b]{0.5\textwidth}
    \begin{circuitikz}[>=triangle 45, scale=1, transform shape]
      \draw[default]
      (0, 0) to[R, l=$\SI{10}\kohm$] ++(0, 2) node[](v1){} to[short, -o]  ++(-0.5, 0) node[left]{\color{red}$v_{i2}$}
      (v1) to[short] ++(0, 0.5) node[nmos, anchor=S, xscale=-1.0](q1){}
      (q1.G) -| ++(0.3, -0.3) node[ground]{}
      (q1.D) to[short, -*] ++(0, 0.5) coordinate(v2) to[R, l=5<\kohm>] ++(0, 2)  coordinate(v3)
      (v2) to[C, l=$\infty$] ++(2.5, 0) coordinate(v4) to[R, l=2<\kohm>] ++(0, -2) node[ground]{}
      (v4) to[short, *-o] ++(0.5, 0) node[right]{$v_o$}
      ;
      \draw[default, ->] (v3) --  ++(0, 0.5);
      \draw[default, ->] (0, 0) --  ++(0, -0.5);
    \end{circuitikz}
    \caption{}
  \end{subfigure}
\end{figure}


%%%%%%%%%%%%%%%%%%%%%%%%%%%%%%%%%%%%%%%%%%%%%%%%%%%%%%%%%%%%%
% \bibliographystyle{plain}
% \begin{thebibliography}{99}
% \bibitem[1]{ex}\url{http://www.example.com/}
% \end{thebibliography}
\end{document}

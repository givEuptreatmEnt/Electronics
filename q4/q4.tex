\documentclass[12pt, a4paper]{article}
%%%%%%%%%%%%%%%紙張大小設定%%%%%%%%%%%%%%%
% \paperwidth=65cm
% \paperheight=160cm

%%%%%%%%%%%%%%%引入Package%%%%%%%%%%%%%%%
\usepackage[margin=3cm]{geometry} % 上下左右距離邊緣2cm
\usepackage{mathtools,amsthm,amssymb} % 引入 AMS 數學環境
\usepackage{yhmath}      % math symbol
\usepackage{graphicx}    % 圖形插入用
\usepackage{fontspec}    % 加這個就可以設定字體
\usepackage{type1cm}	 % 設定fontsize用
\usepackage{titlesec}   % 設定section等的字體
\usepackage{titling}    % 加強 title 功能
\usepackage{fancyhdr}   % 頁首頁尾
\usepackage{tabularx}   % 加強版 table
\usepackage{multirow}   % colspan
\usepackage[square, comma, numbers, super, sort&compress]{natbib}
% cite加強版
\usepackage[unicode=true, pdfborder={0 0 0}, bookmarksdepth=-1]{hyperref}
% ref加強版
\usepackage[usenames, dvipsnames]{color}  % 可以使用顏色
\usepackage[shortlabels, inline]{enumitem}  % 加強版enumerate
\usepackage{xpatch}

% \usepackage{tabto}      % tab
% \usepackage{soul}       % highlight
% \usepackage{ulem}       % 字加裝飾
\usepackage{wrapfig}     % 文繞圖
%\usepackage{floatflt}    % 浮動 figure
\usepackage{float}       % 浮動環境
\usepackage{caption}    % caption 增強
\usepackage{subcaption}    % subfigures
% \usepackage{setspace}    % 控制空行
% \usepackage{mdframed}   % 可以加文字方框
% \usepackage{multicol}   % 多欄
\usepackage[abbreviations, per-mode=symbol]{siunitx}      % SI unit
% \usepackage{dsfont}     % more mathbb

%%%%%%%%%%%%%%%%%%%TikZ%%%%%%%%%%%%%%%%%%%%%%
% \usepackage{tikz}
\usepackage[siunitx, americanvoltages, americancurrents, arrowmos]{circuitikz}
\usetikzlibrary{calc}

%%%%%%%%%%%%%%中文 Environment%%%%%%%%%%%%%%%
\usepackage[CheckSingle, CJKmath]{xeCJK}  % xelatex 中文
\usepackage{CJKulem}	% 中文字裝飾
\setCJKmainfont[BoldFont=cwTeX Q Hei]{cwTeX Q Ming}
\setCJKsansfont[BoldFont=cwTeX Q Hei]{cwTeX Q Ming}
\setCJKmonofont[BoldFont=cwTeX Q Hei]{cwTeX Q Ming}
% 設定中文為系統上的字型,而英文不去更動,使用原TeX字型

% \XeTeXlinebreaklocale "zh"             %這兩行一定要加,中文才能自動換行
% \XeTeXlinebreakskip = 0pt plus 1pt     %這兩行一定要加,中文才能自動換行

%%%%%%%%%%%%%%%字體大小設定%%%%%%%%%%%%%%%
% \def\normalsize{\fontsize{10}{15}\selectfont}
% \def\large{\fontsize{40}{60}\selectfont}
% \def\Large{\fontsize{50}{75}\selectfont}
% \def\LARGE{\fontsize{90}{20}\selectfont}
% \def\huge{\fontsize{34}{51}\selectfont}
% \def\Huge{\fontsize{38}{57}\selectfont}

%%%%%%%%%%%%%%%Theme Input%%%%%%%%%%%%%%%%
% \input{themes/chapter/neat}
% \input{themes/env/problist}

%%%%%%%%%%%titlesec settings%%%%%%%%%%%%%%
% \titleformat{\chapter}{\bf\Huge}
            % {\arabic{section}}{0em}{}
% \titleformat{\section}{\centering\Large}
            % {\arabic{section}}{0em}{}
% \titleformat{\subsection}{\large}
            % {\arabic{subsection}}{0em}{}
% \titleformat{\subsubsection}{\bf\normalsize}
            % {\arabic{subsubsection}}{0em}{}
% \titleformat{command}[shape]{format}{label}
            % {編號與標題距離}{before}[after]

%%%%%%%%%%%%variable settings%%%%%%%%%%%%%%
% \numberwithin{equation}{section}
% \setcounter{secnumdepth}{4}  %章節標號深度
% \setcounter{tocdepth}{1}  %目錄深度
% \graphicspath{{images/}}  % 搜尋圖片目錄

%%%%%%%%%%%%%%%頁面設定%%%%%%%%%%%%%%%
\newcolumntype{C}[1]{>{\centering\arraybackslash}p{#1}}
\setlength{\headheight}{15pt}  %with titling
\setlength{\droptitle}{-1.5cm} %title 與上緣的間距
% \posttitle{\par\end{center}} % title 與內文的間距
\parindent=24pt %設定縮排的距離
% \parskip=1ex  %設定行距
% \pagestyle{empty}  % empty: 無頁碼
% \pagestyle{fancy}  % fancy: fancyhdr

% use with fancygdr
% \lhead{\leftmark}
% \chead{}
% \rhead{}
% \lfoot{}
% \cfoot{}
% \rfoot{\thepage}
% \renewcommand{\headrulewidth}{0.4pt}
% \renewcommand{\footrulewidth}{0.4pt}

% \fancypagestyle{firststyle}
% {
  % \fancyhf{}
  % \fancyfoot[C]{\footnotesize Page \thepage\ of \pageref{LastPage}}
  % \renewcommand{\headrule}{\rule{\textwidth}{\headrulewidth}}
% }

%%%%%%%%%%%%%%%重定義一些command%%%%%%%%%%%%%%%
\renewcommand{\contentsname}{目錄}  %設定目錄的標題名稱
\renewcommand{\refname}{參考資料}  %設定參考資料的標題名稱
\renewcommand{\abstractname}{\LARGE Abstract} %設定摘要的標題名稱

%%%%%%%%%%%%%%%特殊功能函數符號設定%%%%%%%%%%%%%%%
% \newcommand{\citet}[1]{\textsuperscript{\cite{#1}}}
\DeclarePairedDelimiter{\abs}{\lvert}{\rvert}
\DeclarePairedDelimiter{\norm}{\lVert}{\rVert}
\DeclarePairedDelimiter{\inpd}{\langle}{\rangle} % inner product
\DeclarePairedDelimiter{\ceil}{\lceil}{\rceil}
\DeclarePairedDelimiter{\floor}{\lfloor}{\rfloor}
\DeclareMathOperator{\adj}{adj}
\DeclareMathOperator{\sech}{sech}
\DeclareMathOperator{\csch}{csch}
\DeclareMathOperator{\arcsec}{arcsec}
\DeclareMathOperator{\arccot}{arccot}
\DeclareMathOperator{\arccsc}{arccsc}
\DeclareMathOperator{\arccosh}{arccosh}
\DeclareMathOperator{\arcsinh}{arcsinh}
\DeclareMathOperator{\arctanh}{arctanh}
\DeclareMathOperator{\arcsech}{arcsech}
\DeclareMathOperator{\arccsch}{arccsch}
\DeclareMathOperator{\arccoth}{arccoth}
\newcommand{\np}[1]{\\[{#1}] \indent}
\newcommand{\transpose}[1]{{#1}^\mathrm{T}}
%%%% Geometry Symbol %%%%
\newcommand{\degree}{^\circ}
\newcommand{\Arc}[1]{\wideparen{{#1}}}
\newcommand{\Line}[1]{\overleftrightarrow{{#1}}}
\newcommand{\Ray}[1]{\overrightarrow{{#1}}}
\newcommand{\Segment}[1]{\overline{{#1}}}

%%%% SI unit short cut %%%%
\newcommand{\siua}{\micro\ampere}
\newcommand{\sima}{\milli\ampere}
\newcommand{\simv}{\milli\volt}
\newcommand{\siko}{\kilo\ohm}
\newcommand{\sio}{\ohm}
\newcommand{\sia}{\ampere}
\newcommand{\siv}{\volt}

\newcommand{\img}{\mathrm{i}}
\newcommand{\ex}{\mathsf{e}}
\newcommand{\dD}{\mathrm{d}}
\newcommand{\iD}{\:\mathrm{d}}

%%%%%%%%%%%%%%%證明、結論、定義等等的環境%%%%%%%%%%%%%%%
\renewcommand{\proofname}{\bf 證明:} %修改Proof 標頭
\newtheoremstyle{mystyle}% 自定義Style
  {6pt}{15pt}%       上下間距
  {}%               內文字體
  {}%               縮排
  {\bf}%            標頭字體
  {.}%              標頭後標點
  {1em}%            內文與標頭距離
  {}%               Theorem head spec (can be left empty, meaning 'normal')

% 改用粗體,預設 remark style 是斜體
\theoremstyle{mystyle}	% 定理環境Style
\newtheorem{theorem}{定理}
\newtheorem{definition}{定義}
\newtheorem{formula}{公式}
\newtheorem{condition}{條件}
\newtheorem{supposition}{假設}
\newtheorem{conclusion}{結論}
\newtheorem{lemma}{引理}
\newtheorem{property}{性質}

%% Label set %%
\captionsetup[figure]{labelsep=period}

%% Ans %%
\newcommand{\Ans}{\noindent{\bf Ans:}}

%% No Indent %%
\setlength\parindent{0pt}

\newcommand{\red}{\color{red}}
\newcommand{\blue}{\color{blue}}

\newcommand{\paral}{\mathbin{\|}}
\newcommand\numeq{\addtocounter{equation}{1}\tag{\theequation}}


%% Note that for label to be correct, compile more than 2 times is needed ... %%

%%%%%%%%%%%%%%%Title的資訊%%%%%%%%%%%%%%%
\title{} %標題
\author{} %作者
\date{} %日期

\begin{document}
\tikzstyle{default}=[thick, color=black]
% \maketitle %製作tilte page
% \thispagestyle{empty}  %去除頁碼
% \thispagestyle{fancy}  %使用fancyhdr
% \tableofcontents %目錄
%%%%%%%%%%%%%%%%%%%include file here%%%%%%%%%%%%%%%%%%%%%%%%%

% remote:local

% 20:29
\section{4.20}
Measurements on the circuits of Figure~\ref{fig:4.20} produce labeled voltages as indicated. Find the value of $\beta$ for each transistor. 

\begin{figure}[H]
  \centering
  \begin{subfigure}{0.32\textwidth}
    \centering
    \begin{circuitikz}[scale=0.8, transform shape, >=triangle 45]
      \draw[default] 
      (0, 0) node[ground]{} to[R, l_=1<\kohm>] ++(0, 3) coordinate(v1) 
      node [pnp, anchor=C] (h1) {}
      (h1.B) -- ++ (-0.5, 0) coordinate(v2) to[R, *-, l_=200<\kohm>] ++(0, -3) node[ground]{}
      (v2) to[short, -o] ++(-0.5, 0) node[left]{$+\SI{4.3}{\V}$}
      (v1) to[short, -o] ++(.5, 0) node[right]{$+\SI{2}{\V}$}
        ;
      \draw[->, default] (h1.E) -- ++(0, 2) node[above]{$+\SI{5}{\V}$};
        
    \end{circuitikz}
  \caption{}
  \label{fig:4.20a}
  \end{subfigure}
  \begin{subfigure}{0.32\textwidth}
    \centering
    \begin{circuitikz}[scale=0.8, transform shape, >=triangle 45]
      \draw[default] 
      (0, 0) node[ground]{} to[R, l_=230<\ohm>, i<^={\blue $I_1$}] ++(0, 3) coordinate(v1) 
      node [pnp, anchor=C] (h1) {}
      (h1.B) to[short, -*] ++ (-1.5, 0) coordinate(v2) -- ++(0, -.75) to[R, -*, l_=20<\kohm>] (v1)
      (v2) to[short, -o] ++(-0.5, 0) node[left]{$+\SI{4.3}{\V}$}
      (v1) to[short, -o] ++(.5, 0) node[right]{$+\SI{2.3}{\V}$}
        ;
      \draw[->, default] (h1.E) -- ++(0, 2) node[above]{$+\SI{5}{\V}$};
        
    \end{circuitikz}
  \caption{}
  \label{fig:4.20b}
  \end{subfigure}
  \begin{subfigure}{0.32\textwidth}
    \centering
    \begin{circuitikz}[scale=0.8, transform shape, >=triangle 45]
      \draw[default] 
      (0, 0) node[ground]{} to[R, l_=1<\kohm>, i<^={\blue $I_1$}] ++(0, 3) coordinate(v1) 
      node [pnp, anchor=C] (h1) {}
      (h1.B) to[short, -*] ++ (-1.5, 0) coordinate(v2) -- ++(0, -.75) to[R, -*, l_=100<\kohm>] (v1)
      (v2) to[short, -o] ++(-0.5, 0) node[left]{$+\SI{6.3}{\V}$}
      (h1.E) to[R, l=1<\kohm>] ++ (0, 2.5) coordinate(v3)
      (h1.E) to[short, -o] ++(.5, 0) node[right]{$+\SI{7}{\V}$}
        ;
        \draw[->, default] (v3) -- ++(0, .5) node[above]{$+\SI{10}{\V}$};
        
    \end{circuitikz}
  \caption{}
  \label{fig:4.20c}
  \end{subfigure}
  \caption{}
  \label{fig:4.20}
\end{figure}

\Ans \\
Since $V_{EB} = \SI{0.7}{\V} > \SI{0.5}{\V}$ for all the circuits, so all the pnp are at active mode.
\begin{enumerate}[(a)]

% 25:34
\section{4.25}
Design the circuit in Figure~\ref{fig:4.25} to establish $I_C = \SI{0.1}{\mA}$ and $V_C = \SI{0.5}{\V}$. The transistor exhibits $v_{BE}$ of $\SI{0.7}{V}$ at $i_C = \SI{1}{\mA} \text{, and } \beta = 30$.

\begin{figure}[H]
  \centering
  \begin{circuitikz}[scale=0.8, transform shape, >=triangle 45]
    \draw[default] 
    (0, 0) to[R, l_=$R_E$] ++(0, 3) coordinate(v1) 
    node [npn, anchor=E] (h1) {}
    (h1.B) -- ++(-1, 0) node[ground]{}
    
    (h1.C) to[R, l_=$R_C$, i<^=$I_C$] ++ (0, 2.5) coordinate(v3)
    (h1.C) to[short, -o] ++(.5, 0) node[right]{$V_C$}
      ;
    \draw[->, default] 
    (v3) -- ++(0, .5) node[above]{$+\SI{1.5}{\V}$};
    \draw[->, default] 
    (0, 0) -- ++(0, -.5) node[below]{$-\SI{1.5}{\V}$};
      
  \end{circuitikz}
  \caption{}
  \label{fig:4.25}
\end{figure}

\Ans \\
Since $i = I_S \ex^{v/V_T}$, 
\[
  \ln \frac{I_C}{i_C} = \frac{V_{BE} - v_{BE}}{V_T} \quad \Rightarrow \quad V_{BE} \approx \SI{642}{\mV}
\]

\begin{align*}
  R_E &= \frac{V_E - \SI{1.5}{\V}}{I_E} = \frac{-V_{BE} - \SI{1.5}\V}{(\beta+1)I_C/\beta} && \approx \SI{8.3}{\kohm} \\
  R_C &= \frac{\SI{1.5}{\V} - V_C}{I_C} &&=  \SI{10}{\kohm}
\end{align*}

Another solution:  \\

First we can calculate $V_{BE}$ at $I_C = \SI{0.1}{\mA}$:
\[ V_{BE} = v_{BE} + \ln \left(\frac{I_C}{i_c}\right) \approx \SI{0.642}{\V}. \]
$V_C = \SI{0.5}{\V}$ when $I_C = \SI{0.1}{\mA}$, so
\[ I_C R_C = 1.5 - V_C \Rightarrow R_C = \SI{10}{\kohm}. \]
And $I_E = \frac{\beta + 1}{\beta} I_C \approx \SI{0.103}{\mA}$, so
\[ I_E R_E = 1.5 - V_{BE} \Rightarrow R_E \approx \SI{8.303}{\kohm}. \]

% 45:61
\section{4.45}
For the circuit in Figure~\ref{fig:4.45}, find $V_B, V_E, \text{ and } V_C$ for $R_B = \SI{100}{\kohm}, \SI{10}{\kohm}, \text{ and } \SI{1}{\kohm}$. Let $\beta = 100$.

\begin{figure}[H]
  \centering
  \begin{circuitikz}[scale=0.8, transform shape, >=triangle 45]
    \draw[default] 
    (0, 0) node[ground]{} to[R, l_=1<\kohm>] ++(0, 3) coordinate(v1) 
    node [npn, anchor=E] (h1) {}
    (h1.B) -- ++(-1, 0) to[R, l=$R_B$] ++ (0, 2.5) coordinate(v2)
    (h1.C) to[R, l_=1<\kohm>] ++ (0, 2.5) coordinate(v3)
    (h1.C) node[circ]{} ++ (.5, 0) node{\color{red}$V_C$}
    (h1.E) node[circ]{} ++ (.5, 0) node{\color{red}$V_E$}
    (h1.B) node[circ]{} ++ (-.5, -.5) node{\color{red}$V_B$}
      ;
    \draw[->, default] 
    (v3) -- ++(0, .5) ;
    \draw[->, default] 
    (v2) -- ++(0, 1.25);
    \node at (-1, 8) {$+\SI{5}{\V}$};
      
  \end{circuitikz}
  \caption{}
  \label{fig:4.45}
\end{figure}

\Ans \\
\begin{enumerate}[1.]
  \item $R_B = \SI{100}{\kohm}$ \\
    First we guess that the npn is at active mode. Then $V_{BE} = \SI{0.7}{\V}$. so
    \[
    \SI{5}{\V} - R_B \frac{1}{\beta+1} I_E - \SI{0.7}{\V} - (\SI{1}{\kohm}) I_E = 0 \quad \Rightarrow \quad I_E \approx \SI{2.16}{\mA} \]
    \begin{align*}
      V_E &= (\SI{1}\kohm) I_E && \approx \SI{2.16}{\V} \\
      V_B &= V_E + \SI{0.7}\V && \approx \SI{2.86}\V \\
      V_C &= \SI{5}\V - (\SI{1}\kohm) I_C = \SI{5}{\V} - (\SI{1}\kohm) \frac{\beta}{\beta+1} I_E && \approx \SI{2.86}\V
    \end{align*}
    Since $V_{BC} \approx \SI{0}{\V} < \SI{0.4}\V$, the assumption is correct.
  \item $R_B = \SI{10}\kohm$
    If we still assume the npn is at active mode, similarly, we would find
    \[
    \SI{5}{\V} - R_B \frac{1}{\beta+1} I_E - \SI{0.7}{\V} - (\SI{1}{\kohm}) I_E = 0 \quad \Rightarrow \quad I_E \approx \SI{3.91}{\mA} \]
    But then 
    \[ V_{BC} > V_{EC} = ( \SI{1}\kohm ) I_E - (\SI{5}\V - (\SI{1}\kohm) (\beta/(\beta+1)) I_E) \approx \SI{2.79}\V > \SI{0.4}{\V} \]
    So we change our assumption that the npn is working at saturation, and $V_B = V_C + 0.5 = V_E + 0.7$. Do a nodal analysis at npn yield
    \[ \frac{V_B - 5}{ R_B } + \frac{V_B - 0.5 - 5}{1} + \frac{V_B - 0.7}{1} = 0 \quad \Rightarrow \quad V_B \approx 3.19 \]
    \begin{alignat*}{2}
      V_C &= V_B - 0.5 && \quad \approx 2.69 \\
      V_E &= V_B - 0.7 && \quad \approx 2.49 \\
    \end{alignat*}
    Finally we check that 
    \[
    \beta' = \frac{I_C}{I_B} = \frac{(5 - V_C) / 1}{(5 - V_B) / 10} \approx 12.76 < 100 \]
    And hence the assumption is correct.
  \item $R_B = \SI{1}\kohm$
    By previous, We should assume the npn is at saturation mode.
    \[ \frac{V_B - 5}{ R_B } + \frac{V_B - 0.5 - 5}{1} + \frac{V_B - 0.7}{1} = 0 \quad \Rightarrow \quad V_B \approx 3.73 \]
    \begin{alignat*}{2}
      V_C &= V_B - 0.5 && \quad \approx 3.23 \\
      V_E &= V_B - 0.7 && \quad \approx 3.03 \\
    \end{alignat*}
    We check that 
    \[
    \beta' = \frac{I_C}{I_B} = \frac{(5 - V_C) / 1}{(5 - V_B) / 1} \approx 1.38 < 100 \]
    And hence the assumption is correct.
\end{enumerate}

% 47:63
\section{4.47}
For the circuits in Figure~\ref{fig:4.47}, find values for the labeled node voltages and branch currents. Assume $\beta = 100$.

\begin{figure}[H]
  \centering
  \begin{subfigure}{0.32\textwidth}
    \centering
    \begin{circuitikz}[scale=0.8, transform shape, >=triangle 45]
      \draw[default] 
      (0, 3) node[npn, anchor=E] (h1) {} to[I, l^=2<\mA>] (0, 0) 
      (h1.B) -- ++ (-0.5, 0) coordinate(v2) to[R, *-, l_=22<\kohm>] ++(0, -3) node[ground]{}
      (h1.C) to[R, l=1.6<\kohm>] ++(0, 2.5) coordinate(v3)
      (h1.E) to[short, *-o] ++(1, 0) node[right]{\red $V_1$}
      (h1.C) to[short, *-o] ++(1, 0) node[right]{\red $V_2$}
        ;
      \draw[->, default] (v3) -- ++(0, .5) node[above]{$+\SI{5}{\V}$};
      \draw[->, default] (0, 0) -- ++(0, -.5) node[below]{};
        
    \end{circuitikz}
  \caption{}
  \label{fig:4.47a}
  \end{subfigure}
  \begin{subfigure}{0.32\textwidth}
    \centering
    \begin{circuitikz}[scale=0.8, transform shape, >=triangle 45]
      \draw[default] 
      (0, 3) node[npn, anchor=E] (h1) {} to[R, l^=2.2<\kohm>, i>_={\color{red}$I_4$}] (0, 0) 
      (h1.B) -- ++ (-1, 0) node[ground]{}
      (h1.C) to[R, l=1.6<\kohm>] ++(0, 2.5) coordinate(v3)
      (h1.C) to[short, *-o] ++(1, 0) node[right]{\red $V_3$}
        ;
      \draw[->, default] (v3) -- ++(0, .5) node[above]{$+\SI{5}{\V}$};
      \draw[->, default] (0, 0) -- ++(0, -.5) node[below]{$\SI{-5}{\V}$};
        
    \end{circuitikz}
    \caption{}
    \label{fig:4.47b}
  \end{subfigure}
  \begin{subfigure}{0.32\textwidth}
    \centering
    \begin{circuitikz}[scale=0.8, transform shape, >=triangle 45]
      \draw[default] 
      (0, 3) node[npn, anchor=E] (h1) {} to[R, l^=2.2<\kohm>] (0, 0) 
      (h1.B) -- ++ (-0.5, 0) coordinate(v2) to[R, *-, l_=22<\kohm>] ++(0, -3) node[ground]{}
      (h1.C) to[R, l=1.6<\kohm>] ++(0, 2.5) coordinate(v3)
      (h1.E) to[short, *-o] ++(1, 0) node[right]{\red $V_5$}
      (h1.C) to[short, *-o] ++(1, 0) node[right]{\red $V_7$}
      (v2) to[short, -o] ++(-.5, 0) node[left]{\red $V_6$}
        ;
      \draw[->, default] (v3) -- ++(0, .5) node[above]{$+\SI{5}{\V}$};
      \draw[->, default] (0, 0) -- ++(0, -.5) node[below]{$-\SI{5}{\V}$};
    \end{circuitikz}
  \caption{}
  \label{fig:4.47c}
  \end{subfigure}
  \begin{subfigure}{0.4\textwidth}
    \centering
    \begin{circuitikz}[scale=0.8, transform shape, >=triangle 45]
      \draw[default] 
      (0, 3) node[npn, anchor=E] (h1) {} to[R, l^=5.1<\kohm>] (0, 0) 
      (h1.B) -- ++ (-1.5, 0) coordinate(v2) to[R, *-, l^=56<\kohm>] ++(0, 3) coordinate(v4)
      (h1.C) to[R, l_=3.3<\kohm>] ++(0, 2.5) coordinate(v3)
      (h1.E) to[short, *-o] ++(1, 0) node[right]{\red $V_9$}
      (h1.C) to[short, *-o] ++(1, 0) node[right]{\red $V_8$}
        ;
      \draw[->, default] (v3) -- ++(0, .5) node[above]{$+\SI{5}{\V}$};
      \draw[->, default] (0, 0) -- ++(0, -.5) node[below]{$-\SI{5}{\V}$};
      \draw[->, default] (v4) -- ++(0, .5) node[above]{$+\SI{1.2}{\V}$};
    \end{circuitikz}
  \caption{}
  \label{fig:4.47d}
  \end{subfigure}
  \begin{subfigure}{0.4\textwidth}
    \centering
    \begin{circuitikz}[scale=0.8, transform shape, >=triangle 45]
      \draw[default] 
      (0, 3) node[npn, anchor=E] (h1) {} to[R, l^=5.1<\kohm>] (0, 0) 
      (h1.B) -- ++ (-1.5, 0) coordinate(v2) to[R, *-, l^=91<\kohm>] ++(0, 3.25) coordinate(v4)
      (v2) to[R, l_=150<\kohm>] ++(0, -3.75) coordinate(v5)
      (h1.C) to[R, l_=3.3<\kohm>] ++(0, 2.5) coordinate(v3)
      (h1.E) to[short, *-o] ++(1, 0) node[right]{\red $V_{12}$}
      (h1.C) to[short, *-o] ++(1, 0) node[right]{\red $V_{11}$}
      (v2) to[short, -o] ++(-.5, 0) node[left]{\red $V_{10}$}
        ;
      \draw[->, default] (v3) -- ++(0, .5) node[above, xshift=-1cm]{$+\SI{5}{\V}$};
      \draw[->, default] (0, 0) -- ++(0, -.5) node[below, xshift=-1cm]{$-\SI{5}{\V}$};
      \draw[->, default] (v4) -- ++(0, .5);
      \draw[->, default] (v5) -- ++(0, -.5);
    \end{circuitikz}
  \caption{}
  \label{fig:4.47e}
  \end{subfigure}
  \caption{}
  \label{fig:4.47}
\end{figure}

\Ans \\

\begin{enumerate}
  \item \\
    First we assume that it is at active mode.
    \begin{alignat*}{2}
      V_B &  = \SI{22}\kohm \frac{\SI{2}\mA}{\beta + 1} && \quad \approx \SI{-0.43}{\V} \\
      V_1 &  = V_E = V_B - 0.7 &&\approx \SI{-1.13}\V \\
      V_2 &  = V_C = \SI{5}\V - \SI{1.6}\kohm \frac{\beta}{\beta + 1} \SI{2}\mA &&\approx \SI{1.83}\V
    \end{alignat*}
\end{enumerate}


% 51:68
\section{4.51}
Using $\beta = \infty$, design the circuit shown in Figure~\ref{fig:4.51} so that the bias currents in $Q_1, Q_2, \text{ and } Q_3$ are $\SI{2}{\mA}, \SI{2}{\mA}, \text{ and } \SI{4}{\mA}$, respectively, and $V_3 = 0, V_5 = \SI{-4}{\V}, \text{ and } V_7 = \SI{2}{\V}$. For each resistor, select the nearest standard value utilizing the table of standard values for $5\%$ resistors in Appendix G. Now, for $\beta = 100$, find the values of $V_3, V_4, V_5, V_6, \text{ and } V_7$.
\begin{figure}[H]
  \centering
  \begin{circuitikz}[>=triangle 45]
    \draw[default] 
    (0, 0) to[R, l=$R_1$, -*] ++(0, 2.5) node[npn, anchor=E] (q1){} to[short, -o] ++(.5, 0) node[below right]{\red $V_2$}
    (q1) node[xshift=.3cm] {$Q_1$}
    (q1.B) -- ++(-1, 0) node[ground]{}
    (q1.C) to[R=$R_2$] ++(0, 2.5) coordinate(o1)
    (q1.C) to[short, -*] ++(1, 0) node[pnp, anchor=B](q2){} to[short, -o] ++(0, -.5) node[below]{\red $V_3$}
    (q2) node[xshift=.3cm] {$Q_2$}
    (q2.E) to[R=$R_3$] ++(0, 2) coordinate(o3)
    (q2.E) to[short, *-o] ++(.5, 0) node[right]{\red $V_4$}
    (q2.C) to[R=$R_4$, *-] ++(0, -2.5) coordinate(o2)
    (q2.C) to[short] ++(1, 0) node[npn, anchor=B](q3){} to[short, *-o] ++(0, -.5) node[below]{\red $V_5$}
    (q3) node[xshift=.3cm] {$Q_3$}
    (q3.E) to[R=$R_6$] ++(0, -2.5) coordinate(o4)
    (q3.E) to[short, *-o] ++(.5, 0) node[right]{\red $V_6$}
    (q3.C) to[R=$R_5$] ++(0, 2.5) coordinate(o5)
    (q3.C) to[short, *-o] ++(.5, 0) node[right]{\red $V_7$}
    ; 
    \draw[default, ->] (0, 0) -- ++(0, -.5);
    \draw[default, ->] (o1) -- ++(0, .5);
    \draw[default, ->] (o2) -- ++(0, -1.25) node[below] {$\SI{-10}{\V}$};
    \draw[default, ->] (o3) -- ++(0, .25) node[above] {$+\SI{10}{\V}$};
    \draw[default, ->] (o4) -- ++(0, -.5);
    \draw[default, ->] (o5) -- ++(0, .5);
    
  \end{circuitikz}
  \caption{}
  \label{fig:4.51}
\end{figure}

\Ans \\

% 75:95  7.5\kohm <- 8.2\kohm
\section{4.75}
The transistor amplifier in Figure~\ref{fig:4.75} is biased with a current source $I$ and has a very high $\beta$. find the dc voltage at the collector, $V_C$. Also, find the value of $g_m$. Replace the transistor with the simplified hybrid-$\pi$ model of Figure~\ref{fig:4.75} (note that the dc current source $I$ should be replaced with an open circuit). Hence find the voltage gain $v_c/v_i$. 

\begin{figure}[H]
  \centering
  \begin{circuitikz}[transform shape, >=triangle 45]
    \draw[default] 
    (0, 2.5) to[I=$\SI{0.5}{\mA}$] (0, 0)
    (0, 2.5) node [npn, anchor=E, xscale=-1] (h1) {} to[C=$\infty$, *-o] ++(-2.5, 0) coordinate(v1)  ++(0, -2) node[ground]{} to[V, l^=$v_i$] (v1)  
    (h1.B) -- ++(1, 0) node[ground]{}
    
    (h1.C) to[R, l_=$\SI{7.5}{\kohm}$] ++ (0, 2.5) coordinate(v3)
    (h1.C) to[short, -o] ++(.5, 0) node[right]{\red $V_C + v_c$}
      ;
    \draw[->, default] 
    (v3) -- ++(0, .5) node[above]{$+\SI{5}{\V}$};
    \draw[->, default] 
    (0, 0) -- ++(0, -.5) ;
      
  \end{circuitikz}
  \caption{}
  \label{fig:4.75}
\end{figure}

\Ans \\

% 80:101
\section{4.80}
In the circuit shown in Figure~\ref{fig:4.80}, the transistor has a $\beta$ of 200. What is the dc voltage at the collector? Find the input resistances $R_{ib}$ and $R_{in}$ and the overall voltage gain $(v_o/v_sig)$. For an output signal of $\pm \SI{0.4}{\V}$, what values of $v_{sig}$ and $v_b$ are required?

\begin{figure}[H]
  \centering
  \begin{circuitikz}[transform shape, >=triangle 45]
    \draw[default] 
    (0, 0) to[I=$\SI{10}{\mA}$] ++(0, -2.5) node[pnp, anchor=E] (q1){} to[C, l_=$\infty$, *-] ++(2.5, 0) node[ground]{}
    (q1.C) to[R=$R_C{=}\SI{100}{\ohm}$] ++(0, -2.5) node[ground]{}
    (q1.C) to[short, *-o] ++(1, 0) node[right]{\red $v_o$}
    (q1.B) to[short, -o] ++(-.5, 0) node[above]{\red $v_b$} to[short, -*]  ++(-.5, 0) coordinate(v2) to[C, l_=$\infty$, -o] ++(-2.5, 0) to[R, l_=$R_{sig}{=}\SI{1}{\kohm}$] ++(-2.5, 0) ++(0, -2.5) node[ground]{} to[V=$v_{sig}$] ++(0, 2.5) 
    (v2) to[R, l^=$\SI{10}{\kohm}$] ++(0, 2.5) coordinate(o1)
      ;
    \draw[->, default] 
    (0, 0) -- ++(0, .5) node[above]{$+\SI{5}{\V}$};
    \draw[->, default] 
    (o1) -- ++(0, .5) node[above]{$+\SI{5}{\V}$};
    \draw[red, ->, rounded corners]
    (-2, -7) node[below]{$R_{ib}$}  |-  ++(1, 3);
    \draw[red, ->, rounded corners]
    (-4, -7) node[below]{$R_{in}$} |- ++(1, 3);
      
  \end{circuitikz}
  \caption{}
  \label{fig:4.80}
\end{figure}

\Ans \\

% 109:138
\section{4.109}
The circuit in Figure~\ref{fig:4.109} provides a constant current $I_O$ as long as the circuit to which the collector is connected maintains the BJT in the active mode. Show that
\[
  I_O = \alpha \frac{V_CC (R_2 / (R_1 + R_2)) - V_{BE}}{R_E + (R_1 \paral R_2) / (\beta + 1)} 
\]
\begin{figure}[H]
  \centering
  \begin{circuitikz}[transform shape, >=triangle 45]
    \draw[default] 
    (0, 0) node[ground]{} to[R=$R_2$] ++(0, 3.25) coordinate(v1) to[R=$R_1$] ++(0, 2.5) coordinate(o1)
    (v1) to[short, *-] ++(2, 0) node[npn, anchor=B] (q1) {}
    (q1.E) to[R=$R_E$] ++(0, -2.5) node[ground]{}
    (q1.C) to[short, i<_=$I_O$] ++(0, .5) coordinate(o2)
      ;
    \draw[dashed, default] 
    (o2) -- ++(0, 1);
    \draw[->, default] 
    (o1) -- ++(0, .5) node[above]{$V_{CC}$};
      
  \end{circuitikz}
\caption{}
\label{fig:4.109}
\end{figure}

\Ans\\

% 110:139
\section{4.110}
The current-source biasing circuit shown in Figure~\ref{fig:110} provides a bias current to $Q_1$ that is determined by the current source formed by $Q_2, R_1, R_2, \text{ and } R_E$. The bias current is independent of $R_B$ and nearly independent of $\beta_1$ (as long as both $Q_1$ and $Q_2$ operate in the active mode).
It is required to design the circuit using $\pm 5 \si{\V}$ dc supplies to establish $I_{C1} = \SI{0.1}{\mA}$ and $V_{CE1} = \SI{1.5}{\V}$, in the ideal situation of infinite $\beta_1$ and $\beta_2$. In designing the current source, use $\SI{2}{\V}$ dc voltage drop across $R_E$ and impose the requirement that $I_{E2}$ remain within $5\%$ of its ideal value of $\beta_2$ as low as $50$. 
In selecting a value for $R_B$, ensure that for the lowest value of $\beta_1 = 50$, $V_{CE2}$ is $\SI{2.5}{\V}$. Use standard $5 \%$ resistor values (see Appendix H). What values for $R_1, R_2, R_E, R_B, R_C$ do you choose? What values of $I_{C1} \text{ and } I_{CE1}$ result for $\beta_1 = \beta_2 = 50, 100, \text{ and } , 200$?

\begin{figure}[H]
\begin{center}
  \begin{circuitikz}[>=triangle 45, scale=1, transform shape]
    \draw[default]
    (0, 0) to[R=$R_E$] ++(0, 2.5) node[npn, anchor=E] (q2) {}
    ($(q2.E) !.5! (q2.C)$) node[right] {$Q_2$}
    (q2.B) to[short, -*] ++ (-1, 0) coordinate(v1) to[R=$R_1$, -*] ++(0, 2.5) 
    coordinate(v2) to[R=$R_B$] ++(0, 2) to[short, *-] ++(1, 0) node[npn, anchor=B](q1){}
    ($(q1.E) !.5! (q1.C)$) node[right] {$Q_2$}
    (v1) to[R=$R_2$] ++(0, -2.5) coordinate(o1)
    (v2) -- ++(-.5, 0) node[ground]{}
    (q1.E) -- (q2.C)
    (q1.C) to[R=$R_C$] ++(0, 2.5) coordinate(o2)
    ;
    \draw[default, ->] (0, 0) -- ++(0, -.5);
    \draw[default, ->] (o1) -- ++(0, -1.25);
    \draw[default, ->] (o2) -- ++(0, .5);
    \node[below] at (-1, -.5) {$-V_{EE}$};

  \end{circuitikz}
\end{center}
\caption{}
\label{fig:110}
\end{figure}

\Ans \\

% 117:147
\section{4.117}
The amplifier of Figure~\ref{fig:4.117} consists of two identical common-emitter amplifiers connected in cascade. Observe that the input resistance of the second stage, $R_{in2}$, constitutes the load resistance of the first stage.

\begin{enumerate}[(a)]
  \item For $V_{CC} = \SI{15}{\V}, R_1 = \SI{100}{\kohm}, R_2 = \SI{47}{\kohm}, R_E = \SI{3.9}{\kohm}, R_C = \SI{6.8}{\kohm}, \text{ and } \beta = 100$, determine the dc collector current and dc collector voltage of each transistor.
  \item Draw the small-signal equivalent circuit of the entire amplifier and give the values of all its components. Neglect $r_{o1} \text{ and } r_{o2}$.
  \item Find $R_{in1}$ and $v_{b1}/v_{sig}$ for $R_{sig} = \SI{5}{\kohm}$.
  \item Find $R_{in2}$ and $v_{b2}/v_{b1}$.
  \item For $R_L = \SI{2}{\kohm}$, find $v_o / v_{b2}$.
  \item Find the overall voltage gain $v_o / v_{sig}$.
\end{enumerate}
\begin{figure}[H]
  \centering
  \begin{circuitikz}[>=triangle 45, scale=.8, transform shape]
    \draw[default]
    (0, 0) node[ground]{} to[V=$v_{sig}$] ++(0, 3) to[R=$R_{sig}$] ++(2.5, 0) to[C=$\infty$, o-*] ++(2.5, 0) coordinate(v1) to[short] ++(.5, 0) node[npn, anchor=B](q1){}
    (v1) to[R, l^=$R_1$] ++(0, 3) coordinate(o1)
    (v1) to[R, l_=$R_2$] ++(0, -3) node[ground]{}
    ($(q1.C) !.5! (q1.E)$) node[right]{$Q_1$}
    (q1.E) to[R, l_=$R_E$, *-] ++(0, -2.25) node[ground]{}
    (q1.E) -- ++(1, 0) to[C=$\infty$] ++(0, -2.25) node[ground]{}
    (q1.C) to[R, l_=$R_C$] ++(0, 2.25) coordinate(o2)
    (q1.C) to[C=$\infty$] ++(4, 0) to[short] ++(0, -.75) coordinate(v2) to[short, *-o] ++(0, -.5) node[below]{$v_{b2}$}
    (v2) to[short, -*] ++(.5, 0) coordinate(v3) to[short] ++(.5, 0) node[npn, anchor=B](q2){}
    ($(q2.C) !.5! (q2.E)$) node[right]{$Q_2$}
    (v3) to[R=$R_1$] ++(0, 3) coordinate(o3)
    (v3) to[R, l_=$R_2$] ++(0, -3) node[ground]{}
    (q2.E) to[R, l_=$R_E$, *-] ++(0, -2.25) node[ground]{}
    (q2.E) -- ++(1, 0) to[C=$\infty$] ++(0, -2.25) node[ground]{}
    (q2.C) to[R, l_=$R_C$] ++(0, 2.25) coordinate(o4)
    (q2.C) to[short] ++(1, 0) to[C=$\infty$] ++(3, 0) coordinate(v4) to[R=$R_L$] ++(0, -3.75) node[ground]{}
    (v4) to[short, *-o] ++(.5, 0)
    ($(o1) !.5! (o2)$) node[above]{$V_{CC}$} node[yshift=0.8cm] {Stage 1}
    ($(o3) !.5! (o4)$) node[above]{$V_{CC}$} node[yshift=0.8cm] {Stage 2}
    (.5, 6.8) node{Source}
    (16, 6.8) node{Load}
    ;
    \draw[default, ->] (o1) -- ++(0, .2);
    \draw[default, ->] (o2) -- ++(0, .2);
    \draw[default, ->] (o3) -- ++(0, .2);
    \draw[default, ->] (o4) -- ++(0, .2);
    \draw[dashed] (2.2, -.5) -- (2.2, 6.5);
    \draw[dashed] (9, -.5) -- (9, 6.5);
    \draw[dashed] (14.3, -.5) -- (14.3, 6.5);
    \draw[red, ->] (1.8, -.5) node[below]{$R_{in1}$} |- ++(1, 2.7);
    \draw[red, ->] (8.8, -.5) node[below]{$R_{in2}$} |- ++(1, 2.7);
  \end{circuitikz}
  \caption{}
  \label{fig:4.117}
\end{figure}

\Ans
\begin{enumerate}[(a)]
  \item 
\end{enumerate}

% 122:152
\section{4.122}
For the emitter-follower circuit shown in Figure~\ref{fig:4.122}, the BJT used  is specified to have $\beta$ values in the range of $40 \text{ to } 200$ (a distressing situation for the circuit designer). For the two extreme values of $\beta$ ($\beta = 400$ and $\beta = 200$), find:

\begin{enumerate}
  \item $I_E, V_E, \text{ and } V_B$.
  \item the input resistance $R_{in}$.
  \item the voltage gain $v_o/v_{sig}$.
\end{enumerate}

\begin{figure}[H]
  \centering
  \begin{circuitikz}[>=triangle 45, transform shape]
    \draw[default]
    (0, 0) node[ground]{} to[V=$v_{sig}$] ++(0, 3) to[R=$\SI{10}{\kohm}$] ++(2.5, 0) to[C=$\infty$, o-*] ++(2.5, 0) coordinate(v1) to[short] ++(.5, 0) node[npn, anchor=B](q1){}
    (v1) to[R, l^=$\SI{100}{\kohm}$] ++(0, 3) coordinate(o1)
    (q1.E) to[R, l^=$\SI{1}{\kohm}$, *-] ++(0, -2.25) node[ground]{}
    (q1.E) to[C=$\infty$] ++(2.5, 0) coordinate(v2) to[R=$\SI{1}{\kohm}$] ++(0, -2.25) node[ground]{}
    (q1.C) -- ++(0, 2.25) coordinate(o2)
    (v2) to[short, *-o] ++(.5, 0) node[right]{\red $v_o$}
    ($(o1) !.5! (o2)$) node[above]{$+\SI{9}{\V}$}
    ;
    \draw[default, ->] (o1) -- ++(0, .2);
    \draw[default, ->] (o2) -- ++(0, .2);
    \draw[red, ->, rounded corners] (2.2, -.5) node[below]{$R_{in}$} |- ++(1, 2.7);
  \end{circuitikz}
  \caption{}
  \label{fig:4.122}
\end{figure}

\Ans
\begin{enumerate}
  \item 
\end{enumerate}

\end{document}

\documentclass[12pt, a4paper]{article}
%%%%%%%%%%%%%%%紙張大小設定%%%%%%%%%%%%%%%
% \paperwidth=65cm
% \paperheight=160cm

%%%%%%%%%%%%%%%引入Package%%%%%%%%%%%%%%%
\usepackage[margin=3cm]{geometry} % 上下左右距離邊緣2cm
\usepackage{mathtools,amsthm,amssymb} % 引入 AMS 數學環境
\usepackage{yhmath}      % math symbol
\usepackage{graphicx}    % 圖形插入用
\usepackage{fontspec}    % 加這個就可以設定字體
\usepackage{type1cm}	 % 設定fontsize用
\usepackage{titlesec}   % 設定section等的字體
\usepackage{titling}    % 加強 title 功能
\usepackage{fancyhdr}   % 頁首頁尾
\usepackage{tabularx}   % 加強版 table
\usepackage{multirow}   % colspan
\usepackage[square, comma, numbers, super, sort&compress]{natbib}
% cite加強版
\usepackage[unicode=true, pdfborder={0 0 0}, bookmarksdepth=-1]{hyperref}
% ref加強版
\usepackage[usenames, dvipsnames]{color}  % 可以使用顏色
\usepackage[shortlabels, inline]{enumitem}  % 加強版enumerate
\usepackage{xpatch}

% \usepackage{tabto}      % tab
% \usepackage{soul}       % highlight
% \usepackage{ulem}       % 字加裝飾
\usepackage{wrapfig}     % 文繞圖
%\usepackage{floatflt}    % 浮動 figure
\usepackage{float}       % 浮動環境
\usepackage{caption}    % caption 增強
\usepackage{subcaption}    % subfigures
% \usepackage{setspace}    % 控制空行
% \usepackage{mdframed}   % 可以加文字方框
% \usepackage{multicol}   % 多欄
\usepackage[abbreviations, per-mode=symbol]{siunitx}      % SI unit
% \usepackage{dsfont}     % more mathbb

%%%%%%%%%%%%%%%%%%%TikZ%%%%%%%%%%%%%%%%%%%%%%
% \usepackage{tikz}
\usepackage[siunitx, americanvoltages, americancurrents, arrowmos]{circuitikz}
\usetikzlibrary{calc}

%%%%%%%%%%%%%%中文 Environment%%%%%%%%%%%%%%%
\usepackage[CheckSingle, CJKmath]{xeCJK}  % xelatex 中文
\usepackage{CJKulem}	% 中文字裝飾
\setCJKmainfont[BoldFont=cwTeX Q Hei]{cwTeX Q Ming}
\setCJKsansfont[BoldFont=cwTeX Q Hei]{cwTeX Q Ming}
\setCJKmonofont[BoldFont=cwTeX Q Hei]{cwTeX Q Ming}
% 設定中文為系統上的字型,而英文不去更動,使用原TeX字型

% \XeTeXlinebreaklocale "zh"             %這兩行一定要加,中文才能自動換行
% \XeTeXlinebreakskip = 0pt plus 1pt     %這兩行一定要加,中文才能自動換行

%%%%%%%%%%%%%%%字體大小設定%%%%%%%%%%%%%%%
% \def\normalsize{\fontsize{10}{15}\selectfont}
% \def\large{\fontsize{40}{60}\selectfont}
% \def\Large{\fontsize{50}{75}\selectfont}
% \def\LARGE{\fontsize{90}{20}\selectfont}
% \def\huge{\fontsize{34}{51}\selectfont}
% \def\Huge{\fontsize{38}{57}\selectfont}

%%%%%%%%%%%%%%%Theme Input%%%%%%%%%%%%%%%%
% \input{themes/chapter/neat}
% \input{themes/env/problist}

%%%%%%%%%%%titlesec settings%%%%%%%%%%%%%%
% \titleformat{\chapter}{\bf\Huge}
            % {\arabic{section}}{0em}{}
% \titleformat{\section}{\centering\Large}
            % {\arabic{section}}{0em}{}
% \titleformat{\subsection}{\large}
            % {\arabic{subsection}}{0em}{}
% \titleformat{\subsubsection}{\bf\normalsize}
            % {\arabic{subsubsection}}{0em}{}
% \titleformat{command}[shape]{format}{label}
            % {編號與標題距離}{before}[after]

%%%%%%%%%%%%variable settings%%%%%%%%%%%%%%
% \numberwithin{equation}{section}
% \setcounter{secnumdepth}{4}  %章節標號深度
% \setcounter{tocdepth}{1}  %目錄深度
% \graphicspath{{images/}}  % 搜尋圖片目錄

%%%%%%%%%%%%%%%頁面設定%%%%%%%%%%%%%%%
\newcolumntype{C}[1]{>{\centering\arraybackslash}p{#1}}
\setlength{\headheight}{15pt}  %with titling
\setlength{\droptitle}{-1.5cm} %title 與上緣的間距
% \posttitle{\par\end{center}} % title 與內文的間距
\parindent=24pt %設定縮排的距離
% \parskip=1ex  %設定行距
% \pagestyle{empty}  % empty: 無頁碼
% \pagestyle{fancy}  % fancy: fancyhdr

% use with fancygdr
% \lhead{\leftmark}
% \chead{}
% \rhead{}
% \lfoot{}
% \cfoot{}
% \rfoot{\thepage}
% \renewcommand{\headrulewidth}{0.4pt}
% \renewcommand{\footrulewidth}{0.4pt}

% \fancypagestyle{firststyle}
% {
  % \fancyhf{}
  % \fancyfoot[C]{\footnotesize Page \thepage\ of \pageref{LastPage}}
  % \renewcommand{\headrule}{\rule{\textwidth}{\headrulewidth}}
% }

%%%%%%%%%%%%%%%重定義一些command%%%%%%%%%%%%%%%
\renewcommand{\contentsname}{目錄}  %設定目錄的標題名稱
\renewcommand{\refname}{參考資料}  %設定參考資料的標題名稱
\renewcommand{\abstractname}{\LARGE Abstract} %設定摘要的標題名稱

%%%%%%%%%%%%%%%特殊功能函數符號設定%%%%%%%%%%%%%%%
% \newcommand{\citet}[1]{\textsuperscript{\cite{#1}}}
\DeclarePairedDelimiter{\abs}{\lvert}{\rvert}
\DeclarePairedDelimiter{\norm}{\lVert}{\rVert}
\DeclarePairedDelimiter{\inpd}{\langle}{\rangle} % inner product
\DeclarePairedDelimiter{\ceil}{\lceil}{\rceil}
\DeclarePairedDelimiter{\floor}{\lfloor}{\rfloor}
\DeclareMathOperator{\adj}{adj}
\DeclareMathOperator{\sech}{sech}
\DeclareMathOperator{\csch}{csch}
\DeclareMathOperator{\arcsec}{arcsec}
\DeclareMathOperator{\arccot}{arccot}
\DeclareMathOperator{\arccsc}{arccsc}
\DeclareMathOperator{\arccosh}{arccosh}
\DeclareMathOperator{\arcsinh}{arcsinh}
\DeclareMathOperator{\arctanh}{arctanh}
\DeclareMathOperator{\arcsech}{arcsech}
\DeclareMathOperator{\arccsch}{arccsch}
\DeclareMathOperator{\arccoth}{arccoth}
\newcommand{\np}[1]{\\[{#1}] \indent}
\newcommand{\transpose}[1]{{#1}^\mathrm{T}}
%%%% Geometry Symbol %%%%
\newcommand{\degree}{^\circ}
\newcommand{\Arc}[1]{\wideparen{{#1}}}
\newcommand{\Line}[1]{\overleftrightarrow{{#1}}}
\newcommand{\Ray}[1]{\overrightarrow{{#1}}}
\newcommand{\Segment}[1]{\overline{{#1}}}

%%%% SI unit short cut %%%%
\newcommand{\siua}{\micro\ampere}
\newcommand{\sima}{\milli\ampere}
\newcommand{\simv}{\milli\volt}
\newcommand{\siko}{\kilo\ohm}
\newcommand{\sio}{\ohm}
\newcommand{\sia}{\ampere}
\newcommand{\siv}{\volt}

\newcommand{\img}{\mathrm{i}}
\newcommand{\ex}{\mathsf{e}}
\newcommand{\dD}{\mathrm{d}}
\newcommand{\iD}{\:\mathrm{d}}

%%%%%%%%%%%%%%%證明、結論、定義等等的環境%%%%%%%%%%%%%%%
\renewcommand{\proofname}{\bf 證明:} %修改Proof 標頭
\newtheoremstyle{mystyle}% 自定義Style
  {6pt}{15pt}%       上下間距
  {}%               內文字體
  {}%               縮排
  {\bf}%            標頭字體
  {.}%              標頭後標點
  {1em}%            內文與標頭距離
  {}%               Theorem head spec (can be left empty, meaning 'normal')

% 改用粗體,預設 remark style 是斜體
\theoremstyle{mystyle}	% 定理環境Style
\newtheorem{theorem}{定理}
\newtheorem{definition}{定義}
\newtheorem{formula}{公式}
\newtheorem{condition}{條件}
\newtheorem{supposition}{假設}
\newtheorem{conclusion}{結論}
\newtheorem{lemma}{引理}
\newtheorem{property}{性質}

%% Label set %%
\captionsetup[figure]{labelsep=period}

%% Ans %%
\newcommand{\Ans}{\noindent{\bf Ans:}}

%% No Indent %%
\setlength\parindent{0pt}

\newcommand{\red}{\color{red}}
\newcommand{\blue}{\color{blue}}

\newcommand{\paral}{\mathbin{\|}}
\newcommand\numeq{\addtocounter{equation}{1}\tag{\theequation}}


%% Note that for label to be correct, compile more than 2 times is needed ... %%


%%%%%%%%%%%%%%%Title的資訊%%%%%%%%%%%%%%%
\title{} %標題
\author{} %作者
\date{} %日期


\begin{document}
\tikzstyle{default}=[thick, color=black]
% \maketitle %製作tilte page
% \thispagestyle{empty}  %去除頁碼
% \thispagestyle{fancy}  %使用fancyhdr
% \tableofcontents %目錄
%%%%%%%%%%%%%%%%%%%include file here%%%%%%%%%%%%%%%%%%%%%%%%%

% remote:local

% 20:29
\section{4.20}
Measurements on the circuits of Figure~\ref{fig:} produce labeled voltages as indicated. Find the value of $\beta$ for each transistor. 

\begin{figure}[H]
  \centering
  \begin{subfigure}{0.32\textwidth}
    \centering
    \begin{circuitikz}[scale=0.8, transform shape, >=triangle 45]
      \draw[default] 
      (0, 0) node[ground]{} to[R, l_=10<\kohm>] ++(0, 3) coordinate(v1) 
      node [pnp, anchor=C] (h1) {}
      (h1.B) -- ++ (-0.5, 0) coordinate(v2) to[R, *-, l_=200<\kohm>] ++(0, -3) node[ground]{}
      (v2) to[short, -o] ++(-0.5, 0) node[left]{$+\SI{4.3}{\V}$}
      (v1) to[short, -o] ++(.5, 0) node[right]{$+\SI{2}{\V}$}
        ;
      \draw[->, default] (h1.E) -- ++(0, 2) node[above]{$+\SI{5}{\V}$};
        
    \end{circuitikz}
  \caption{}
  \label{fig:5.29a}
  \end{subfigure}
  \begin{subfigure}{0.32\textwidth}
    \centering
    \begin{circuitikz}[scale=0.8, transform shape, >=triangle 45]
      \draw[default] 
      (0, 0) node[ground]{} to[R, l_=230<\ohm>] ++(0, 3) coordinate(v1) 
      node [pnp, anchor=C] (h1) {}
      (h1.B) to[short, -*] ++ (-1.5, 0) coordinate(v2) -- ++(0, -.75) to[R, -*, l_=20<\kohm>] (v1)
      (v2) to[short, -o] ++(-0.5, 0) node[left]{$+\SI{4.3}{\V}$}
      (v1) to[short, -o] ++(.5, 0) node[right]{$+\SI{2}{\V}$}
        ;
      \draw[->, default] (h1.E) -- ++(0, 2) node[above]{$+\SI{5}{\V}$};
        
    \end{circuitikz}
  \caption{}
  \label{fig:5.29b}
  \end{subfigure}
  \begin{subfigure}{0.32\textwidth}
    \centering
    \begin{circuitikz}[scale=0.8, transform shape, >=triangle 45]
      \draw[default] 
      (0, 0) node[ground]{} to[R, l_=1<\kohm>] ++(0, 3) coordinate(v1) 
      node [pnp, anchor=C] (h1) {}
      (h1.B) to[short, -*] ++ (-1.5, 0) coordinate(v2) -- ++(0, -.75) to[R, -*, l_=100<\kohm>] (v1)
      (v2) to[short, -o] ++(-0.5, 0) node[left]{$+\SI{6.3}{\V}$}
      (h1.E) to[R, l=1<\kohm>] ++ (0, 2.5) coordinate(v3)
      (h1.E) to[short, -o] ++(.5, 0) node[right]{$+\SI{7}{\V}$}
        ;
        \draw[->, default] (v3) -- ++(0, .5) node[above]{$+\SI{10}{\V}$};
        
    \end{circuitikz}
  \caption{}
  \label{fig:5.29c}
  \end{subfigure}
  \caption{}
  \label{fig:5.29}
\end{figure}



% 25:34
\section{4.25}
Design the circuit in Figure~\ref{fig:4.25} to establish $I_C = \SI{0.1}{\mA}$ and $V_C = \SI{0.5}{\V}$. The transistor exhibits $v_{BE}$ of $\SI{0.7}{V}$ at $i_C = \SI{1}{\mA} \text{, and } \beta = 30$.


\begin{figure}[H]
  \centering
  \begin{circuitikz}[scale=0.8, transform shape, >=triangle 45]
    \draw[default] 
    (0, 0) to[R, l_=$R_E$] ++(0, 3) coordinate(v1) 
    node [npn, anchor=E] (h1) {}
    (h1.B) -- ++(-1, 0) node[ground]{}
    
    (h1.C) to[R, l_=$R_C$, i<^=$I_C$] ++ (0, 2.5) coordinate(v3)
    (h1.C) to[short, -o] ++(.5, 0) node[right]{$V_C$}
      ;
    \draw[->, default] 
    (v3) -- ++(0, .5) node[above]{$+\SI{1.5}{\V}$};
    \draw[->, default] 
    (0, 0) -- ++(0, -.5) node[below]{$-\SI{1.5}{\V}$};
      
  \end{circuitikz}
  \caption{}
  \label{fig:4.25}
\end{figure}

\Ans \\
\begin{center}
  \begin{tabular}{|l|c|c|c|c|c|c|c|}
    \hline
    & \multicolumn{6}{c|}{Voltage \si\V} & \\ \cline{2-7}
    Case & $V_S$ & $V_G$ & $V_D$ & $V_{GS}$ & $V_{OV}$ & $V_{DS}$ & Region of operation \\
    \hline
    a & $+1.0$ & $+1.0$ & $+2.0$ & $0$ & $0$ & $1.0$ & Cut-off \\
    b & $+1.0$ & $+2.5$ & $+2.0$ & $1.5$ & $0.5$ & $1.0$ & Saturation \\
    c & $+1.0$ & $+2.5$ & $+1.5$ & $1.5$ & $0.5$ & $0.5$ & Boundary of Sat./Tri. \\
    d & $+1.0$ & $+1.5$ & $0$ & $0.5$ & $0$ & $-1.0$ & Cut-off \\
    e & $0$ & $+2.5$ & $+1.0$ & $2.5$ & $1.5$ & $1$ & Triode \\
    f & $+1.0$ & $+1.0$ & $+1.0$ & $0$ & $0$ & $0$ & Cut-off \\
    g & $-1.0$ & $0$ & $0$ & $1.0$ & $0$ & $1.0$ & Boundary of Cut./Sat. \\
    h & $-1.5$ & $0$ & $0$ & $1.5$ & $0.5$ & $1.5$ & Saturation \\
    i & $-1.0$ & $0$ & $+1.0$ & $1.0$ & $0$ & $2.0$ & Boundary of Cut./Sat. \\
    j & $+0.5$ & $+2.0$ & $+0.5$ & $1.5$ & $0.5$ & $0$ & Triode \\
    \hline
  \end{tabular}
\end{center}

% 45:61
\section{4.45}
For the circuit in Figure~\ref{fig:}, find $V_B, V_E, \text{ and } V_C$ for $R_B = \SI{100}{\kohm}, \SI{10}{\kohm}, \text{ and } \SI{1}{\kohm}$. Let $\beta = 100$.

\begin{figure}[H]
  \centering
  \begin{circuitikz}[scale=0.8, transform shape, >=triangle 45]
    \draw[default] 
    (0, 0) node[ground]{} to[R, l_=1<\kohm>] ++(0, 3) coordinate(v1) 
    node [npn, anchor=E] (h1) {}
    (h1.B) -- ++(-1, 0) to[R, l=$R_B$] ++ (0, 2.5) coordinate(v2)
    (h1.C) to[R, l_=1<\kohm>] ++ (0, 2.5) coordinate(v3)
    (h1.C) node[circ]{} ++ (.5, 0) node{\color{red}$V_C$}
    (h1.E) node[circ]{} ++ (.5, 0) node{\color{red}$V_E$}
    (h1.B) node[circ]{} ++ (-.5, -.5) node{\color{red}$V_B$}
      ;
    \draw[->, default] 
    (v3) -- ++(0, .5) ;
    \draw[->, default] 
    (v2) -- ++(0, 1.25);
    \node at (-1, 8) {$+\SI{5}{\V}$};
      
  \end{circuitikz}
  \caption{}
  \label{fig:4.25}
\end{figure}


\Ans \\
For (a), $V_1 = \SI{3}{\V}$. For (b), $V_2 = \SI{-2}{\V}$. For (c), $V_3 = \SI{3}{\V}$. For (d), $V_4 = \SI{2}{\V}$.  \\
Notice that $V_{GD} = 0$ in each situation, so in order to remain saturation, $V_{GD} \geq V_t = \SI{1}{\V}$ in NMOS and $V_{GD} \leq \SI{-1}{\V}$ in PMOS, so $R_{\text{max}} = \abs{V_t} / I_D = \SI{0.5}{\kohm}$. \\
Finally, notice that $\abs{V_{CC} - V_{S}} = \SI{10}{\V}$ in each configuration \footnote{the problem is designed quite user-friendly..}, so $\abs{V_{DS} + V_R} = \abs{V_{CC} - V_{S}} - 2 = \SI{8}{\V}$, and since $\abs{V_{DS}} = \SI{3}{\V}$, $V_R = \SI{5}{\V}$, $R = V_R / I_D = \SI{2.5}{\kohm}$.

% 47:63
\section{4.47}
For the circuits in Figure~\ref{fig:4.47}, find values for the labeled node voltages and branch currents. Assume $\beta = 100$.

\begin{figure}[H]
  \centering
  \begin{subfigure}{0.32\textwidth}
    \centering
    \begin{circuitikz}[scale=0.8, transform shape, >=triangle 45]
      \draw[default] 
      (0, 3) node[npn, anchor=E] (h1) {} to[I, l^=2<\mA>] (0, 0) 
      (h1.B) -- ++ (-0.5, 0) coordinate(v2) to[R, *-, l_=22<\kohm>] ++(0, -3) node[ground]{}
      (h1.C) to[R, l=1.6<\kohm>] ++(0, 2.5) coordinate(v3)
      (h1.E) to[short, *-o] ++(1, 0) node[right]{\red $V_1$}
      (h1.C) to[short, *-o] ++(1, 0) node[right]{\red $V_2$}
        ;
      \draw[->, default] (v3) -- ++(0, .5) node[above]{$+\SI{5}{\V}$};
      \draw[->, default] (0, 0) -- ++(0, -.5) node[below]{};
        
    \end{circuitikz}
  \caption{}
  \label{fig:5.29a}
  \end{subfigure}
  \begin{subfigure}{0.32\textwidth}
    \centering
    \begin{circuitikz}[scale=0.8, transform shape, >=triangle 45]
      \draw[default] 
      (0, 3) node[npn, anchor=E] (h1) {} to[R, l^=2.2<\kohm>, i>_={\color{red}$I_4$}] (0, 0) 
      (h1.B) -- ++ (-1, 0) node[ground]{}
      (h1.C) to[R, l=1.6<\kohm>] ++(0, 2.5) coordinate(v3)
      (h1.C) to[short, *-o] ++(1, 0) node[right]{\red $V_3$}
        ;
      \draw[->, default] (v3) -- ++(0, .5) node[above]{$+\SI{5}{\V}$};
      \draw[->, default] (0, 0) -- ++(0, -.5) node[below]{$\SI{-5}{\V}$};
        
    \end{circuitikz}
    \caption{}
    \label{fig:5.29b}
  \end{subfigure}
  \begin{subfigure}{0.32\textwidth}
    \centering
    \begin{circuitikz}[scale=0.8, transform shape, >=triangle 45]
      \draw[default] 
      (0, 3) node[npn, anchor=E] (h1) {} to[R, l^=2.2<\kohm>] (0, 0) 
      (h1.B) -- ++ (-0.5, 0) coordinate(v2) to[R, *-, l_=22<\kohm>] ++(0, -3) node[ground]{}
      (h1.C) to[R, l=1.6<\kohm>] ++(0, 2.5) coordinate(v3)
      (h1.E) to[short, *-o] ++(1, 0) node[right]{\red $V_5$}
      (h1.C) to[short, *-o] ++(1, 0) node[right]{\red $V_7$}
      (v2) to[short, -o] ++(-.5, 0) node[left]{\red $V_6$}
        ;
      \draw[->, default] (v3) -- ++(0, .5) node[above]{$+\SI{5}{\V}$};
      \draw[->, default] (0, 0) -- ++(0, -.5) node[below]{$-\SI{5}{\V}$};
    \end{circuitikz}
  \caption{}
  \label{fig:5.29c}
  \end{subfigure}
  \begin{subfigure}{0.4\textwidth}
    \centering
    \begin{circuitikz}[scale=0.8, transform shape, >=triangle 45]
      \draw[default] 
      (0, 3) node[npn, anchor=E] (h1) {} to[R, l^=5.1<\kohm>] (0, 0) 
      (h1.B) -- ++ (-1.5, 0) coordinate(v2) to[R, *-, l^=56<\kohm>] ++(0, 3) coordinate(v4)
      (h1.C) to[R, l_=3.3<\kohm>] ++(0, 2.5) coordinate(v3)
      (h1.E) to[short, *-o] ++(1, 0) node[right]{\red $V_9$}
      (h1.C) to[short, *-o] ++(1, 0) node[right]{\red $V_8$}
        ;
      \draw[->, default] (v3) -- ++(0, .5) node[above]{$+\SI{5}{\V}$};
      \draw[->, default] (0, 0) -- ++(0, -.5) node[below]{$-\SI{5}{\V}$};
      \draw[->, default] (v4) -- ++(0, .5) node[above]{$+\SI{1.2}{\V}$};
    \end{circuitikz}
  \caption{}
  \label{fig:5.29c}
  \end{subfigure}
  \begin{subfigure}{0.4\textwidth}
    \centering
    \begin{circuitikz}[scale=0.8, transform shape, >=triangle 45]
      \draw[default] 
      (0, 3) node[npn, anchor=E] (h1) {} to[R, l^=5.1<\kohm>] (0, 0) 
      (h1.B) -- ++ (-1.5, 0) coordinate(v2) to[R, *-, l^=91<\kohm>] ++(0, 3.25) coordinate(v4)
      (v2) to[R, l_=150<\kohm>] ++(0, -3.75) coordinate(v5)
      (h1.C) to[R, l_=3.3<\kohm>] ++(0, 2.5) coordinate(v3)
      (h1.E) to[short, *-o] ++(1, 0) node[right]{\red $V_{12}$}
      (h1.C) to[short, *-o] ++(1, 0) node[right]{\red $V_{11}$}
      (v2) to[short, -o] ++(-.5, 0) node[left]{\red $V_{10}$}
        ;
      \draw[->, default] (v3) -- ++(0, .5) node[above, xshift=-1cm]{$+\SI{5}{\V}$};
      \draw[->, default] (0, 0) -- ++(0, -.5) node[below, xshift=-1cm]{$-\SI{5}{\V}$};
      \draw[->, default] (v4) -- ++(0, .5);
      \draw[->, default] (v5) -- ++(0, -.5);
    \end{circuitikz}
  \caption{}
  \label{fig:5.29c}
  \end{subfigure}
  \caption{}
  \label{fig:5.29}
\end{figure}

\Ans \\
Notice that every MOS is working on saturation region, since $v_D = v_G$,  hence $v_{DS} > v_{GS} - V_t$. Now for $Q_1, Q_3$, $V_{OV} = v_{GS} - V_t = \SI{0.5}{\V} $ and for $Q_2$, $V_{OV} = \SI{1}{\V} $.
Hence by solving 
\[
  \frac{1}{2} \mu_n C_{ox} \frac{W}{L} V_{OV}^2 = I_D = \SI{120}{\uA} 
\]
We get $W_1 = W_3 = \SI{8}{\um} , W_2 = \SI{2}{\um} $

% 51:68
\section{4.51}
Using $\beta = \infty$, design the circuit shown in Figure~\ref{fig:} so that the bias currents in $Q_1, Q_2, \text{ and } Q_3$ are $\SI{2}{\mA}, \SI{2}{\mA}, \text{ and } \SI{4}{\mA}$, respectively, and $V_3 = 0, V_5 = \SI{-4}{\V}, \text{ and } V_7 = \SI{2}{\V}$. For each resistor, select the nearest standard value utilizing the table of standard values for $5\%$ resistors in Appendix G. Now, for $\beta = 100$, find the values of $V_3, V_4, V_5, V_6, \text{ and } V_7$.
\begin{figure}[H]
  \centering
  \begin{circuitikz}[scale=0.8, transform shape, >=triangle 45]
    \draw[default] 
    (0, 0) to[R, l=$R_1$, -*] ++(0, 2.5) node[npn, anchor=E] (q1){} to[short, -o] ++(.5, 0) node[below right]{\red $V_2$}
    (q1) node[xshift=.3cm] {$Q_1$}
    (q1.B) -- ++(-1, 0) node[ground]{}
    (q1.C) to[R=$R_2$] ++(0, 2.5)
    (q1.C) to[short, -*] ++(1, 0) node[pnp, anchor=B](q2){} to[short, -o] ++(0, -.5) node[below]{\red $V_3$}
    (q2.E) to[R=$R_3$] ++(0, 2)
    (q2.E) to[short, *-o] ++(.5, 0) node[right]{\red $V_4$}
    ; 
    \draw[default, ->] (Q4.D) -- ++(0, 1);
    \draw[default, ->] (Q4.G) -| ++(-0.2, 1.75);
    \draw[default, ->] (Q2.D) -- ++(0, 1);
    \draw[default, ->] (Q2.G) -| ++(0.2, 1.75);
    \draw[dashed, color=blue] (p1) -- (p3);
  \end{circuitikz}
  \caption{}
  \label{fig:5.41}
\end{figure}


\Ans \\
Notice that if we connect the blue dashed line in the figure, the circuit is now symmetric. Moreover, all the MOS are working in saturation region.
Now $Q_1, Q_2, Q_3, Q_4$ has the same $k_n$, and the current go through $Q_4, Q_2$ is equal to the current that go through $Q_3, Q_1$. So by symmetry, $V_2 = V_1 =  \SI{5}{V} / 2 = \SI{2.5}{\V}$. Notice that there is no currenct go through the blue dashed line, so if we remove the line, the voltage remains unchanged. Hence we know that in the origin circuit, $V_2 = \SI{2.5}{\V}$.
\begin{align*}
  k_n &= \mu_n C_{ox} (W / L) = \SI{0.5}{\mA} \\
  V_2 &= v_{GS} = \SI{2.5}{\V} \\
  I_2 &= \frac{1}{2} k_n (v_{GS} - V_t) ^2 = \SI{562.5}{\uA}
\end{align*}
Simmilarly, if now $W_{3,4} = \SI{100}{\um}$, $k_n$ is $10$ times greater than before, so $I_2 = \SI{5.625}{\mA}$.

% 75:95  7.5\kohm <- 8.2\kohm
\section{4.75}
Consider the CS amplifier of Figure~\ref{fig:5.43} for the case $V_{DD} = \SI{5}{\V}, R_D = \SI{24}{\kohm}, k'_n(W/L) = \SI{1}{\mA\per\V\squared},\text{and } V_t = \SI{1}{\V}$.
\begin{enumerate}[(a)]
  \item Find the coordinates of the two end points of the saturation-region segment of the amplifier transfer characteristic, that is, points A and B on the sketch of Figure 5.29(b) (On textbook).
  \item If the amplifier is biased to operate with an overdrive voltage $V_{OV}$ of $\SI{0.5}{\V}$, find the coordinates of the bias point $Q_1$ on the transfer characteristic. Also, find the value of $I_D$ and of the incremental gain $A_v$ at the bias point.
  \item For the situation in (b), and disregarding the distortion caused by the MOSFET's square-law characteristic, what is the largest amplitude of a sine-wave voltage signal that can be applied at the input while the transistor remains in saturation? What is the amplitude of the output voltage signal that results? What gain value does the combination of these amplitudes imply? By what percentage is this gain value different from the incremental gain value calculated above? Why is there a difference?
\end{enumerate}

\begin{figure}[H]
  \centering
  \begin{circuitikz}[>=triangle 45]
    \draw[default] 
    (0, 0) node[ground]{} -- (0, 1.23) node[nmos, anchor=S](Q1){}
    (Q1.D) to[R, l=$R_D$, i<_={\color{red}$i_D$}] ++(0, 3) node[](vdd){}
    (Q1.G) to[short, o-] ++(-1, 0) node[](v+){}
    ($(v+) - (0, 1.5)$)node[](p1){} to[V, l=$v_{gs}$] (v+)
    ($(p1) - (0, 0.5)$)node[ground]{} to[battery1, l=$V_{GS}$] (p1)
    (Q1.D) to[short, *-o] ++(1, 0) node[right]{$V_{DS}$}
      ;
      
    \draw[default, ->] ($(vdd) -(0, 0.5)$) -- ++(0, 0.5) node[above]{$V_{DD}$};
  \end{circuitikz}
  \caption{}
  \label{fig:5.43}
\end{figure}


\begin{enumerate}[(a)]
  \item \Ans \\
    At point $A$, $V_{GS} = V_t = \SI{1}{\V}$, $V_{DS} = V_{DD} = \SI{5}{\V}$. At point $B$, 
    \[ V_{OV} = V_{DD} - R_D \frac{1}{2} k_n V_{OV}^2 \quad \Rightarrow
    \quad V_{OV} \approx \SI{0.605}{\V} \]
    So $V_{GS} \approx \SI{1.605}{\V}, V_{DS} = V_{GS} - V_t \approx \SI{0.605}{\V}$.
  \item \Ans \\
    If $V_{OV} = \SI{0.5}{\V}$, $I_D = \frac{1}{2} k_n V_{OV}^2 = \SI{0.125}{\mA}$.$V_{GS} = V_t + V_{OV} = \SI{1.5}{\V}, \; V_{DS} = V_{CC} - R_D I_D = \SI{2}{\V}$.
    \[
    A_{v} = -\frac{2 I_D R_D}{V_{OV}} = -\frac{2 (\SI{3}{\V}) }{\SI{0.5}{\V}} =  -\SI{12}{\V\per\V} \]
  \item \Ans \\
    By (a), $V_{GS,A} = \SI{1}{\V}, V_{GS, B} \approx \SI{1.605}{\V}$, so the maximum amplitute $\hat{V}_i = \abs{V_{GS, B} - V_{GS, Q}} \approx \SI{0.105}{\V}$.
    The amplitute of output is $\hat{V}_o \abs{V_{DS, Q} - V_{DS, B}} \approx \SI{1.395}{\V}$. The Gain $G = \hat{V}_o / \hat{V}_i \approx 13.28$. The error $\epsilon = (13.28 - 12)/13.28  \approx 9.6\%$. \\
    The different appears because the curve is quadratic but not linear.
\end{enumerate}

% 80:101
\section{4.80}
Figure~\ref{fig:5.46} shows an amplifier in which the load resistor $R_D$ has been replaced with another NMOS transistor $Q_1$ connected as a two-terminal device. Note that because $v_{DG}$ of $Q_2$ is zero, it will be operating in saturation at all times, even when $v_I = 0$ and $i_{D2} = i_{D1} = 0$. Note also that the two transistors conduct equal drain currents. Using $i_{D1} = i_{D2}$, show that for the range of $v_I$ over which $Q_1$ is operating in saturation, that is , for
\[ V_{t1} \leq v_I \leq v_O +V_{t1} \]
the output voltage will be given by
\[ v_O = V_{DD} - V_t + \sqrt{ \frac{W_1/L_1}{W_2/L_2} }V_t - \sqrt{ \frac{W_1/L_1}{W_2/L_2} }v_I \]

\begin{figure}[H]
  \centering
  \begin{circuitikz}[>=triangle 45]
    \draw[default] 
    (0, 0) node[ground]{} node[nmos, anchor=S](Q1){}
    (Q1.G) to[short, -o] ++(-0.5, 0) node[left]{$v_I$}
    (Q1.D) to[short, i^<={\color{red}$i_{D1}$}] ++(0, 0.5) node[](p1){} to[short, *-o] ++(0.5, 0) node[right]{$v_O$}
    (p1) to[short] ++(0, 0.5) node[nmos, anchor=S](Q2){}
    ($(Q2.D) + (0, 0.5)$) node[circ](p2){} -| (Q2.G)
    (p2) to[short, i^>={\color{red}$i_{D2}$}] (Q2.D)
    (Q1) node[right] {$Q_1$}
    (Q2) node[right] {$Q_2$}
      ;
      
    \draw[default, ->] (p2) -- ++(0, 0.5) node[above]{$V_{DD}$};
  \end{circuitikz}
  \caption{}
  \label{fig:5.46}
\end{figure}

\Ans
Since both the MOS is operating in saturation, we have
\begin{align*}
  i_{D1} &= \frac{1}{2} k'_n \frac{W_1}{L_1} (v_I - V_t)^2 \\
  i_{D2} &= \frac{1}{2} k'_n \frac{W_2}{L_2} (V_{DD} - v_O - V_t)^2 
\end{align*}

By $i_{D1} = i_{D2}$ we obtain
\begin{gather*}
  \frac{W_1}{L_1} (v_I - V_t)^2 = \frac{W_2}{L_2} (V_{DD} - v_O - V_t)^2 \\
  \Rightarrow v_O = V_{DD} - V_t + \sqrt{ \frac{W_1/L_1}{W_2/L_2} } (V_t - v_I) 
\end{gather*}
Which is the desired result.

% 109:138
\section{4.109}
In this problem we investigate an optimum design of the CS amplifier circuit of Figure~\ref{fig:5.43}. First. use the voltage gain expression $A_v = -g_m R_D$ together with Eq(5.57) for $g_m$ to show that
\[ A_v = -\frac{2I_D R_D}{V_{OV}} = - \frac{2(V_{DD} - V_D)}{V_{OV}} \]
Next, let the maximum positive input signal be $\hat{v_i}$. To keep the second-harmonic distortion to an acceptable level, we bias the MOSFET to operate at an overdrive voltage $V_{OV} \gg \hat{v_i}$. Let $V_{OV} = m\hat{v_i}$. Now, to maximize the voltage gain $\abs{A_v}$, we design for the lowest possible $V_D$. Show that the minimum $V_D$ that is consistent with allowing a negative signal voltage swing at the drain of $\abs{A_v}\hat{v_i}$ while maintaining saturation-mode operation is given by
\[
  V_D = \frac{V_{OV} + \hat{v_i} + 2 V_{DD} (\hat{v_i}/V_{OV})}{1 + 2(\hat{v_i}/V_{OV})} 
\]
Now, find $V_{OV}, V_D, A_v$, and $\hat{v_o}$ for the case $V_{DD} = \SI{3}{\V}, \hat{v_i} = \SI{20}{\mV}, \text{and } m = 10$. If it is desired to operate this transistor at $I_D = \SI{100}{\uA}$, find the values of $R_D$ and $W/L$, assuming that for this process technology $k'_n = \SI{100}{\uA\per\V\squared}$.

\Ans\\
\begin{align*}
  g_m &= \frac{2I_D}{V_{OV}}\\
  A_v &= -g_m R_D = -\frac{2 I_D R_D}{V_{OV}}  = -\frac{2 (V_{DD} - V_D)}{V_{OV}} 
\end{align*}
Let $v_g = V_G + v_i$ be the total gate voltage, $v_d = V_D + A_v v_i$ be the total drain voltage, the saturation condition required that $v_d \geq v_g - V_t$. so
\begin{align*}
  & v_d \geq v_g - V_t \\
  \Rightarrow & V_D + A_v v_i \geq V_G + v_i - V_t  \\
  \Rightarrow & V_D - \frac{2 (V_{DD} - V_D)}{V_{OV}} v_i \geq V_G + v_i - V_t \\
  \Rightarrow  & V_D - \frac{2 (V_{DD} - V_D)}{V_{OV}} \hat{v_i} \geq V_G + \hat{v_i} - V_t \quad \footnotemark \\
  \Rightarrow & \left(1 + \frac{2\hat{v_i}}{V_{OV}}\right) V_D \geq V_G + \hat{v_i} + \frac{2V_{DD}}{V_{OV}} \hat{v_i}  - V_t  \\
  \Rightarrow & V_D \geq \frac{V_{OV} + \hat{v_i} + 2 V_{DD} (\hat{v_i} / V_{OV}) }{1 + 2(\hat{v_i}/V_{OV})} 
\end{align*}
\footnotetext{Since $A_v, v_i$ and $v_i$ are in opposite sign, so in the worse case, we shall consider $-A_v \hat{v_i}$ and $\hat{v_i}$}
Which is the desired result.
Plug in the value and let $V_D$ be the maximum possible value we found out that
\begin{alignat*}{3}
  & V_{OV} &&= m \hat{v_i} && = \SI{200}{\mV} \\
  & V_{D} &&= \frac{V_{OV} + \hat{v_i} + 2 V_{DD} (\hat{v_i} / V_{OV}) }{1 + 2(\hat{v_i}/V_{OV})} && \approx \SI{683}{\mV} \\
  & A_v &&= -\frac{2(V_{DD} - V_D)}{V_{OV}}  && \approx -23.2 \\
  & \hat{v_o} &&= \abs{A_v} \hat{v_i} && \approx \SI{464}\mV \\
\end{alignat*}
If moreover, $I_D = \SI{100}{\uA}$,
\begin{alignat*}{3}
  & R_D &&= \frac{V_{DD} - V_D}{I_D}  && \approx \SI{23.17}{\kohm} \\
  & \frac{W}{L} &&= I_D / \left(\frac{1}{2} k'_n V_{OV}^2 \right) && = 50
\end{alignat*}

% 110:139
\section{4.110}
Figure~\ref{fig:5.56} shows a discrete-circuit amplifier. The input signal $v_{sig}$ is coupled to the gate through a very large capacitor (shown as infinite). The transistor source is conected to ground at signal frequencies via a very large capacitor (shown as infinite). The output voltage signal that develops at the drain is coupled to a load resistance via a very large capacitor(show as infinite).

\begin{enumerate}[(a)]
  \item If the transistor has $V_t = \SI{1}{\V}$, and $k_n = \SI{2}{\mA\per\V\squared}$, verify that the bias circuit establishes $V_{GS} = \SI{2}{\V}, I_D = \SI{1}{\mA}$, and $V_D = + \SI{7.5}{\V}$. That is, assume these values, and verify that they are consistent with the values of the circuit components and the device parameters.
  \item Find $g_m$ and $r_o$ if $V_A = \SI{100}{\V}$.
  \item Draw a complete small-signal equivalent circuit for the amplifier, assuming all capacitors behave as short circuits at signal frequencies.
  \item Find $R_{in}, v_{gs}/v_{sig}, v_o/v_{gs}, \text{and } v_o/v_{sig}$.
\end{enumerate}

\begin{figure}[H]
\begin{center}
  \begin{circuitikz}[>=triangle 45, scale=1, transform shape]
    \draw[default]
    (0, 0) node[ground]{} to[V, l=$v_{sig}$] ++(0, 3) to[R, l=100<\kohm>, -o] ++(2, 0) to[C, l=$\infty$, -*] ++(2, 0) to[short, o-] ++(1, 0) 
    coordinate(v1) to[R, l^=5<\mega\ohm>] ++(0, -3) node[ground]{}
    (v1) to[R, l=10<\mega\ohm>] ++(0, 3) coordinate(v5)
    (v1) to[short] ++(1, 0) node[nmos, anchor=G](q1){}
    (q1.S) to[short] ++(0, -.5) coordinate(v2) to[R, l=3<\kohm>] ++(0, -1.75) node[ground]{}
    (v2) to[C, l=$\infty$] ++(2, 0) node[ground]{}
    (q1.D) coordinate(v3) to[R, l=7.5<\kohm>, *-] ++(0, 2) coordinate(v4)
    (v3) to[C, l=$C_{C2}$, -*] ++(3, 0) coordinate(v6) to[R, l=10<\kohm>] ++(0, -2) node[ground]{} 
    (v6) to[short, -o] ++(.5, 0) node[right]{\color{red}$v_o$}
    ($(v5)!0.5!(v4)$) node[]{$+\SI{15}{\V}$}
    ;
    \draw[default, ->] (v5) -- ++(0, .5) ;
    \draw[default, ->] (v4) -- ++(0, .75) ;
    \draw[default, ->, red] (3.6, 0) node[below]{$R_{in}$} |- ++(1, 2.5);

  \end{circuitikz}
\end{center}
\caption{}
\label{fig:5.56}
\end{figure}

\begin{enumerate}[(a)]
  \item \Ans \\
    Simply check that 
    \begin{alignat*}{3}
      V_G &= \frac{5}{10+5} \cdot \SI{15}{\V} &&= \SI{5}{\V} \\
      V_S &= (\SI{3}{\kohm}) I_D &&= \SI{3}{\V} \\
      V_{GS} &= V_{G} - V_{S} &&= \SI{2}{\V} \quad \text{(check)}\\
      V_D &= 15 - \SI{7.5}{\kohm} \cdot I_D &&= \SI{7.5}{\V} \\
      V_D & \geq V_G - V_t, V_G \geq V_S + V_t && \quad\text{(check in saturation)} \\
      k_n &= \frac{2 I_D}{(V_{GS} - V_t)^2} &&= \SI{2}{\mA\per\V\squared}
      \quad \text{(check)}
    \end{alignat*}
  \item \Ans \\
    \begin{alignat*}{3}
      g_m &= \frac{2 I_D}{V_{OV}} = \frac{2 I_D}{V_{GS} - V_t} && = \SI{2}{\mA\per\V} \\
      r_o &= \frac{V_A}{I_D} &&= \SI{100}{\kohm}
    \end{alignat*}
  \item 
    \begin{figure}[H]
    \begin{center}
      \begin{circuitikz}[>=triangle 45, scale=1, transform shape]
        \draw[default]
        (0, 0) node[ground]{} to[V, l=$v_{sig}$] ++(0, 3) to[R, l=100<\kohm>, -o]  ++(3, 0) coordinate(v1) to[R, l_=5<\mega\ohm>] ++(0, -3) node[ground]{}
        (v1) to[short] ++(1, 0) coordinate(v2) to[R, l=10<\mega\ohm>]  ++(0, -3) node[ground]{}
        (v2) to[short] ++(2.5, 0) coordinate(v3) to [R, l=$1/g_m$, i=$I$] ++(0, -3) node[ground]{} 
        (v3)  ++(0, 3) coordinate(uu) to[short] ++(1, 0) coordinate(v4) to[R, l=$r_o$] ++(0, -3) node[ground]{} 
        (v4) to[short] ++(1.5, 0) coordinate(v5) to[R, l=5<\kohm>] ++(0, -3) node[ground]{}
        (v5) to[short] ++(2, 0) to[R, l=7.5<\kohm>] ++(0, -3) node[ground]{}
        (uu) to[cI, l_=$I$] (v3)
        ;

      \end{circuitikz}
    \end{center}
    \caption{}
    \label{fig:}
    \end{figure}
  \item \Ans \\
    \begin{alignat*}{3}
      R_{in} &= (\SI{10}{\mega\ohm}) \paral (\SI{5}{\mega\ohm}) &&\approx \SI{3.33}{\mega\ohm} \\
      \frac{v_{gs}}{v_{sig}}  &= \frac{R_{in}}{R_{in} + (\SI{3.33}{\mega\ohm})} && \approx 0.97 \\
      \frac{v_{o}}{v_{gs}}  &= g_m (\SI{7.5}{\kohm} \paral \SI{10}{\kohm} \paral r_o) && \approx 8.22 \\
      \frac{v_{o}}{v_{sig}} &= \frac{v_{gs}}{v_{sig}} \frac{v_{o}}{v_{gs}} \approx 7.97
    \end{alignat*}

\end{enumerate}

Another solution \\
\begin{enumerate}[(a)]
  \item \Ans \\$I_D = \SI{1}{\mA} \Rightarrow V_S = \SI{3}{\V}$, and $V_G = \SI{15}{\V}
    \cdot 5/(5+10) = \SI{5}{\V}$. So $V_{GS} = V_G - V_S = \SI{2}{\V}$ and
    $V_D = \SI{15}{\V} - \SI{7.5}{\kohm} \cdot \SI{1}{\mA} = \SI{7.5}{\V}$
    which is the desired result.
  \item \Ans \\
    $V_{OV} = V_{GS} - V_t = \SI{1}{\V}$, so $g_m = k_n V_{OV} = \SI{2}{\mA/\V}$.
    $r_o = V_A / I_D = \SI{100}{\kohm}$.
  \item \Ans \\
    For small-signal we can remove all large capacitor, change the two upper
    terminals to ground, and use T model to substitute the transistor.
    (Don't forget $r_o$).
  \item \Ans \\
    Since there is no current flowing into the gate of transistor, 
    $R_{in} = \SI{10}{\mega\ohm} \parallel \SI{5}{\mega\ohm} = \SI{3.33}{\mega\ohm}$.
    And for small signal, $v_d = 0$, so $v_{gs} = v_g = v_{sig} \cdot
    R_{in} / (R_{in} + \SI{100}{\kohm}) \Rightarrow v_{gs}/v_{sig} = 0.97$.
    Notice that $r_o$ connect drain to ground, so we have $v_o/v_{gs} =
    -g_m (r_o \parallel \SI{7.5}{\kohm} \parallel \SI{10}{\kohm}) = 8.22$.
    $v_o / v_{sig} = 8.22 \cdot 0.97 = 7.97$.
\end{enumerate}

% 117:147
\section{4.117}
The bias circuit of Figure~\ref{fig:5.72} is used in a design with $V_G = \SI{5}{\V}$ and $R_S = \SI{1}{\kohm}$. For an enhancement MOSFET with $k'_n(W/L) = \SI{2}{\mA\per\V\squared}$, the source voltage was measured and found to be $\SI{2}{\V}$. What must $V_t$ be for this device? If a device for which $V_t$ is $\SI{0.5}{\V}$ less is used, what does $V_s$ become? What bias current result?
\begin{figure}[H]
  \centering
  \begin{circuitikz}[>=triangle 45, scale=1, transform shape]
    \draw[default]
    (0, 0) node[ground]{} to[R] (0, 3) to[R] (0, 6) 
    (3, 0) node[ground]{} to[R, l=$R_S$, -*] (3, 2.23) node[right]{$V_S$} node[nmos, anchor=S](q1){}
    (0, 3) to[short, -*] (q1.G) node[above left]{$V_G$}
    (q1.D) to[R] (3, 6)
    (1.5, 7) node[]{$V_{DD}$}
    ;
    \draw[default, ->] (0, 6) -- (0, 6.5);
    \draw[default, ->] (3, 6) -- (3, 6.5);
  \end{circuitikz}
  \caption{}
  \label{fig:5.72}
\end{figure}
\Ans \\
Since $I_D = V_S / R_S = \SI{2}{\V} / \SI{1}{\kohm} = \SI{2}{\mA}$, so
\begin{align*}
  & \frac{1}{2} k'_n(W/L) (V_G - V_S - V_t)^2 = I_D \\
  \Rightarrow & (\SI{3} - V_t)^2 = \SI{2}{} \\
  \Rightarrow & V_t \approx \SI{1.59}{\V} 
\end{align*} 
If now $V'_t = V_t - \SI{0.5}{\V} = \SI{1.09}{\V}$, We have
\begin{align*}
  & \frac{1}{2} (\SI{2}{\mA\per\V}) ((\SI{5}{\V}) - V_S - (\SI{1.09}{\V}))^2 = V_S / \SI{1}{\kohm}  \\
  \Rightarrow & V_S \approx \SI{2.37}{\V} \quad \text{ \footnotemark}\\
  \Rightarrow & I_S = V_S / R_S \approx \SI{2.37}{\mA}
\end{align*}
\footnotetext{There are two solutions, but the other one would let $V_S > V_G$.} 

% 122:152
\section{4.122}
The NMOS transistor in the CS amplifier shown in Figure~\ref{fig:5.80} has $V_t = \SI{0.7}{\V}$ and $V_A = \SI{50}{\V}$.

\begin{enumerate}[(a)]
  \item Neglecting the Early effect \footnote{Early effect is the same as channel-length modulation, recall that $V_A$ is called Early voltage. (J.M Early is the one to discover this effect)}
    , verify that the MOSFET is operating in saturation with $I_D = \SI{0.5}{\mA}$  and $V_{OV} = \SI{0.3}{\V}$. What must the MOSFET's $k_n$ be? What is the dc voltage at the drain?
  \item Find $R_{in}$ \footnote{It seems that the other problem $R_{in}$ is the resistance looking into from $v_G$ but not $v_{sig}$, so the safest way is to write both the answer}
    and $G_v$.
  \item If $v_{sig}$ is a sinusoid with a peak amplitude $\hat{v}_{sig}$, find the maximum allowable value of $\hat{v}_{sig}$ for which the transistor remains in saturation. What is the corresponding amplitude of the output voltage?
  \item What is the value of resistance $R_s$ that needs to be inserted in series with capacitor $C_S$ in order to allow us to double the input signal $\hat{v}_{sig}$? What output voltage now results?
\end{enumerate}

\begin{figure}[H]
\begin{center}
  \begin{circuitikz}[>=triangle 45, scale=1, transform shape]
    \draw[default]
    (0, 0) node[ground]{} to[V, l=$v_{sig}$] ++(0, 3) to[R, l=120<\kohm>, -o] ++(2, 0) to[C, l=$C_{C1}$, -*] ++(3, 0) 
    coordinate(v1) to[R, l=200<\kohm>] ++(0, -3) node[ground]{}
    (v1) to[R, l=300<\kohm>] ++(0, 3) coordinate(v5)
    (v1) to[short] ++(1, 0) node[nmos, anchor=G](q1){}
    (q1.S) to[short] ++(0, -.5) coordinate(v2) to[R, l=2<\kohm>] ++(0, -1.75) node[ground]{}
    (v2) to[C, l=$C_S$] ++(2, 0) node[ground]{}
    (q1.D) coordinate(v3) to[R, l=5<\kohm>, *-] ++(0, 2) coordinate(v4)
    (v3) to[C, l=$C_{C2}$, -*] ++(3, 0) coordinate(v6) to[R, l=5<\kohm>] ++(0, -2) node[ground]{} 
    (v6) to[short, -o] ++(.5, 0) node[right]{$v_o$}
    ($(v5)!0.5!(v4)$) node[]{$+\SI{5}{\V}$}
    ;
    \draw[default, ->] (v5) -- ++(0, .5) ;
    \draw[default, ->] (v4) -- ++(0, .75) ;

  \end{circuitikz}
\end{center}
\caption{}
\label{fig:5.80}
\end{figure}
\begin{enumerate}[(a)]
  \item \Ans \\
    \begin{alignat*}{3}
      V_G &= \frac{200}{300+200} (\SI{5}{\V}) &&= \SI{2}{\V} \\
      V_S &= (\SI{2}{\kohm}) (\SI{0.5}{\mA}) &&= \SI{1}{\V} \\
      V_D &= \SI{5}{\V} - (\SI{5}{\kohm}) (\SI{0.5}{\mA}) &&= \SI{2.5}{\V} \\
    \end{alignat*}
    Since $V_D \geq V_G - V_t, \; V_G \geq V_S + V_t$, the nmos is indeed operate in saturation region.
    \[
      k_n = \frac{2 I_D}{V_{OV}^2} = \SI{11.11}{\mA\per\V\squared}
    \]
  \item \Ans \\
    Calculate $r_o = V_A / I_D = \SI{100}{\kohm}$.
    \[
      R_{in} = \SI{300}{\kohm} \paral \SI{200}{\kohm} = \SI{120}{\kohm} \]
    And
    \begin{align*}
      g_m &= \frac{2 I_D}{V_{OV}} = \SI{3.33}{\mA\per\V} \\
      v_{i} &= \frac{R_{in}}{\SI{120}{\kohm} + R_{in}} v_{sig} = \frac{1}{2} v_{sig}  \\
      G_v &= v_o / v_{sig} = \frac{1}{2} v_o / v_i = -\frac{1}{2} g_m (\SI{5}{\kohm} \paral \SI{5}{\kohm} \paral r_o) \approx \SI{-4.05}{\V\per\V} \\
    \end{align*}
  \item 
    Remains in saturation requires $v_d + V_D \geq v_g + V_G - V_{t}$. Notice that \\ $v_d = v_o = G_v v_{sig}, \; v_g = \frac{1}{2} v_{sig}$, so
    \[
      -4.05 \hat{v}_{sig} + \SI{2.5}{\V} = \frac{1}{2} \hat{v}_{sig} + \SI{2}{\V} - \SI{0.7}{\V} \quad \Rightarrow \quad 
      \hat{v}_{sig} \approx \SI{263}{\mV} \numeq \label{eq:80-1}
    \]
    \[
      \hat{v}_{o} = \hat{v}_{sig} G_v \approx \SI{1.07}{\V}
    \]
  \item 
    By modifing Equation \eqref{eq:80-1} (let $-4.05 \rightarrow G'_v$)
    We have
    \[
     \SI{1.2}{\V} = -G'_v \cdot 2\hat{v}_{sig} + \frac{1}{2} \cdot 2\hat{v}_{sig} \quad \Rightarrow \quad G'_v \approx -1.78
    \]
    if we let $R = R_S \paral \SI{2}{\kohm}$, $R_D = \SI{5}{\kohm} \paral \SI{5}{\kohm}$
    , By using T model we obtain \footnote{a little complicate calculation}
    \[
      2G'_v = -g_m R_D/ (1 + g_mR + (R+R_D)/r_o) \quad \Rightarrow \quad
      R = \SI{393}{\ohm}
    \]
    finally,
    \[
      \frac{\SI{2}{\kohm} \cdot R_S}{\SI{2}{\kohm} + R_S } = \SI{393}{\ohm}
      \quad \Rightarrow \quad R_S \approx \SI{489}{\ohm}
    \]
    \[
      \hat{v}_o = \abs{G'_v} 2\hat{v}_{sig} \approx  \SI{936}{\mV}
    \]
    
\end{enumerate}



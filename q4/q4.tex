\documentclass[12pt, a4paper]{article}
\input{header.tex}

%% Note that for label to be correct, compile more than 2 times is needed ... %%


%%%%%%%%%%%%%%%Title的資訊%%%%%%%%%%%%%%%
\title{} %標題
\author{} %作者
\date{} %日期


\begin{document}
\tikzstyle{default}=[thick, color=black]
% \maketitle %製作tilte page
% \thispagestyle{empty}  %去除頁碼
% \thispagestyle{fancy}  %使用fancyhdr
% \tableofcontents %目錄
%%%%%%%%%%%%%%%%%%%include file here%%%%%%%%%%%%%%%%%%%%%%%%%

% remote:local

% 20:29
\section{4.20}
Measurements on the circuits of Figure~\ref{fig:} produce labeled voltages as indicated. Find the value of $\beta$ for each transistor. 

\begin{figure}[H]
  \centering
  \begin{subfigure}{0.32\textwidth}
    \centering
    \begin{circuitikz}[scale=0.8, transform shape, >=triangle 45]
      \draw[default] 
      (0, 0) node[ground]{} to[R, l_=10<\kohm>] ++(0, 3) coordinate(v1) 
      node [pnp, anchor=C] (h1) {}
      (h1.B) -- ++ (-0.5, 0) coordinate(v2) to[R, *-, l_=200<\kohm>] ++(0, -3) node[ground]{}
      (v2) to[short, -o] ++(-0.5, 0) node[left]{$+\SI{4.3}{\V}$}
      (v1) to[short, -o] ++(.5, 0) node[right]{$+\SI{2}{\V}$}
        ;
      \draw[->, default] (h1.E) -- ++(0, 2) node[above]{$+\SI{5}{\V}$};
        
    \end{circuitikz}
  \caption{}
  \label{fig:5.29a}
  \end{subfigure}
  \begin{subfigure}{0.32\textwidth}
    \centering
    \begin{circuitikz}[scale=0.8, transform shape, >=triangle 45]
      \draw[default] 
      (0, 0) node[ground]{} to[R, l_=230<\ohm>] ++(0, 3) coordinate(v1) 
      node [pnp, anchor=C] (h1) {}
      (h1.B) to[short, -*] ++ (-1.5, 0) coordinate(v2) -- ++(0, -.75) to[R, -*, l_=20<\kohm>] (v1)
      (v2) to[short, -o] ++(-0.5, 0) node[left]{$+\SI{4.3}{\V}$}
      (v1) to[short, -o] ++(.5, 0) node[right]{$+\SI{2}{\V}$}
        ;
      \draw[->, default] (h1.E) -- ++(0, 2) node[above]{$+\SI{5}{\V}$};
        
    \end{circuitikz}
  \caption{}
  \label{fig:5.29b}
  \end{subfigure}
  \begin{subfigure}{0.32\textwidth}
    \centering
    \begin{circuitikz}[scale=0.8, transform shape, >=triangle 45]
      \draw[default] 
      (0, 0) node[ground]{} to[R, l_=1<\kohm>] ++(0, 3) coordinate(v1) 
      node [pnp, anchor=C] (h1) {}
      (h1.B) to[short, -*] ++ (-1.5, 0) coordinate(v2) -- ++(0, -.75) to[R, -*, l_=100<\kohm>] (v1)
      (v2) to[short, -o] ++(-0.5, 0) node[left]{$+\SI{6.3}{\V}$}
      (h1.E) to[R, l=1<\kohm>] ++ (0, 2.5) coordinate(v3)
      (h1.E) to[short, -o] ++(.5, 0) node[right]{$+\SI{7}{\V}$}
        ;
        \draw[->, default] (v3) -- ++(0, .5) node[above]{$+\SI{10}{\V}$};
        
    \end{circuitikz}
  \caption{}
  \label{fig:5.29c}
  \end{subfigure}
  \caption{}
  \label{fig:5.29}
\end{figure}



% 25:34
\section{4.25}
Design the circuit in Figure~\ref{fig:4.25} to establish $I_C = \SI{0.1}{\mA}$ and $V_C = \SI{0.5}{\V}$. The transistor exhibits $v_{BE}$ of $\SI{0.7}{V}$ at $i_C = \SI{1}{\mA} \text{, and } \beta = 30$.


\begin{figure}[H]
  \centering
  \begin{circuitikz}[scale=0.8, transform shape, >=triangle 45]
    \draw[default] 
    (0, 0) to[R, l_=$R_E$] ++(0, 3) coordinate(v1) 
    node [npn, anchor=E] (h1) {}
    (h1.B) -- ++(-1, 0) node[ground]{}
    
    (h1.C) to[R, l_=$R_C$, i<^=$I_C$] ++ (0, 2.5) coordinate(v3)
    (h1.C) to[short, -o] ++(.5, 0) node[right]{$V_C$}
      ;
    \draw[->, default] 
    (v3) -- ++(0, .5) node[above]{$+\SI{1.5}{\V}$};
    \draw[->, default] 
    (0, 0) -- ++(0, -.5) node[below]{$-\SI{1.5}{\V}$};
      
  \end{circuitikz}
  \caption{}
  \label{fig:4.25}
\end{figure}

\Ans \\
\begin{center}
  \begin{tabular}{|l|c|c|c|c|c|c|c|}
    \hline
    & \multicolumn{6}{c|}{Voltage \si\V} & \\ \cline{2-7}
    Case & $V_S$ & $V_G$ & $V_D$ & $V_{GS}$ & $V_{OV}$ & $V_{DS}$ & Region of operation \\
    \hline
    a & $+1.0$ & $+1.0$ & $+2.0$ & $0$ & $0$ & $1.0$ & Cut-off \\
    b & $+1.0$ & $+2.5$ & $+2.0$ & $1.5$ & $0.5$ & $1.0$ & Saturation \\
    c & $+1.0$ & $+2.5$ & $+1.5$ & $1.5$ & $0.5$ & $0.5$ & Boundary of Sat./Tri. \\
    d & $+1.0$ & $+1.5$ & $0$ & $0.5$ & $0$ & $-1.0$ & Cut-off \\
    e & $0$ & $+2.5$ & $+1.0$ & $2.5$ & $1.5$ & $1$ & Triode \\
    f & $+1.0$ & $+1.0$ & $+1.0$ & $0$ & $0$ & $0$ & Cut-off \\
    g & $-1.0$ & $0$ & $0$ & $1.0$ & $0$ & $1.0$ & Boundary of Cut./Sat. \\
    h & $-1.5$ & $0$ & $0$ & $1.5$ & $0.5$ & $1.5$ & Saturation \\
    i & $-1.0$ & $0$ & $+1.0$ & $1.0$ & $0$ & $2.0$ & Boundary of Cut./Sat. \\
    j & $+0.5$ & $+2.0$ & $+0.5$ & $1.5$ & $0.5$ & $0$ & Triode \\
    \hline
  \end{tabular}
\end{center}

% 45:61
\section{4.45}
For the circuit in Figure~\ref{fig:}, find $V_B, V_E, \text{ and } V_C$ for $R_B = \SI{100}{\kohm}, \SI{10}{\kohm}, \text{ and } \SI{1}{\kohm}$. Let $\beta = 100$.

\begin{figure}[H]
  \centering
  \begin{circuitikz}[scale=0.8, transform shape, >=triangle 45]
    \draw[default] 
    (0, 0) node[ground]{} to[R, l_=1<\kohm>] ++(0, 3) coordinate(v1) 
    node [npn, anchor=E] (h1) {}
    (h1.B) -- ++(-1, 0) to[R, l=$R_B$] ++ (0, 2.5) coordinate(v2)
    (h1.C) to[R, l_=1<\kohm>] ++ (0, 2.5) coordinate(v3)
    (h1.C) node[circ]{} ++ (.5, 0) node{\color{red}$V_C$}
    (h1.E) node[circ]{} ++ (.5, 0) node{\color{red}$V_E$}
    (h1.B) node[circ]{} ++ (-.5, -.5) node{\color{red}$V_B$}
      ;
    \draw[->, default] 
    (v3) -- ++(0, .5) ;
    \draw[->, default] 
    (v2) -- ++(0, 1.25);
    \node at (-1, 8) {$+\SI{5}{\V}$};
      
  \end{circuitikz}
  \caption{}
  \label{fig:4.25}
\end{figure}


\Ans \\
For (a), $V_1 = \SI{3}{\V}$. For (b), $V_2 = \SI{-2}{\V}$. For (c), $V_3 = \SI{3}{\V}$. For (d), $V_4 = \SI{2}{\V}$.  \\
Notice that $V_{GD} = 0$ in each situation, so in order to remain saturation, $V_{GD} \geq V_t = \SI{1}{\V}$ in NMOS and $V_{GD} \leq \SI{-1}{\V}$ in PMOS, so $R_{\text{max}} = \abs{V_t} / I_D = \SI{0.5}{\kohm}$. \\
Finally, notice that $\abs{V_{CC} - V_{S}} = \SI{10}{\V}$ in each configuration \footnote{the problem is designed quite user-friendly..}, so $\abs{V_{DS} + V_R} = \abs{V_{CC} - V_{S}} - 2 = \SI{8}{\V}$, and since $\abs{V_{DS}} = \SI{3}{\V}$, $V_R = \SI{5}{\V}$, $R = V_R / I_D = \SI{2.5}{\kohm}$.

% 47:63
\section{4.47}
For the circuits in Figure~\ref{fig:4.47}, find values for the labeled node voltages and branch currents. Assume $\beta = 100$.

\begin{figure}[H]
  \centering
  \begin{subfigure}{0.32\textwidth}
    \centering
    \begin{circuitikz}[scale=0.8, transform shape, >=triangle 45]
      \draw[default] 
      (0, 3) node[npn, anchor=E] (h1) {} to[I, l^=2<\mA>] (0, 0) 
      (h1.B) -- ++ (-0.5, 0) coordinate(v2) to[R, *-, l_=22<\kohm>] ++(0, -3) node[ground]{}
      (h1.C) to[R, l=1.6<\kohm>] ++(0, 2.5) coordinate(v3)
      (h1.E) to[short, *-o] ++(1, 0) node[right]{\red $V_1$}
      (h1.C) to[short, *-o] ++(1, 0) node[right]{\red $V_2$}
        ;
      \draw[->, default] (v3) -- ++(0, .5) node[above]{$+\SI{5}{\V}$};
      \draw[->, default] (0, 0) -- ++(0, -.5) node[below]{};
        
    \end{circuitikz}
  \caption{}
  \label{fig:5.29a}
  \end{subfigure}
  \begin{subfigure}{0.32\textwidth}
    \centering
    \begin{circuitikz}[scale=0.8, transform shape, >=triangle 45]
      \draw[default] 
      (0, 3) node[npn, anchor=E] (h1) {} to[R, l^=2.2<\kohm>, i>_={\color{red}$I_4$}] (0, 0) 
      (h1.B) -- ++ (-1, 0) node[ground]{}
      (h1.C) to[R, l=1.6<\kohm>] ++(0, 2.5) coordinate(v3)
      (h1.C) to[short, *-o] ++(1, 0) node[right]{\red $V_3$}
        ;
      \draw[->, default] (v3) -- ++(0, .5) node[above]{$+\SI{5}{\V}$};
      \draw[->, default] (0, 0) -- ++(0, -.5) node[below]{$\SI{-5}{\V}$};
        
    \end{circuitikz}
    \caption{}
    \label{fig:5.29b}
  \end{subfigure}
  \begin{subfigure}{0.32\textwidth}
    \centering
    \begin{circuitikz}[scale=0.8, transform shape, >=triangle 45]
      \draw[default] 
      (0, 3) node[npn, anchor=E] (h1) {} to[R, l^=2.2<\kohm>] (0, 0) 
      (h1.B) -- ++ (-0.5, 0) coordinate(v2) to[R, *-, l_=22<\kohm>] ++(0, -3) node[ground]{}
      (h1.C) to[R, l=1.6<\kohm>] ++(0, 2.5) coordinate(v3)
      (h1.E) to[short, *-o] ++(1, 0) node[right]{\red $V_5$}
      (h1.C) to[short, *-o] ++(1, 0) node[right]{\red $V_7$}
      (v2) to[short, -o] ++(-.5, 0) node[left]{\red $V_6$}
        ;
      \draw[->, default] (v3) -- ++(0, .5) node[above]{$+\SI{5}{\V}$};
      \draw[->, default] (0, 0) -- ++(0, -.5) node[below]{$-\SI{5}{\V}$};
    \end{circuitikz}
  \caption{}
  \label{fig:5.29c}
  \end{subfigure}
  \begin{subfigure}{0.4\textwidth}
    \centering
    \begin{circuitikz}[scale=0.8, transform shape, >=triangle 45]
      \draw[default] 
      (0, 3) node[npn, anchor=E] (h1) {} to[R, l^=5.1<\kohm>] (0, 0) 
      (h1.B) -- ++ (-1.5, 0) coordinate(v2) to[R, *-, l^=56<\kohm>] ++(0, 3) coordinate(v4)
      (h1.C) to[R, l_=3.3<\kohm>] ++(0, 2.5) coordinate(v3)
      (h1.E) to[short, *-o] ++(1, 0) node[right]{\red $V_9$}
      (h1.C) to[short, *-o] ++(1, 0) node[right]{\red $V_8$}
        ;
      \draw[->, default] (v3) -- ++(0, .5) node[above]{$+\SI{5}{\V}$};
      \draw[->, default] (0, 0) -- ++(0, -.5) node[below]{$-\SI{5}{\V}$};
      \draw[->, default] (v4) -- ++(0, .5) node[above]{$+\SI{1.2}{\V}$};
    \end{circuitikz}
  \caption{}
  \label{fig:5.29c}
  \end{subfigure}
  \begin{subfigure}{0.4\textwidth}
    \centering
    \begin{circuitikz}[scale=0.8, transform shape, >=triangle 45]
      \draw[default] 
      (0, 3) node[npn, anchor=E] (h1) {} to[R, l^=5.1<\kohm>] (0, 0) 
      (h1.B) -- ++ (-1.5, 0) coordinate(v2) to[R, *-, l^=91<\kohm>] ++(0, 3.25) coordinate(v4)
      (v2) to[R, l_=150<\kohm>] ++(0, -3.75) coordinate(v5)
      (h1.C) to[R, l_=3.3<\kohm>] ++(0, 2.5) coordinate(v3)
      (h1.E) to[short, *-o] ++(1, 0) node[right]{\red $V_{12}$}
      (h1.C) to[short, *-o] ++(1, 0) node[right]{\red $V_{11}$}
      (v2) to[short, -o] ++(-.5, 0) node[left]{\red $V_{10}$}
        ;
      \draw[->, default] (v3) -- ++(0, .5) node[above, xshift=-1cm]{$+\SI{5}{\V}$};
      \draw[->, default] (0, 0) -- ++(0, -.5) node[below, xshift=-1cm]{$-\SI{5}{\V}$};
      \draw[->, default] (v4) -- ++(0, .5);
      \draw[->, default] (v5) -- ++(0, -.5);
    \end{circuitikz}
  \caption{}
  \label{fig:5.29c}
  \end{subfigure}
  \caption{}
  \label{fig:5.29}
\end{figure}

\Ans \\
Notice that every MOS is working on saturation region, since $v_D = v_G$,  hence $v_{DS} > v_{GS} - V_t$. Now for $Q_1, Q_3$, $V_{OV} = v_{GS} - V_t = \SI{0.5}{\V} $ and for $Q_2$, $V_{OV} = \SI{1}{\V} $.
Hence by solving 
\[
  \frac{1}{2} \mu_n C_{ox} \frac{W}{L} V_{OV}^2 = I_D = \SI{120}{\uA} 
\]
We get $W_1 = W_3 = \SI{8}{\um} , W_2 = \SI{2}{\um} $

% 51:68
\section{4.51}
Using $\beta = \infty$, design the circuit shown in Figure~\ref{fig:} so that the bias currents in $Q_1, Q_2, \text{ and } Q_3$ are $\SI{2}{\mA}, \SI{2}{\mA}, \text{ and } \SI{4}{\mA}$, respectively, and $V_3 = 0, V_5 = \SI{-4}{\V}, \text{ and } V_7 = \SI{2}{\V}$. For each resistor, select the nearest standard value utilizing the table of standard values for $5\%$ resistors in Appendix G. Now, for $\beta = 100$, find the values of $V_3, V_4, V_5, V_6, \text{ and } V_7$.
\begin{figure}[H]
  \centering
  \begin{circuitikz}[>=triangle 45]
    \draw[default] 
    (0, 0) to[R, l=$R_1$, -*] ++(0, 2.5) node[npn, anchor=E] (q1){} to[short, -o] ++(.5, 0) node[below right]{\red $V_2$}
    (q1) node[xshift=.3cm] {$Q_1$}
    (q1.B) -- ++(-1, 0) node[ground]{}
    (q1.C) to[R=$R_2$] ++(0, 2.5) coordinate(o1)
    (q1.C) to[short, -*] ++(1, 0) node[pnp, anchor=B](q2){} to[short, -o] ++(0, -.5) node[below]{\red $V_3$}
    (q2) node[xshift=.3cm] {$Q_2$}
    (q2.E) to[R=$R_3$] ++(0, 2) coordinate(o3)
    (q2.E) to[short, *-o] ++(.5, 0) node[right]{\red $V_4$}
    (q2.C) to[R=$R_4$, *-] ++(0, -2.5) coordinate(o2)
    (q2.C) to[short] ++(1, 0) node[npn, anchor=B](q3){} to[short, *-o] ++(0, -.5) node[below]{\red $V_5$}
    (q3) node[xshift=.3cm] {$Q_3$}
    (q3.E) to[R=$R_6$] ++(0, -2.5) coordinate(o4)
    (q3.E) to[short, *-o] ++(.5, 0) node[right]{\red $V_6$}
    (q3.C) to[R=$R_5$] ++(0, 2.5) coordinate(o5)
    (q3.C) to[short, *-o] ++(.5, 0) node[right]{\red $V_7$}
    ; 
    \draw[default, ->] (0, 0) -- ++(0, -.5);
    \draw[default, ->] (o1) -- ++(0, .5);
    \draw[default, ->] (o2) -- ++(0, -1.25) node[below] {$\SI{-10}{\V}$};
    \draw[default, ->] (o3) -- ++(0, .25) node[above] {$+\SI{10}{\V}$};
    \draw[default, ->] (o4) -- ++(0, -.5);
    \draw[default, ->] (o5) -- ++(0, .5);
    
  \end{circuitikz}
  \caption{}
  \label{fig:5.41}
\end{figure}


\Ans \\
Notice that if we connect the blue dashed line in the figure, the circuit is now symmetric. Moreover, all the MOS are working in saturation region.
Now $Q_1, Q_2, Q_3, Q_4$ has the same $k_n$, and the current go through $Q_4, Q_2$ is equal to the current that go through $Q_3, Q_1$. So by symmetry, $V_2 = V_1 =  \SI{5}{V} / 2 = \SI{2.5}{\V}$. Notice that there is no currenct go through the blue dashed line, so if we remove the line, the voltage remains unchanged. Hence we know that in the origin circuit, $V_2 = \SI{2.5}{\V}$.
\begin{align*}
  k_n &= \mu_n C_{ox} (W / L) = \SI{0.5}{\mA} \\
  V_2 &= v_{GS} = \SI{2.5}{\V} \\
  I_2 &= \frac{1}{2} k_n (v_{GS} - V_t) ^2 = \SI{562.5}{\uA}
\end{align*}
Simmilarly, if now $W_{3,4} = \SI{100}{\um}$, $k_n$ is $10$ times greater than before, so $I_2 = \SI{5.625}{\mA}$.

% 75:95  7.5\kohm <- 8.2\kohm
\section{4.75}
The transistor amplifier in Figure~\ref{fig:} is biased with a current source $I$ and has a very high $\beta$. find the dc voltage at the collector, $V_C$. Also, find the value of $g_m$. Replace the transistor with the simplified hybrid-$\pi$ model of Figure~\ref{fig:} (note that the dc current source $I$ should be replaced with an open circuit). Hence find the voltage gain $v_c/v_i$. 

\begin{figure}[H]
  \centering
  \begin{circuitikz}[transform shape, >=triangle 45]
    \draw[default] 
    (0, 2.5) to[I=$\SI{0.5}{\mA}$] (0, 0)
    (0, 2.5) node [npn, anchor=E, xscale=-1] (h1) {} to[C=$\infty$, *-o] ++(-2.5, 0) coordinate(v1)  ++(0, -2) node[ground]{} to[V, l^=$v_i$] (v1)  
    (h1.B) -- ++(1, 0) node[ground]{}
    
    (h1.C) to[R, l_=$\SI{7.5}{\kohm}$] ++ (0, 2.5) coordinate(v3)
    (h1.C) to[short, -o] ++(.5, 0) node[right]{\red $V_C + v_c$}
      ;
    \draw[->, default] 
    (v3) -- ++(0, .5) node[above]{$+\SI{5}{\V}$};
    \draw[->, default] 
    (0, 0) -- ++(0, -.5) ;
      
  \end{circuitikz}
  \caption{}
  \label{fig:4.25}
\end{figure}


\begin{enumerate}[(a)]
  \item \Ans \\
    At point $A$, $V_{GS} = V_t = \SI{1}{\V}$, $V_{DS} = V_{DD} = \SI{5}{\V}$. At point $B$, 
    \[ V_{OV} = V_{DD} - R_D \frac{1}{2} k_n V_{OV}^2 \quad \Rightarrow
    \quad V_{OV} \approx \SI{0.605}{\V} \]
    So $V_{GS} \approx \SI{1.605}{\V}, V_{DS} = V_{GS} - V_t \approx \SI{0.605}{\V}$.
  \item \Ans \\
    If $V_{OV} = \SI{0.5}{\V}$, $I_D = \frac{1}{2} k_n V_{OV}^2 = \SI{0.125}{\mA}$.$V_{GS} = V_t + V_{OV} = \SI{1.5}{\V}, \; V_{DS} = V_{CC} - R_D I_D = \SI{2}{\V}$.
    \[
    A_{v} = -\frac{2 I_D R_D}{V_{OV}} = -\frac{2 (\SI{3}{\V}) }{\SI{0.5}{\V}} =  -\SI{12}{\V\per\V} \]
  \item \Ans \\
    By (a), $V_{GS,A} = \SI{1}{\V}, V_{GS, B} \approx \SI{1.605}{\V}$, so the maximum amplitute $\hat{V}_i = \abs{V_{GS, B} - V_{GS, Q}} \approx \SI{0.105}{\V}$.
    The amplitute of output is $\hat{V}_o \abs{V_{DS, Q} - V_{DS, B}} \approx \SI{1.395}{\V}$. The Gain $G = \hat{V}_o / \hat{V}_i \approx 13.28$. The error $\epsilon = (13.28 - 12)/13.28  \approx 9.6\%$. \\
    The different appears because the curve is quadratic but not linear.
\end{enumerate}

% 80:101
\section{4.80}
In the circuit shown in Figure~\ref{fig:}, the transistor has a $\beta$ of 200. What is the dc voltage at the collector? Find the input resistances $R_{ib}$ and $R_{in}$ and the overall voltage gain $(v_o/v_sig)$. For an output signal of $\pm \SI{0.4}{\V}$, what values of $v_{sig}$ and $v_b$ are required?

\begin{figure}[H]
  \centering
  \begin{circuitikz}[transform shape, >=triangle 45]
    \draw[default] 
    (0, 0) to[I=$\SI{10}{\mA}$] ++(0, -2.5) node[pnp, anchor=E] (q1){} to[C, l_=$\infty$, *-] ++(2.5, 0) node[ground]{}
    (q1.C) to[R=$R_C{=}\SI{100}{\ohm}$] ++(0, -2.5) node[ground]{}
    (q1.C) to[short, *-o] ++(1, 0) node[right]{\red $v_o$}
    (q1.B) to[short, -o] ++(-.5, 0) node[above]{\red $v_b$} to[short, -*]  ++(-.5, 0) coordinate(v2) to[C, l_=$\infty$, -o] ++(-2.5, 0) to[R, l_=$R_{sig}{=}\SI{1}{\kohm}$] ++(-2.5, 0) ++(0, -2.5) node[ground]{} to[V=$v_{sig}$] ++(0, 2.5) 
    (v2) to[R, l^=$\SI{10}{\kohm}$] ++(0, 2.5) coordinate(o1)
      ;
    \draw[->, default] 
    (0, 0) -- ++(0, .5) node[above]{$+\SI{5}{\V}$};
    \draw[->, default] 
    (o1) -- ++(0, .5) node[above]{$+\SI{5}{\V}$};
    \draw[red, ->]
    (-2, -7) node[below]{$R_{ib}$}  |-  ++(1, 3);
    \draw[red, ->, rounded corner]
    (-4, -7) node[below]{$R_{in}$} |- ++(1, 3);
      
  \end{circuitikz}
  \caption{}
  \label{fig:4.25}
\end{figure}

\Ans
Since both the MOS is operating in saturation, we have
\begin{align*}
  i_{D1} &= \frac{1}{2} k'_n \frac{W_1}{L_1} (v_I - V_t)^2 \\
  i_{D2} &= \frac{1}{2} k'_n \frac{W_2}{L_2} (V_{DD} - v_O - V_t)^2 
\end{align*}

By $i_{D1} = i_{D2}$ we obtain
\begin{gather*}
  \frac{W_1}{L_1} (v_I - V_t)^2 = \frac{W_2}{L_2} (V_{DD} - v_O - V_t)^2 \\
  \Rightarrow v_O = V_{DD} - V_t + \sqrt{ \frac{W_1/L_1}{W_2/L_2} } (V_t - v_I) 
\end{gather*}
Which is the desired result.

% 109:138
\section{4.109}
The circuit in Figure~\ref{fig:} provides a constant current $I_O$ as long as the circuit to which the collector is connected maintains the BJT in the active mode. Show that
\[
  I_O = \alpha \frac{V_CC (R_2 / (R_1 + R_2)) - V_{BE}}{R_E + (R_1 \paral R_2) / (\beta + 1)} 
\]
\begin{figure}[H]
  \centering
  \begin{circuitikz}[transform shape, >=triangle 45]
    \draw[default] 
    (0, 0) node[ground]{} to[R=$R_2$] ++(0, 3.25) coordinate(v1) to[R=$R_1$] ++(0, 2.5) coordinate(o1)
    (v1) to[short, *-] ++(2, 0) node[npn, anchor=B] (q1) {}
    (q1.E) to[R=$R_E$] ++(0, -2.5) node[ground]{}
    (q1.C) to[short, i<_=$I_O$] ++(0, .5) coordinate(o2)
      ;
    \draw[dashed, default] 
    (o2) -- ++(0, 1);
    \draw[->, default] 
    (o1) -- ++(0, .5) node[above]{$V_{CC}$};
      
  \end{circuitikz}
\caption{}
\label{fig:}
\end{figure}


\Ans\\
\begin{align*}
  g_m &= \frac{2I_D}{V_{OV}}\\
  A_v &= -g_m R_D = -\frac{2 I_D R_D}{V_{OV}}  = -\frac{2 (V_{DD} - V_D)}{V_{OV}} 
\end{align*}
Let $v_g = V_G + v_i$ be the total gate voltage, $v_d = V_D + A_v v_i$ be the total drain voltage, the saturation condition required that $v_d \geq v_g - V_t$. so
\begin{align*}
  & v_d \geq v_g - V_t \\
  \Rightarrow & V_D + A_v v_i \geq V_G + v_i - V_t  \\
  \Rightarrow & V_D - \frac{2 (V_{DD} - V_D)}{V_{OV}} v_i \geq V_G + v_i - V_t \\
  \Rightarrow  & V_D - \frac{2 (V_{DD} - V_D)}{V_{OV}} \hat{v_i} \geq V_G + \hat{v_i} - V_t \quad \footnotemark \\
  \Rightarrow & \left(1 + \frac{2\hat{v_i}}{V_{OV}}\right) V_D \geq V_G + \hat{v_i} + \frac{2V_{DD}}{V_{OV}} \hat{v_i}  - V_t  \\
  \Rightarrow & V_D \geq \frac{V_{OV} + \hat{v_i} + 2 V_{DD} (\hat{v_i} / V_{OV}) }{1 + 2(\hat{v_i}/V_{OV})} 
\end{align*}
\footnotetext{Since $A_v, v_i$ and $v_i$ are in opposite sign, so in the worse case, we shall consider $-A_v \hat{v_i}$ and $\hat{v_i}$}
Which is the desired result.
Plug in the value and let $V_D$ be the maximum possible value we found out that
\begin{alignat*}{3}
  & V_{OV} &&= m \hat{v_i} && = \SI{200}{\mV} \\
  & V_{D} &&= \frac{V_{OV} + \hat{v_i} + 2 V_{DD} (\hat{v_i} / V_{OV}) }{1 + 2(\hat{v_i}/V_{OV})} && \approx \SI{683}{\mV} \\
  & A_v &&= -\frac{2(V_{DD} - V_D)}{V_{OV}}  && \approx -23.2 \\
  & \hat{v_o} &&= \abs{A_v} \hat{v_i} && \approx \SI{464}\mV \\
\end{alignat*}
If moreover, $I_D = \SI{100}{\uA}$,
\begin{alignat*}{3}
  & R_D &&= \frac{V_{DD} - V_D}{I_D}  && \approx \SI{23.17}{\kohm} \\
  & \frac{W}{L} &&= I_D / \left(\frac{1}{2} k'_n V_{OV}^2 \right) && = 50
\end{alignat*}

% 110:139
\section{4.110}
The current-source biasing circuit shown in Figure~\ref{fig:} provides a bias current to $Q_1$ that is determined by the current source formed by $Q_2, R_1, R_2, \text{ and } R_E$. The bias current is independent of $R_B$ and nearly independent of $\beta_1$ (as long as both $Q_1$ and $Q_2$ operate in the active mode).
It is required to design the circuit using $\pm 5 \si{\V}$ dc supplies to establish $I_{C1} = \SI{0.1}{\mA}$ and $V_{CE1} = \SI{1.5}{\V}$, in the ideal situation of infinite $\beta_1$ and $\beta_2$. In designing the current source, use $\SI{2}{\V}$ dc voltage drop across $R_E$ and impose the requirement that $I_{E2}$ remain within $5\%$ of its ideal value of $\beta_2$ as low as $50$. 
In selecting a value for $R_B$, ensure that for the lowest value of $\beta_1 = 50$, $V_{CE2}$ is $\SI{2.5}{\V}$. Use standard $5 \%$ resistor values (see Appendix H). What values for $R_1, R_2, R_E, R_B, R_C$ do you choose? What values of $I_{C1} \text{ and } I_{CE1}$ result for $\beta_1 = \beta_2 = 50, 100, \text{ and } , 200$?

\begin{figure}[H]
\begin{center}
  \begin{circuitikz}[>=triangle 45, scale=1, transform shape]
    \draw[default]
    (0, 0) to[R=$R_E$] ++(0, 2.5) node[npn, anchor=E] (q2) {}
    ($(q2.E) !.5! (q2.C)$) node[right] {$Q_2$}
    (q2.B) to[short, -*] ++ (-1, 0) coordinate(v1) to[R=$R_1$, -*] ++(0, 2.5) 
    coordinate(v2) to[R=$R_B$] ++(0, 2) to[short, *-] ++(1, 0) node[npn, anchor=B](q1){}
    ($(q1.E) !.5! (q1.C)$) node[right] {$Q_2$}
    (v1) to[R=$R_2$] ++(0, -2.5) coordinate(o1)
    (v2) -- ++(-.5, 0) node[ground]{}
    (q1.E) -- (q2.C)
    (q1.C) to[R=$R_C$] ++(0, 2.5) coordinate(o2)
    ;
    \draw[default, ->] (0, 0) -- ++(0, -.5);
    \draw[default, ->] (o1) -- ++(0, -1.25);
    \draw[default, ->] (o2) -- ++(0, .5);
    \node[below] at (-1, -.5) {$-V_{EE}$};

  \end{circuitikz}
\end{center}
\caption{}
\label{fig:5.56}
\end{figure}

\begin{enumerate}[(a)]
  \item \Ans \\
    Simply check that 
    \begin{alignat*}{3}
      V_G &= \frac{5}{10+5} \cdot \SI{15}{\V} &&= \SI{5}{\V} \\
      V_S &= (\SI{3}{\kohm}) I_D &&= \SI{3}{\V} \\
      V_{GS} &= V_{G} - V_{S} &&= \SI{2}{\V} \quad \text{(check)}\\
      V_D &= 15 - \SI{7.5}{\kohm} \cdot I_D &&= \SI{7.5}{\V} \\
      V_D & \geq V_G - V_t, V_G \geq V_S + V_t && \quad\text{(check in saturation)} \\
      k_n &= \frac{2 I_D}{(V_{GS} - V_t)^2} &&= \SI{2}{\mA\per\V\squared}
      \quad \text{(check)}
    \end{alignat*}
  \item \Ans \\
    \begin{alignat*}{3}
      g_m &= \frac{2 I_D}{V_{OV}} = \frac{2 I_D}{V_{GS} - V_t} && = \SI{2}{\mA\per\V} \\
      r_o &= \frac{V_A}{I_D} &&= \SI{100}{\kohm}
    \end{alignat*}
  \item 
    \begin{figure}[H]
    \begin{center}
      \begin{circuitikz}[>=triangle 45, scale=1, transform shape]
        \draw[default]
        (0, 0) node[ground]{} to[V, l=$v_{sig}$] ++(0, 3) to[R, l=100<\kohm>, -o]  ++(3, 0) coordinate(v1) to[R, l_=5<\mega\ohm>] ++(0, -3) node[ground]{}
        (v1) to[short] ++(1, 0) coordinate(v2) to[R, l=10<\mega\ohm>]  ++(0, -3) node[ground]{}
        (v2) to[short] ++(2.5, 0) coordinate(v3) to [R, l=$1/g_m$, i=$I$] ++(0, -3) node[ground]{} 
        (v3)  ++(0, 3) coordinate(uu) to[short] ++(1, 0) coordinate(v4) to[R, l=$r_o$] ++(0, -3) node[ground]{} 
        (v4) to[short] ++(1.5, 0) coordinate(v5) to[R, l=5<\kohm>] ++(0, -3) node[ground]{}
        (v5) to[short] ++(2, 0) to[R, l=7.5<\kohm>] ++(0, -3) node[ground]{}
        (uu) to[cI, l_=$I$] (v3)
        ;

      \end{circuitikz}
    \end{center}
    \caption{}
    \label{fig:}
    \end{figure}
  \item \Ans \\
    \begin{alignat*}{3}
      R_{in} &= (\SI{10}{\mega\ohm}) \paral (\SI{5}{\mega\ohm}) &&\approx \SI{3.33}{\mega\ohm} \\
      \frac{v_{gs}}{v_{sig}}  &= \frac{R_{in}}{R_{in} + (\SI{3.33}{\mega\ohm})} && \approx 0.97 \\
      \frac{v_{o}}{v_{gs}}  &= g_m (\SI{7.5}{\kohm} \paral \SI{10}{\kohm} \paral r_o) && \approx 8.22 \\
      \frac{v_{o}}{v_{sig}} &= \frac{v_{gs}}{v_{sig}} \frac{v_{o}}{v_{gs}} \approx 7.97
    \end{alignat*}

\end{enumerate}

Another solution \\
\begin{enumerate}[(a)]
  \item \Ans \\$I_D = \SI{1}{\mA} \Rightarrow V_S = \SI{3}{\V}$, and $V_G = \SI{15}{\V}
    \cdot 5/(5+10) = \SI{5}{\V}$. So $V_{GS} = V_G - V_S = \SI{2}{\V}$ and
    $V_D = \SI{15}{\V} - \SI{7.5}{\kohm} \cdot \SI{1}{\mA} = \SI{7.5}{\V}$
    which is the desired result.
  \item \Ans \\
    $V_{OV} = V_{GS} - V_t = \SI{1}{\V}$, so $g_m = k_n V_{OV} = \SI{2}{\mA/\V}$.
    $r_o = V_A / I_D = \SI{100}{\kohm}$.
  \item \Ans \\
    For small-signal we can remove all large capacitor, change the two upper
    terminals to ground, and use T model to substitute the transistor.
    (Don't forget $r_o$).
  \item \Ans \\
    Since there is no current flowing into the gate of transistor, 
    $R_{in} = \SI{10}{\mega\ohm} \parallel \SI{5}{\mega\ohm} = \SI{3.33}{\mega\ohm}$.
    And for small signal, $v_d = 0$, so $v_{gs} = v_g = v_{sig} \cdot
    R_{in} / (R_{in} + \SI{100}{\kohm}) \Rightarrow v_{gs}/v_{sig} = 0.97$.
    Notice that $r_o$ connect drain to ground, so we have $v_o/v_{gs} =
    -g_m (r_o \parallel \SI{7.5}{\kohm} \parallel \SI{10}{\kohm}) = 8.22$.
    $v_o / v_{sig} = 8.22 \cdot 0.97 = 7.97$.
\end{enumerate}

% 117:147
\section{4.117}
The amplifier of Figure~\ref{fig:} consists of two identical common-emitter amplifiers connected in cascade. Observe that the input resistance of the second stage, $R_{in2}$, constitutes the load resistance of the first stage.

\begin{enumerate}[(a)]
  \item For $V_{CC} = \SI{15}{\V}, R_1 = \SI{100}{\kohm}, R_2 = \SI{47}{\kohm}, R_E = \SI{3.9}{\kohm}, R_C = \SI{6.8}{\kohm}, \text{ and } \beta = 100$, determine the dc collector current and dc collector voltage of each transistor.
  \item Draw the small-signal equivalent circuit of the entire amplifier and give the values of all its components. Neglect $r_{o1} \text{ and } r_{o2}$.
  \item Find $R_{in1}$ and $v_{b1}/v_{sig}$ for $R_{sig} = \SI{5}{\kohm}$.
  \item Find $R_{in2}$ and $v_{b2}/v_{b1}$.
  \item For $R_L = \SI{2}{\kohm}$, find $v_o / v_{b2}$.
  \item Find the overall voltage gain $v_o / v_{sig}$.
\end{enumerate}
\begin{figure}[H]
  \centering
  \begin{circuitikz}[>=triangle 45, scale=.8, transform shape]
    \draw[default]
    (0, 0) node[ground]{} to[V=$v_{sig}$] ++(0, 3) to[R=$R_{sig}$] ++(2.5, 0) to[C=$\infty$, o-*] ++(2.5, 0) coordinate(v1) to[short] ++(.5, 0) node[npn, anchor=B](q1){}
    (v1) to[R, l^=$R_1$] ++(0, 3) coordinate(o1)
    (v1) to[R, l_=$R_2$] ++(0, -3) node[ground]{}
    ($(q1.C) !.5! (q1.E)$) node[right]{$Q_1$}
    (q1.E) to[R, l_=$R_E$, *-] ++(0, -2.25) node[ground]{}
    (q1.E) -- ++(1, 0) to[C=$\infty$] ++(0, -2.25) node[ground]{}
    (q1.C) to[R, l_=$R_C$] ++(0, 2.25) coordinate(o2)
    (q1.C) to[C=$\infty$] ++(4, 0) to[short] ++(0, -.75) coordinate(v2) to[short, *-o] ++(0, -.5) node[below]{$v_{b2}$}
    (v2) to[short, -*] ++(.5, 0) coordinate(v3) to[short] ++(.5, 0) node[npn, anchor=B](q2){}
    ($(q2.C) !.5! (q2.E)$) node[right]{$Q_2$}
    (v3) to[R=$R_1$] ++(0, 3) coordinate(o3)
    (v3) to[R, l_=$R_2$] ++(0, -3) node[ground]{}
    (q2.E) to[R, l_=$R_E$, *-] ++(0, -2.25) node[ground]{}
    (q2.E) -- ++(1, 0) to[C=$\infty$] ++(0, -2.25) node[ground]{}
    (q2.C) to[R, l_=$R_C$] ++(0, 2.25) coordinate(o4)
    (q2.C) to[short] ++(1, 0) to[C=$\infty$] ++(3, 0) coordinate(v4) to[R=$R_L$] ++(0, -3.75) node[ground]{}
    (v4) to[short, *-o] ++(.5, 0)
    ($(o1) !.5! (o2)$) node[above]{$V_{CC}$} node[yshift=0.8cm] {Stage 1}
    ($(o3) !.5! (o4)$) node[above]{$V_{CC}$} node[yshift=0.8cm] {Stage 2}
    (.5, 6.8) node{Source}
    (16, 6.8) node{Load}
    ;
    \draw[default, ->] (o1) -- ++(0, .2);
    \draw[default, ->] (o2) -- ++(0, .2);
    \draw[default, ->] (o3) -- ++(0, .2);
    \draw[default, ->] (o4) -- ++(0, .2);
    \draw[dashed] (2.2, -.5) -- (2.2, 6.5);
    \draw[dashed] (9, -.5) -- (9, 6.5);
    \draw[dashed] (14.3, -.5) -- (14.3, 6.5);
    \draw[red, ->] (1.8, -.5) node[below]{$R_{in1}$} |- ++(1, 2.7);
    \draw[red, ->] (8.8, -.5) node[below]{$R_{in2}$} |- ++(1, 2.7);
  \end{circuitikz}
  \caption{}
  \label{fig:5.72}
\end{figure}
\Ans \\
Since $I_D = V_S / R_S = \SI{2}{\V} / \SI{1}{\kohm} = \SI{2}{\mA}$, so
\begin{align*}
  & \frac{1}{2} k'_n(W/L) (V_G - V_S - V_t)^2 = I_D \\
  \Rightarrow & (\SI{3} - V_t)^2 = \SI{2}{} \\
  \Rightarrow & V_t \approx \SI{1.59}{\V} 
\end{align*} 
If now $V'_t = V_t - \SI{0.5}{\V} = \SI{1.09}{\V}$, We have
\begin{align*}
  & \frac{1}{2} (\SI{2}{\mA\per\V}) ((\SI{5}{\V}) - V_S - (\SI{1.09}{\V}))^2 = V_S / \SI{1}{\kohm}  \\
  \Rightarrow & V_S \approx \SI{2.37}{\V} \quad \text{ \footnotemark}\\
  \Rightarrow & I_S = V_S / R_S \approx \SI{2.37}{\mA}
\end{align*}
\footnotetext{There are two solutions, but the other one would let $V_S > V_G$.} 

% 122:152
\section{4.122}
For the emitter-follower circuit shown in Figure~\ref{fig:}, the BJT used  is specified to have $\beta$ values in the range of $40 \text{ to } 200$ (a distressing situation for the circuit designer). For the two extreme values of $\beta$ ($\beta = 400$ and $\beta = 200$), find:

\begin{enumerate}
  \item $I_E, V_E, \text{ and } V_B$.
  \item the input resistance $R_{in}$.
  \item the voltage gain $v_o/v_{sig}$.
\end{enumerate}

\begin{figure}[H]
  \centering
  \begin{circuitikz}[>=triangle 45, transform shape]
    \draw[default]
    (0, 0) node[ground]{} to[V=$v_{sig}$] ++(0, 3) to[R=$\SI{10}{\kohm}$] ++(2.5, 0) to[C=$\infty$, o-*] ++(2.5, 0) coordinate(v1) to[short] ++(.5, 0) node[npn, anchor=B](q1){}
    (v1) to[R, l^=$\SI{100}{\kohm}$] ++(0, 3) coordinate(o1)
    (q1.E) to[R, l^=$\SI{1}{\kohm}$, *-] ++(0, -2.25) node[ground]{}
    (q1.E) to[C=$\infty$] ++(2.5, 0) coordinate(v2) to[R=$\SI{1}{\kohm}$] ++(0, -2.25) node[ground]{}
    (q1.C) -- ++(0, 2.25) coordinate(o2)
    (v2) to[short, *-o] ++(.5, 0) node[right]{\red $v_o$}
    ($(o1) !.5! (o2)$) node[above]{$+\SI{9}{\V}$}
    ;
    \draw[default, ->] (o1) -- ++(0, .2);
    \draw[default, ->] (o2) -- ++(0, .2);
    \draw[red, ->] (2.2, -.5) node[below]{$R_{in}$} |- ++(1, 2.7);
  \end{circuitikz}
  \caption{}
  \label{fig:5.72}
\end{figure}
\begin{enumerate}[(a)]
  \item \Ans \\
    \begin{alignat*}{3}
      V_G &= \frac{200}{300+200} (\SI{5}{\V}) &&= \SI{2}{\V} \\
      V_S &= (\SI{2}{\kohm}) (\SI{0.5}{\mA}) &&= \SI{1}{\V} \\
      V_D &= \SI{5}{\V} - (\SI{5}{\kohm}) (\SI{0.5}{\mA}) &&= \SI{2.5}{\V} \\
    \end{alignat*}
    Since $V_D \geq V_G - V_t, \; V_G \geq V_S + V_t$, the nmos is indeed operate in saturation region.
    \[
      k_n = \frac{2 I_D}{V_{OV}^2} = \SI{11.11}{\mA\per\V\squared}
    \]
  \item \Ans \\
    Calculate $r_o = V_A / I_D = \SI{100}{\kohm}$.
    \[
      R_{in} = \SI{300}{\kohm} \paral \SI{200}{\kohm} = \SI{120}{\kohm} \]
    And
    \begin{align*}
      g_m &= \frac{2 I_D}{V_{OV}} = \SI{3.33}{\mA\per\V} \\
      v_{i} &= \frac{R_{in}}{\SI{120}{\kohm} + R_{in}} v_{sig} = \frac{1}{2} v_{sig}  \\
      G_v &= v_o / v_{sig} = \frac{1}{2} v_o / v_i = -\frac{1}{2} g_m (\SI{5}{\kohm} \paral \SI{5}{\kohm} \paral r_o) \approx \SI{-4.05}{\V\per\V} \\
    \end{align*}
  \item 
    Remains in saturation requires $v_d + V_D \geq v_g + V_G - V_{t}$. Notice that \\ $v_d = v_o = G_v v_{sig}, \; v_g = \frac{1}{2} v_{sig}$, so
    \[
      -4.05 \hat{v}_{sig} + \SI{2.5}{\V} = \frac{1}{2} \hat{v}_{sig} + \SI{2}{\V} - \SI{0.7}{\V} \quad \Rightarrow \quad 
      \hat{v}_{sig} \approx \SI{263}{\mV} \numeq \label{eq:80-1}
    \]
    \[
      \hat{v}_{o} = \hat{v}_{sig} G_v \approx \SI{1.07}{\V}
    \]
  \item 
    By modifing Equation \eqref{eq:80-1} (let $-4.05 \rightarrow G'_v$)
    We have
    \[
     \SI{1.2}{\V} = -G'_v \cdot 2\hat{v}_{sig} + \frac{1}{2} \cdot 2\hat{v}_{sig} \quad \Rightarrow \quad G'_v \approx -1.78
    \]
    if we let $R = R_S \paral \SI{2}{\kohm}$, $R_D = \SI{5}{\kohm} \paral \SI{5}{\kohm}$
    , By using T model we obtain \footnote{a little complicate calculation}
    \[
      2G'_v = -g_m R_D/ (1 + g_mR + (R+R_D)/r_o) \quad \Rightarrow \quad
      R = \SI{393}{\ohm}
    \]
    finally,
    \[
      \frac{\SI{2}{\kohm} \cdot R_S}{\SI{2}{\kohm} + R_S } = \SI{393}{\ohm}
      \quad \Rightarrow \quad R_S \approx \SI{489}{\ohm}
    \]
    \[
      \hat{v}_o = \abs{G'_v} 2\hat{v}_{sig} \approx  \SI{936}{\mV}
    \]
    
\end{enumerate}



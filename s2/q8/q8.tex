\documentclass[12pt, a4paper]{article}
%%%%%%%%%%%%%%%紙張大小設定%%%%%%%%%%%%%%%
% \paperwidth=65cm
% \paperheight=160cm

%%%%%%%%%%%%%%%引入Package%%%%%%%%%%%%%%%
\usepackage[margin=3cm]{geometry} % 上下左右距離邊緣2cm
\usepackage{mathtools,amsthm,amssymb} % 引入 AMS 數學環境
\usepackage{yhmath}      % math symbol
\usepackage{graphicx}    % 圖形插入用
\usepackage{fontspec}    % 加這個就可以設定字體
\usepackage{type1cm}	 % 設定fontsize用
\usepackage{titlesec}   % 設定section等的字體
\usepackage{titling}    % 加強 title 功能
\usepackage{fancyhdr}   % 頁首頁尾
\usepackage{tabularx}   % 加強版 table
\usepackage{multirow}   % colspan
\usepackage[square, comma, numbers, super, sort&compress]{natbib}
% cite加強版
\usepackage[unicode=true, pdfborder={0 0 0}, bookmarksdepth=-1]{hyperref}
% ref加強版
\usepackage[usenames, dvipsnames]{color}  % 可以使用顏色
\usepackage[shortlabels, inline]{enumitem}  % 加強版enumerate
\usepackage{xpatch}

% \usepackage{tabto}      % tab
% \usepackage{soul}       % highlight
% \usepackage{ulem}       % 字加裝飾
\usepackage{wrapfig}     % 文繞圖
%\usepackage{floatflt}    % 浮動 figure
\usepackage{float}       % 浮動環境
\usepackage{caption}    % caption 增強
\usepackage{subcaption}    % subfigures
% \usepackage{setspace}    % 控制空行
% \usepackage{mdframed}   % 可以加文字方框
% \usepackage{multicol}   % 多欄
\usepackage[abbreviations, per-mode=symbol]{siunitx}      % SI unit
% \usepackage{dsfont}     % more mathbb

%%%%%%%%%%%%%%%%%%%TikZ%%%%%%%%%%%%%%%%%%%%%%
% \usepackage{tikz}
\usepackage[siunitx, americanvoltages, americancurrents, arrowmos]{circuitikz}
\usetikzlibrary{calc}

%%%%%%%%%%%%%%中文 Environment%%%%%%%%%%%%%%%
\usepackage[CheckSingle, CJKmath]{xeCJK}  % xelatex 中文
\usepackage{CJKulem}	% 中文字裝飾
\setCJKmainfont[BoldFont=cwTeX Q Hei]{cwTeX Q Ming}
\setCJKsansfont[BoldFont=cwTeX Q Hei]{cwTeX Q Ming}
\setCJKmonofont[BoldFont=cwTeX Q Hei]{cwTeX Q Ming}
% 設定中文為系統上的字型,而英文不去更動,使用原TeX字型

% \XeTeXlinebreaklocale "zh"             %這兩行一定要加,中文才能自動換行
% \XeTeXlinebreakskip = 0pt plus 1pt     %這兩行一定要加,中文才能自動換行

%%%%%%%%%%%%%%%字體大小設定%%%%%%%%%%%%%%%
% \def\normalsize{\fontsize{10}{15}\selectfont}
% \def\large{\fontsize{40}{60}\selectfont}
% \def\Large{\fontsize{50}{75}\selectfont}
% \def\LARGE{\fontsize{90}{20}\selectfont}
% \def\huge{\fontsize{34}{51}\selectfont}
% \def\Huge{\fontsize{38}{57}\selectfont}

%%%%%%%%%%%%%%%Theme Input%%%%%%%%%%%%%%%%
% \input{themes/chapter/neat}
% \input{themes/env/problist}

%%%%%%%%%%%titlesec settings%%%%%%%%%%%%%%
% \titleformat{\chapter}{\bf\Huge}
            % {\arabic{section}}{0em}{}
% \titleformat{\section}{\centering\Large}
            % {\arabic{section}}{0em}{}
% \titleformat{\subsection}{\large}
            % {\arabic{subsection}}{0em}{}
% \titleformat{\subsubsection}{\bf\normalsize}
            % {\arabic{subsubsection}}{0em}{}
% \titleformat{command}[shape]{format}{label}
            % {編號與標題距離}{before}[after]

%%%%%%%%%%%%variable settings%%%%%%%%%%%%%%
% \numberwithin{equation}{section}
% \setcounter{secnumdepth}{4}  %章節標號深度
% \setcounter{tocdepth}{1}  %目錄深度
% \graphicspath{{images/}}  % 搜尋圖片目錄

%%%%%%%%%%%%%%%頁面設定%%%%%%%%%%%%%%%
\newcolumntype{C}[1]{>{\centering\arraybackslash}p{#1}}
\setlength{\headheight}{15pt}  %with titling
\setlength{\droptitle}{-1.5cm} %title 與上緣的間距
% \posttitle{\par\end{center}} % title 與內文的間距
\parindent=24pt %設定縮排的距離
% \parskip=1ex  %設定行距
% \pagestyle{empty}  % empty: 無頁碼
% \pagestyle{fancy}  % fancy: fancyhdr

% use with fancygdr
% \lhead{\leftmark}
% \chead{}
% \rhead{}
% \lfoot{}
% \cfoot{}
% \rfoot{\thepage}
% \renewcommand{\headrulewidth}{0.4pt}
% \renewcommand{\footrulewidth}{0.4pt}

% \fancypagestyle{firststyle}
% {
  % \fancyhf{}
  % \fancyfoot[C]{\footnotesize Page \thepage\ of \pageref{LastPage}}
  % \renewcommand{\headrule}{\rule{\textwidth}{\headrulewidth}}
% }

%%%%%%%%%%%%%%%重定義一些command%%%%%%%%%%%%%%%
\renewcommand{\contentsname}{目錄}  %設定目錄的標題名稱
\renewcommand{\refname}{參考資料}  %設定參考資料的標題名稱
\renewcommand{\abstractname}{\LARGE Abstract} %設定摘要的標題名稱

%%%%%%%%%%%%%%%特殊功能函數符號設定%%%%%%%%%%%%%%%
% \newcommand{\citet}[1]{\textsuperscript{\cite{#1}}}
\DeclarePairedDelimiter{\abs}{\lvert}{\rvert}
\DeclarePairedDelimiter{\norm}{\lVert}{\rVert}
\DeclarePairedDelimiter{\inpd}{\langle}{\rangle} % inner product
\DeclarePairedDelimiter{\ceil}{\lceil}{\rceil}
\DeclarePairedDelimiter{\floor}{\lfloor}{\rfloor}
\DeclareMathOperator{\adj}{adj}
\DeclareMathOperator{\sech}{sech}
\DeclareMathOperator{\csch}{csch}
\DeclareMathOperator{\arcsec}{arcsec}
\DeclareMathOperator{\arccot}{arccot}
\DeclareMathOperator{\arccsc}{arccsc}
\DeclareMathOperator{\arccosh}{arccosh}
\DeclareMathOperator{\arcsinh}{arcsinh}
\DeclareMathOperator{\arctanh}{arctanh}
\DeclareMathOperator{\arcsech}{arcsech}
\DeclareMathOperator{\arccsch}{arccsch}
\DeclareMathOperator{\arccoth}{arccoth}
\newcommand{\np}[1]{\\[{#1}] \indent}
\newcommand{\transpose}[1]{{#1}^\mathrm{T}}
%%%% Geometry Symbol %%%%
\newcommand{\degree}{^\circ}
\newcommand{\Arc}[1]{\wideparen{{#1}}}
\newcommand{\Line}[1]{\overleftrightarrow{{#1}}}
\newcommand{\Ray}[1]{\overrightarrow{{#1}}}
\newcommand{\Segment}[1]{\overline{{#1}}}

%%%% SI unit short cut %%%%
\newcommand{\siua}{\micro\ampere}
\newcommand{\sima}{\milli\ampere}
\newcommand{\simv}{\milli\volt}
\newcommand{\siko}{\kilo\ohm}
\newcommand{\sio}{\ohm}
\newcommand{\sia}{\ampere}
\newcommand{\siv}{\volt}

\newcommand{\img}{\mathrm{i}}
\newcommand{\ex}{\mathsf{e}}
\newcommand{\dD}{\mathrm{d}}
\newcommand{\iD}{\:\mathrm{d}}

%%%%%%%%%%%%%%%證明、結論、定義等等的環境%%%%%%%%%%%%%%%
\renewcommand{\proofname}{\bf 證明:} %修改Proof 標頭
\newtheoremstyle{mystyle}% 自定義Style
  {6pt}{15pt}%       上下間距
  {}%               內文字體
  {}%               縮排
  {\bf}%            標頭字體
  {.}%              標頭後標點
  {1em}%            內文與標頭距離
  {}%               Theorem head spec (can be left empty, meaning 'normal')

% 改用粗體,預設 remark style 是斜體
\theoremstyle{mystyle}	% 定理環境Style
\newtheorem{theorem}{Thm}
\newtheorem{definition}{def}
\newtheorem{formula}{公式}
\newtheorem{condition}{條件}
\newtheorem{supposition}{假設}
\newtheorem{conclusion}{結論}
\newtheorem{lemma}{引理}
\newtheorem{property}{性質}

%% Label set %%
\captionsetup[figure]{labelsep=period}

%% Ans %%
\newcommand{\Ans}{\noindent{\bf Ans:}}

%% No Indent %%
\setlength\parindent{0pt}

\newcommand{\red}{\color{red}}
\newcommand{\blue}{\color{blue}}

\newcommand{\paral}{\mathbin{\|}}
\newcommand{\qRq}{\quad \Rightarrow \quad}
\newcommand\numeq{\addtocounter{equation}{1}\tag{\theequation}}


%%%%%%%%%%%%%%%Title的資訊%%%%%%%%%%%%%%%
\title{} %標題
\author{} %作者
\date{} %日期

\begin{document}
\tikzstyle{default}=[thick, color=black]
% \maketitle %製作tilte page
% \thispagestyle{empty}  %去除頁碼
% \thispagestyle{fancy}  %使用fancyhdr
% \tableofcontents %目錄
%%%%%%%%%%%%%%%%%%%include file here%%%%%%%%%%%%%%%%%%%%%%%%%

\section{8.3}

The NMOS transistor in the discrete CS amplifier circuit of Fig
is biased to have $g_m = \SI{1}{\mA/\V}$. Find
$A_M, f_{P1}, f_{P2}, f_{P3}, f_L$.

\Ans \\

$R_{\text{sig}} = \SI{100}{\kohm},
R_G = \SI{47}{\Mohm} \paral \SI{10}{\Mohm}
\approx \SI{8.25}{\Mohm}, R_D = \SI{4.7}{\kohm}, R_L = \SI{10}{\kohm},
g_m = \SI{1}{\mA/\V}$,
\[ A_M = \frac{R_G}{R_{\text{sig}} + R_G}g_m(R_D \paral R_L)
\approx 3.16 (\si{\V/\V}). \]

$C_1 = \SI{0.01}{\micro\F}, C_S = \SI{10}{\micro\F}, C_2 = \SI{0.1}{\micro\F}$,
let $r_m = 1/g_m = \SI{1}{\kohm}, R_S = \SI{2}{\kohm}$,
\begin{alignat*}{3}
  f_{P1} &= \frac{1}{2\pi C_1(R_{\text{sig}}+R_G)} &&\approx \SI{1.9}{\Hz}, \\
  f_{P2} &= \frac{1}{2\pi C_S(r_m \paral R_S)} &&\approx \SI{23.9}{\Hz}, \\
  f_{P3} &= \frac{1}{2\pi C_1(R_{\text{sig}}+R_G)} &&\approx \SI{108.3}{\Hz}.
\end{alignat*}
So $f_L = f_{P3} = \SI{108.3}{\Hz}$.

\section{8.16}
Starting from the expression for the MOSTFET unity-gain frequency,
\[ f_T = \frac{g_m}{2\pi (C_{gs}+C_{gd})} \]
and making the approximation that $C_{gs} \gg C_{gd}$ and that the
overlap component of $C_{gs}$ is negligibly small, show that for an
n-channel device
\[ f_T \approx \frac{3\mu_nV_{OV}}{4\pi L^2} \]
Observe that for a given channel length, $f_T$ can be increased by
operating the MOSFET at a higher overdrive voltage. Evaluate $f_T$ for
devices with $L = \SI{1.0}{\um}$ operated at overdrive voltages of
\SI{0.25}{\V} and \SI{0.5}{\V}. Use $\mu_n = \SI{450}{\cm^2/\V}$s.

\Ans \\
\[ f_T = \frac{g_m}{2\pi (C_{gs}+C_{gd})} \]
Now let $C_{gs} \gg C_{gd}$, we know $g_m = k_nV_{OV} = \mu_nC_{ox}(W/L)$ and
$C_{gs} = \frac{2}{3}WLC_{ox}$, so
\[ f_T \approx \frac{\mu_n C_{ox}(W/L)V_{OV}}{2\pi \frac{2}{3}WLC_{ox}}
= \frac{3\mu_nV_{OV}}{4\pi L^2} \]
$f_T$ will increase as $V_{OV}$ inscreases.

Now $L = \SI{1}{\um}, \mu_n = \SI{450}{\cm^2/\V}$, we can evaluate $f_T$:
\begin{alignat*}{2}
  V_{OV} &= \SI{0.25}{\V} & \qRq f_T &\approx \SI{2.7}{\GHz} \\
  V_{OV} &= \SI{0.5}{\V} & \qRq f_T &\approx \SI{5.4}{\GHz}
\end{alignat*}

\section{8.30}
The analysis of the high-frequency response of the common-source amplifier,
presented in the text, is based on the assumption that the resistance of
the signal source, $R_{\text{sig}}$, is large and, thus, that its interaction
with the input capacitance $C_{\text{in}}$ produces the ``dominat pole''
that determines the upper 3-dB frequency $f_H$. In some situations, however,
the CS amplifier is fed with a very low $R_{\text{sig}}$. To investigate the
high-frequency response of the amplifier in such a case, Fig shows the
equivalent circuit when the CS amplifier is fed with an ideal voltage source
$V_{\text{sig}}$ having $R_{\text{sig}} = 0$. Note that $C_L$ denotes the
total capacitance at the output node. By writing a node equation at the
output, show that the transfer function $V_o/V_{\text{sig}}$ is given by
\[ \frac{V_o}{V_{\text{sig}}} =
-g_mR_L' \frac{1-s(C_{gd}/g_m)}{1+s(C_L+C_{gd})R_L'} \]
At frequencies $\omega \ll (g_m/C_{gd})$, the $s$ term in the numerator
can be neglected. In such case, what is the upper 3-dB frequency resulting?
Compute the values of $A_M$ and $f_H$ for the case:
$C_{gd} = \SI{0.5}{\pF}, C_L = \SI{2}{\pF}, g_m = \SI{4}{\mA/\V}$, and
$R_L' = \SI{5}{\kohm}$.

\Ans \\
Use node analysis, we have
\[
  V_o \left(sC_{gd} + \frac{1}{R_L'} + sC_L\right)
  = V_{\text{sig}} \left(sC_{gd} - g_m\right),
\]
so
\[ \frac{V_o}{V_{\text{sig}}} =
-g_mR_L' \frac{1-s(C_{gd}/g_m)}{1+s(C_L+C_{gd})R_L'}.
\]
as desired.

If $\omega \ll (g_m/C_{gd})$, we have
\[ \frac{V_o}{V_{\text{sig}}} \approx
-g_mR_L' \frac{1}{1+s(C_L+C_{gd})R_L'}.
\]
So $\omega_H = 1/(C_L+C_{gd})R_L'$, the 3-dB frequency $f_H = \omega_H / 2\pi$.

For $C_{gd} = \SI{0.5}{\pF}, C_L = \SI{2}{\pF}, g_m = \SI{4}{\mA/\V}$, and
$R_L' = \SI{5}{\kohm}$.
\begin{gather*}
  A_{M} = -g_mR_L' = -20 (\si{\V/\V}), \\
  f_H = \frac{1}{2\pi (C_L+C_{gd})R_L'} = \SI{12.7}{\MHz}.
\end{gather*}

\section{8.34}
For a version of the CE amplifier circuit in Fig,
$R_{\text{sig}} = \SI{10}{\kohm}, R_1 = \SI{68}{\kohm}, R_2 = \SI{27}{\kohm},
R_E = \SI{2.2}{\kohm}, R_C = \SI{4.7}{\kohm}, R_L = \SI{10}{\kohm}.$
The collector current is $\SI{0.8}{\mA}, \beta = 200, f_T = \SI{1}{\GHz},
C_{\mu} = \SI{0.8}{\pF}$. Neglecting the effect of $r_x$ and $r_o$,
find the midband voltage gain and the upper 3-dB frequency $f_H$.

\Ans \\
First we calculate $g_m = I_C / V_T = \SI{32}{\mA/\V}$, so
$r_{\pi} = \beta/g_m = \SI{6.25}{\kohm},
r_e = \alpha/g_m \approx \SI{31.1}{\ohm}$.
Let $R_B = R_1 \paral R_2 = \SI{19.33}{\kohm}$, then
\[
  A_M = -\frac{R_B \paral r_{\pi}}{R_{\text{sig}}+(R_B\paral r_{\pi})}
  \frac{R_C \paral R_L}{r_e}
  \approx -32.8 (\si{\V/\V}).
\]
Now we know that
\[ f_H = \frac{\omega_H}{2\pi}
= \frac{1}{2\pi C_{\text{in}}R_{\text{sig}}'} \]
where
\begin{gather*}
  C_{\text{in}} = C_{\pi} + C_{\mu}(1+g_mR_L') \\
  R_{\text{sig}}' = R_{\text{sig}} \paral R_B \paral r_{\pi} \\
  R_{L}' = R_C \paral R_L
\end{gather*}
and $C_{\pi} + C_{\mu} = g_m/2\pi f_T$, now use the values given we can get
\[ f_H \approx \SI{571}{\kHz}. \]

\section{8.48}
A common-source MOS amplifier, whose equivalent circuit resembles that
in Fig, is to be evaluated for its high-frequency response. For this
particular design, $R_{\text{sig}} = \SI{1}{\Mohm}, R_G = \SI{5}{\Mohm},
R_L' = \SI{100}{\kohm}, C_{gs} = \SI{0.2}{\pF}, C_{gd} = \SI{0.1}{\pF},
g_m = \SI{0.3}{\mA/\V}$. Estimate the midband gain and the 3-dB frequency.

\Ans \\
To estimate the midband gain $A_M$, we need to assume that the capacitors
in this equivalent circuit are perfect open circuits. Then we have
\[
  A_M = \frac{V_o}{V_\text{sig}}
  = \frac{-g_mV_{gs}R_L'}{V_\text{sig}}
  = -g_mR_L' \frac{R_G}{R_G+R_\text{sig}}
  = -25 (\si{\V/\V}).
\]
To estimate the 3-dB frequency, the resistance seen by
$C_{gs}$ is $R_{gs} = R_G \paral R_\text{sig}$, the resistance
seen by $C_{gd}$ is $R_{gs} + R_L' + g_mR_{gs}R_L'$. So
\begin{gather*}
  \tau_{gs} = C_{gs}R_{gs}, \\
  \tau_{gd} = C_{gd}R_{gd}, \\
  f_H = \frac{\omega_H}{2\pi} = \frac{1}{2\pi (\tau_{gs}+\tau_{gd})}
  \approx \SI{57.7}{\kHz}.
\end{gather*}

\section{8.63}
It is required to analyze the high-frequency response of the CMOS amplifier
shown in Fig. The dc bias current is \SI{100}{\uA}.
For $Q_1$, $\mu_nC_{ox} = \SI{90}{\uA/\V^2}, V_A = \SI{12.8}{\V},
W/L = \SI{100}{\um}/\SI{1.6}{\um}, C_{gs} = \SI{0.2}{\pF},
C_{gd} = \SI{0.015}{\pF}, C_{db} = \SI{20}{\fF}$. For $Q_2$,
$C_{gd} = \SI{0.015}{\pF}$, $C_{db} = \SI{36}{\fF}, \abs{V_A} = \SI{19.2}{\V}$.
Assume that the resistance of the input signal generator is negligibly small.
Also, for simplicity, assume that the signal voltage at the gate of $Q_2$
is zero. Find the low-frequency gain, the frequency of the pole, and the
frequency of the zero.

\Ans \\
By using node analysis, we can write
\[
  \frac{V_o}{V_\text{sig}} =
  \frac{sC_{gd1} - g_{m1}}{s(C_{gd1}+C_{db1}+C_{gd2}+C_{db2}) +
  \frac{1}{r_{o1}}+\frac{1}{r_{o2}}}
\]
Now $I_{D1} = I_{D2} = \SI{0.1}{\mA}, r_o = V_A/I_D,
g_m = k_nV_{OV} = I_D/V_{OV} = \sqrt{2k_nI_D}$. We can evaluate
$g_m = \SI{1.06}{\mA/\V}, r_{o1} = \SI{128}{\kohm}, r_{o2} = \SI{192}{\kohm}$.
The low-frequency gain is
$-g_m(r_{o1} \paral r_{o2})\approx -81.4(\si{\V/\V})$. The frequency of pole
and zero are
\begin{alignat*}{3}
  f_P = \frac{\omega_P}{2\pi} &=
  -\frac{1/r_{o1}+1/r_{o2}}{2\pi (C_{gd1}+C_{db1}+C_{gd2}+C_{db2})}
  && \approx & -\SI{24.1}{\MHz} \\
  f_Z = \frac{\omega_Z}{2\pi} &= \frac{g_m}{2\pi C_{gd1}}
  && \approx & \SI{11.3}{\GHz}
\end{alignat*}

\section{8.67}
Consider a CS amplifier loaded in a current source with an output resistance
equal to $r_o$ of the amplifying transistor. the amplifier is fed from a
signal source with $R_{\text{sig}} = r_o/2$. The transistor is biased to
operate at $g_m = \SI{2}{\mA/\V}$ and $r_o = \SI{20}{\kohm}$;
$C_{gs} = C_{gd} = \SI{0.1}{\pF}$. Use the Miller approximation to
determine an estimate of $f_H$. Repeat for the follwing two cases:
\begin{enumerate*}[(i)]
  \item the bias current $I$ in the entire system is reduced by a factor of
    $4$, and
  \item the bias current $I$ in the entire system is increased by a factor of
    $4$.
\end{enumerate*}
Remember that both $R_{\text{sig}}$ and $R_L$ will change as $r_o$ changes.

\Ans \\
$R_L = r_o \paral r_o = \SI{10}{\kohm}$.
Use Miller approximation, let $C_\text{in}=C_{gs}+C_{gd}(1-K),
C_L=C_{gd}(1-1/K)$. Now $K = -g_mR_L = -20$, we can calculate poles:
\begin{gather*}
  f_{Pi} = \frac{1}{2\pi C_\text{in} R_\text{sig}} \\
  f_{Po} = \frac{1}{2\pi C_L R_L}
\end{gather*}
By approximation we assume $f_H \approx f_{Pi}$,
so $f_H \approx \SI{7.23}{\MHz}$.

Now we know $g_m = \sqrt{2k_nI_D}, r_o = V_A/I_D$. If current $I$ is
reduced by a factor of $4$, $r_o$ will be increased by a factor of $4$,
$g_m$ will be decreased by a factor of $2$. Then
\[
  f_H \approx f_{Pi} = \frac{1}{2\pi (C_{gs}+C_{gd}(1+40)) 4R_\text{sig}}
  \approx \SI{947}{\kHz}
\]
If current $I$ is increased by a factor of $4$, $r_o$ will be decreased by
a factor of $4$, $g_m$ will be increased by a factor of $2$. Then
\[
  f_H \approx f_{Pi} = \frac{1}{2\pi (C_{gs}+C_{gd}(1+10)) R_\text{sig}/4}
  \approx \SI{53}{\MHz}
\]

\section{8.72}
Find the dc gain and the 3-dB frequency of a MOS cascode amplifier operated
at $g_m = \SI{1}{\mA/\V}$ and $r_o = \SI{50}{\kohm}$. The MOSFETs have
$C_{gs} = \SI{30}{\fF}, C_{gd} = \SI{10}{\fF}, C_{db} = \SI{10}{\fF}$.
The amplifier is fed from a signal source with
$R_{\text{sig}} = \SI{100}{\kohm}$ and is connected to a load resistance of
\SI{2}{\Mohm}. There is also a load capacitance $C_L$ of \SI{40}{\fF}.

\Ans \\
Let $R_L = \SI{2}{\Mohm}$, $R_o = r_{o2} + r_{o1} + g_mr_{o2}r_{o1}$.
The dc gain is $-g_m (R_o \paral R_L) = -1130 (\si{\V/\V})$.

For 3-dB frequency, first we calculate the time constants:
\begin{align*}
  \tau_{H} &= R_\text{sig} (C_{gs1} + C_{gd1}(1+g_{m1}R_{d1}))
  + R_{d1}(C_{gd1} + C_{db1} + C_{gs2}) \\
  & \quad + (R_L \paral R_o) (C_{gd2} + C_{db2} + C_L)
\end{align*}
where $R_{d1} = r_{o1} \paral R_\text{in2}, R_\text{in2} =
\frac{r_{o2}+R_L}{1+g_{m2}r_{o2}}$.
And 3-dB frequency is
\[
  f_H \approx \frac{1}{2\pi \tau_H} \approx \SI{1.67}{\MHz}.
\]

\section{8.79}
Refer to Fig. In situations in which $R_{\text{sig}}$ is large, the
high-frequency response of the source follower is determined by the low-pass
circuit formed by $R_{\text{sig}}$ and the input capacitance. An estimate
of $C_{\text{in}}$ can be obtained by using the Miller approximation to
replace $C_{gs}$ with an input capacitance $C_{eq} = C_{gs}(1-K)$ where $K$
is the gain from gate to source. Using the low-frequency value of
$K = g_mR_L'/(1+g_mR_L')$ find $C_{eq}$ and hence $C_{\text{in}}$ and an
estimate of $f_H$. Is this estimate higher or lower than that obtained by the
method of open-circuit time constants?

\section{8.88}
Consider the active-loaded CMOS differential amplifier of Fig for
the case of all transistors operated at the same $\abs{V_{OV}}$ and having
the same $\abs{V_A}$. Also let the total capacitance at the output node
$(C_L)$ be four times the total capacitance at the input node of the
current mirror $C_m$, and show that the unity-gain frequency of $A_d$ is
$g_m/2\pi C_L$. For $V_A=\SI{20}{\V}, V_{OV}=\SI{0.2}{\V}, I=\SI{0.2}{\mA},
C_L=\SI{100}{\fF},C_m=\SI{25}{\fF}$, find the dc value of $A_d$, and the
value of $f_{P1}, f_t, f_{P2}, f_Z$ and sketch a Bode plot for $\abs{A_d}$.


\end{document}
